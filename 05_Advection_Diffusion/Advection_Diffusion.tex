\documentclass{tufte-handout}\usepackage[]{graphicx}\usepackage[]{xcolor}
% maxwidth is the original width if it is less than linewidth
% otherwise use linewidth (to make sure the graphics do not exceed the margin)
\makeatletter
\def\maxwidth{ %
  \ifdim\Gin@nat@width>\linewidth
    \linewidth
  \else
    \Gin@nat@width
  \fi
}
\makeatother

\definecolor{fgcolor}{rgb}{0.345, 0.345, 0.345}
\newcommand{\hlnum}[1]{\textcolor[rgb]{0.686,0.059,0.569}{#1}}%
\newcommand{\hlstr}[1]{\textcolor[rgb]{0.192,0.494,0.8}{#1}}%
\newcommand{\hlcom}[1]{\textcolor[rgb]{0.678,0.584,0.686}{\textit{#1}}}%
\newcommand{\hlopt}[1]{\textcolor[rgb]{0,0,0}{#1}}%
\newcommand{\hlstd}[1]{\textcolor[rgb]{0.345,0.345,0.345}{#1}}%
\newcommand{\hlkwa}[1]{\textcolor[rgb]{0.161,0.373,0.58}{\textbf{#1}}}%
\newcommand{\hlkwb}[1]{\textcolor[rgb]{0.69,0.353,0.396}{#1}}%
\newcommand{\hlkwc}[1]{\textcolor[rgb]{0.333,0.667,0.333}{#1}}%
\newcommand{\hlkwd}[1]{\textcolor[rgb]{0.737,0.353,0.396}{\textbf{#1}}}%
\let\hlipl\hlkwb

\usepackage{framed}
\makeatletter
\newenvironment{kframe}{%
 \def\at@end@of@kframe{}%
 \ifinner\ifhmode%
  \def\at@end@of@kframe{\end{minipage}}%
  \begin{minipage}{\columnwidth}%
 \fi\fi%
 \def\FrameCommand##1{\hskip\@totalleftmargin \hskip-\fboxsep
 \colorbox{shadecolor}{##1}\hskip-\fboxsep
     % There is no \\@totalrightmargin, so:
     \hskip-\linewidth \hskip-\@totalleftmargin \hskip\columnwidth}%
 \MakeFramed {\advance\hsize-\width
   \@totalleftmargin\z@ \linewidth\hsize
   \@setminipage}}%
 {\par\unskip\endMakeFramed%
 \at@end@of@kframe}
\makeatother

\definecolor{shadecolor}{rgb}{.97, .97, .97}
\definecolor{messagecolor}{rgb}{0, 0, 0}
\definecolor{warningcolor}{rgb}{1, 0, 1}
\definecolor{errorcolor}{rgb}{1, 0, 0}
\newenvironment{knitrout}{}{} % an empty environment to be redefined in TeX

\usepackage{alltt}

%\geometry{showframe}% for debugging purposes -- displays the margins

\usepackage{amsmath}
\usepackage{graphicx}
\usepackage{natbib}

\setkeys{Gin}{width=\linewidth,totalheight=\textheight,keepaspectratio}
\bibfont{\small} % Doesn't see to work...

\graphicspath{{graphics/}}

\newcommand{\numolspercm}{$\nu$mols/cm$^{-3}$}

\title{DRAFT ! Advection, Diffusion \& Reaction Modeling}
\author{Marc Los Huertos}
\date{\today~(ver. 0.47 Wait to read until I get to v .8!)}

\setsidenotefont{\color{blue}}
% \setcaptionfont{hfont commandsi}
% \setmarginnotefont{\color{blue}}
% \setcitationfont{\color{gray}}

% The following package makes prettier tables.  We're all about the bling!
\usepackage{booktabs}

% Small sections of multiple columns
\usepackage{multicol}
\IfFileExists{upquote.sty}{\usepackage{upquote}}{}
\begin{document}

\maketitle% this prints the handout title, author, and date
\begin{abstract}
The movement of compounds in the environment is driven by two processes, advection and diffusion. Of course, these processes occur in three dimensions, but for this class we'll begin with one dimensional processes before getting to more complicated examples. The compounds are also subject to transformations or reactions. Thus, to understand the fate and transport of compounds in the environment, we need to understand the processes of advection, diffusion, and reaction. We'll begin with simple examples and then consider more complicated examples. We'll also consider the implications of these processes on the environment and human health.

%\sidenote{Typeset using the Sweave function in R and \LaTeX\ using a \citet{Tufte:1983, Tufte:1997} and style.}
\end{abstract}

\section{Introduction}

\subsection{The Problem}

The fate and transport of compounds in the environment is subject to a complex array of processes (Figure~\ref{fig:dioxaneplume}). However, we can simplify these processes into three subprocesses: 
advection, diffusion, and reaction. Of course, these processes occur in three dimensions. 

\begin{figure*}
\centering
\includegraphics[width=0.8\textwidth]{graphics/Dioxane_plume.png}
\caption{A diagram of the major processes that influence the fate and transport of a dioxane plume from a ce in the environment (Source: \url{https://14d-1.itrcweb.org/environmental-fate-transport-and-investigative-strategies/}). 1,4-Dioxane is often referred to as a "forever" compound.}
\label{fig:dioxaneplume}
\end{figure*}

The implications of these three processes is profound -- they provide the basic framework to understand the fate and transport of pollutants in the environment, the movement of nutrients in the soil, and the movement of solutes in the human body. Our understanding of these processes is critical to understanding issues of environmental quality and justice for human and non-humans alike. Moreover, we often use our understanding of these processes to develop of environmental policy, regulation, and mitigation.

Because these processes are complex and are often difficult to measure, we often rely on models to help us understand the movement of solutes in the environment. These models are based on the fundamental mathemicial equations to describe advection, diffusion, and reaction.

\subsection{The Processes}

Advection and diffusion are two fundamental processes that govern the transport of solutes in the environment. Advection is the process of transport of a solute by the bulk motion of the fluid. Diffusion is the process of transport of a solute by random molecular motion.

\begin{figure}[h]
\centering
\includegraphics[width=0.9\textwidth]{graphics/Diagram-advection-diffusion.png}
\caption{A simple diagram of advection and diffusion that inclues how "solutes" might be deposited downwind of a stationary source if air pollution. At the scale analyzed there, the term turbulant diffusion is different than molecular diffusion, but might be modelled in a similar way.}
\label{fig:advection-diffusion}
\end{figure}

In a simple example, if we have a dose of a solute, like a dye in the center of some media, we'll see it spread out by diffusion (Figure~\ref{fig:diffusion}). If the media is moving, we'll see the dye move with the bulk motion of the media. If the dye reacts with the media, we'll see the dye disappear.

\begin{marginfigure}
\caption{A simple diagram of 2D diffusion.}
\label{fig:diffusion}
\includegraphics[width=1.0\textwidth]{graphics/2D_diffusion.png}
\end{marginfigure}


Besides examples we often see with respect to air pollution sources (Figure~\ref{fig:advection-diffusion}), we might think about solutes as pollutants or nutrients. For example, the movement of a nutrient in a river is driven by the bulk motion of the water (advection) and the random motion of the molecules (diffusion) and the reactions that might occur in the water column and sediments (Figure\ref{fig:nutrientspiraling}).

\begin{figure}[h]
\caption{A simple diagram of advection and nutrient reactions (organic substances and inorganic substances) in a river. Although not shown, you might also think about how diffusion might influence the movement of nutrients in the river and in the sediments.  
The zone where water moves into the sediment bed is called the hyperreic zone. The porosity of the sediments will allow more advective flow that might influence the reaction capacity of solutes in the sediments.}
\label{fig:nutrientspiraling}
\centering
\includegraphics[width=0.7\textwidth]{graphics/NutrientSpiraling.jpg}
\end{figure}


\subsection{Session Goals}

We will not become experts in advection-diffusion-reaction modeling, but we will become familiar with the processes and the equations that describe them. Moreover, we'll see a bit more about how R can be used to model these processes. After this session, I hope you can do the following:

\begin{enumerate}
	\item Describe the physical processes of advection and diffusion and solute reaction
	\item Describe the equations used to model A-D-R. 
	\item Analyze 1-dimensional movement using advection equations in R.
	\item Describe diffusion mathematically
	\item Analyze 1-dimensional adveecton-diffusion using R.
	\item Appreciate how two-dimensional analysis of advection-diffusion can be modeled in R.
\end{enumerate}

In this session, we'll want to think about the movement of solutes in the porous media, i.e. a soil with air space, sediments with water between the particles. We will refer to the porousity as $\xi$, which is a proportion between 0 and 1. 

\begin{marginfigure}
\centering
\includegraphics[width=0.8\textwidth]{graphics/Darcy_permeability.jpg}
\end{marginfigure}

%We'll start with one-dimensional systems and then move to two-dimensional systems. We'll also consider the effects of reactions on the movement of solutes.

\section{The Processes and the Equations to Describe Them}

\subsection{Advection and Convection: Material and Heat}

Advection is the transport of a substance by bulk motion. Convection is the transfer of heat by the actual movement of the warmed matter. The equation that are used to describe advection and convection are similar, but the physical processes are different.


\newthought{An Equation to Create Anxiety}


The advection equation is a partial differential equation that describes the movement of a substance in a fluid. The equation is derived from the conservation of mass. The equation is:

%− 1 Axξx ·( ∂ ∂xAx·(−D· ∂ξxC ∂x )− ∂ ∂x(Ax·v·ξxC)) 

\begin{equation}
- \frac{1}{A_x \xi_x} \cdot \left( \frac{\partial}{\partial x} A_x \cdot \left( -D \cdot \frac{\partial \xi_x C}{\partial x} \right) - \frac{\partial}{\partial x} \left( A_x \cdot v \cdot \xi_x C \right) \right)
\end{equation}

Here $D$ is the ``diffusion coefficient'', $v$ is the ``advection rate'', and $A_x$ and $\xi$ are the surface area and volume fraction, respectively. 

Assuming that A, $\xi$, D and v are constant along x, we can rewrite this in a more general form: 

%D∂2C ∂x2−u∂C ∂x
\begin{equation}
D \frac{\partial^2 C}{\partial x^2} - \nu \frac{\partial C}{\partial x}
\end{equation}

where $u = v/A_x \xi_x$ is the ``velocity'' of the fluid.

The movement of compounds in the environment is driven by two processes, advection and diffusion. Of course, these processes occur in three dimensions, but for this class we'll begin with one dimensional processes before getting to more complicated examples.

Nevertheless, let's look at the 3-D advection-diffusion-reacton equation in three dimensions:

\begin{equation}
\frac{\partial C}{\partial t} = \nabla \cdot (D\nabla C - \nu C) + R
\end{equation}

where $C$ is the concentration of the solute, $t$ is time, $\nu$ is the velocity of the fluid, $D$ is the diffusion coefficient, and $R$ is the reaction term.

Ok, what what is $\nabla$? It's the gradient operator, which is a vector operator that operates on a scalar function to produce a vector whose magnitude is the maximum rate of change of the function at the point of the gradient and that points in the direction of that maximum rate of change. $\nabla \cdot$ represents divergence. In this equation, $\nabla C$ represents concentration gradient.

\subsection{Advection}

Advection is the process of transport of a solute by the bulk motion of the fluid. The rate of advection is proportional to the velocity of the fluid and the concentration of the solute. The rate of advection is given by the equation:

\begin{equation}
\frac{\partial C}{\partial t} + \nabla \cdot (\nu C) = 0
\end{equation}

where $C$ is the concentration of the solute, $t$ is time, and $\nu$ is the velocity of the fluid. 

For one dimensional systems, the equation can be written as:

\begin{equation}
\frac{\partial C}{\partial t} + u \frac{\partial C}{\partial x} = 0
\end{equation}

or

\begin{equation}
\frac{\partial C}{\partial t} + \nu_x \frac{\partial C}{\partial x}
\end{equation}

where $C$ is the concentration of the solute, $t$ is time, $\nu$ is the velocity of the fluid, and $x$ is the spatial coordinate.

The advection equation is not simple to solve numerically: the system is a hyperbolic partial differential equation, and interest typically centers on discontinuous "shock" solutions (which are notoriously difficult for numerical schemes to handle).


\subsection{Diffusion}

Diffusion is the process of transport of a solute by random molecular motion. The rate of diffusion is proportional to the concentration gradient of the solute. The rate of diffusion is given by the equation:

\begin{equation}
\frac{\partial C}{\partial t} = D \nabla^2 C
\end{equation}

where $D$ is the diffusion coefficient.

\subsection{Advection-Diffusion Equation}

The advection-diffusion equation is a combination of the advection and diffusion equations. The advection-diffusion equation is given by the equation:

\begin{equation}
\frac{\partial C}{\partial t} + \nabla \cdot (\nu C) = D \nabla^2 C
\end{equation}

or 

\begin{equation}
\frac{\partial C}{\partial t} + u \frac{\partial C}{\partial x} = D \frac{\partial^2 C}{\partial x^2}
\end{equation}

where $C$ is the concentration of the solute, $t$ is time, $\nu$ is the velocity of the fluid, and $D$ is the diffusion coefficient.

\subsection{Advection-Difffusion-Reaction Equation}

The advection-diffusion-reaction equation is a combination the advection, diffusion, and reaction equations. The advection-diffusion-reaction equation is given by the equation:

\begin{equation}
\frac{\partial C}{\partial t} + \nabla \cdot (\nu C) = D \nabla^2 C + R
\end{equation}

where $C$ is the concentration of the solute, $t$ is time, $\nu$ is the velocity of the fluid, $D$ is the diffusion coefficient, and $R$ is the reaction term.

or in one-dimension:

\begin{equation}
\frac{\partial C}{\partial t} = \nu \frac{\partial C}{\partial x} = D \frac{\partial^2 C}{\partial x^2} + R
\end{equation}

where $C$ is the concentration of the solute, $t$ is time, $\nu$ is the velocity of the fluid, $D$ is the diffusion coefficient, and $R$ is the reaction term.

  
\subsection{Advenction-Difffusion-Reaction in multi-phase systems and for shapes with variable geometry}

The advection-diffusion-reaction equation can be extended to multi-phase systems and to shapes with variable geometry. The advection-diffusion-reaction equation for multi-phase systems and for shapes with variable geometry is given by the equation:

\begin{equation}
\frac{\partial C}{\partial t} + \nabla \cdot (\nu C) = D \nabla^2 C + R
\end{equation}

where $C$ is the concentration of the solute, $t$ is time, $\nu$ is the velocity of the fluid, $D$ is the diffusion coefficient, and $R$ is the reaction term.
  
\section{Applications using R}

\subsection{ReacTran Package}

\citet{soetaert2017package} have developed a nice library in R that solves these equations using finite-difference solutions. 

The ReacTran package is a collection of functions for modeling solute transport in 1D, 2D, and 3D. The package includes functions for solving the advection-diffusion equation, the advection-diffusion-reaction equation, and the advection-diffusion-reaction equation for multi-phase systems. 

The package also includes functions for solving

\begin{itemize}
\item Advection-diffusion equations in 1D, 2D, and 3D 
\item Advection-diffusion-reaction equations in 1D, 2D, and 3D
\item Advection-diffusion-reaction equations for multi-phase systems
\end{itemize}


\subsection{Using R as a Modelling Environment to Solve PDEs and ODEs}

We can use R to solve the advection-diffusion equation. The `deSolve` package to solve the ReacTran library functions decomposing the partial differential equatation (PDE) into a descretized by space and solving be ordinary differential equations (ODE). 

\begin{knitrout}
\definecolor{shadecolor}{rgb}{0.969, 0.969, 0.969}\color{fgcolor}\begin{kframe}
\begin{alltt}
\hlkwd{library}\hlstd{(ReacTran)} \hlcom{# Load the ReacTran package}
\end{alltt}


{\ttfamily\noindent\itshape\color{messagecolor}{\#\# Loading required package: rootSolve}}

{\ttfamily\noindent\itshape\color{messagecolor}{\#\# Loading required package: deSolve}}

{\ttfamily\noindent\itshape\color{messagecolor}{\#\# Loading required package: shape}}\begin{alltt}
\hlkwd{library}\hlstd{(deSolve)} \hlcom{# Load the deSolve package}
\end{alltt}
\end{kframe}
\end{knitrout}

\subsection{1D Transportion Model}

The `ReacTran` package provides a function to solve the advection-diffusion-reaction equation for a simple one-dimensional case.

\begin{knitrout}
\definecolor{shadecolor}{rgb}{0.969, 0.969, 0.969}\color{fgcolor}\begin{kframe}
\begin{alltt}
\hlkwd{tran.1D}\hlstd{(}\hlkwc{C} \hlstd{=} \hlnum{1}\hlstd{,} \hlkwc{D} \hlstd{=} \hlnum{0}\hlstd{,} \hlkwc{flux.up} \hlstd{=} \hlnum{1}\hlstd{,} \hlkwc{v} \hlstd{=} \hlnum{5}\hlstd{,} \hlkwc{A}\hlstd{=} \hlnum{1}\hlstd{,} \hlkwc{dx} \hlstd{=} \hlnum{1}\hlstd{,} \hlkwc{full.output} \hlstd{=} \hlnum{TRUE}\hlstd{)}
\end{alltt}
\begin{verbatim}
## $dC
## [1] -4
## 
## $C.up
## [1] 0.2
## 
## $C.down
## [1] 1
## 
## $dif.flux
## [1] 0 0
## 
## $adv.flux
## [1] 1 5
## 
## $flux
## [1] 1 5
## 
## $flux.up
## [1] 1
## 
## $flux.down
## [1] 5
\end{verbatim}
\end{kframe}
\end{knitrout}


\subsection{Solving a 1-D reaction tranport mdodel}

\begin{knitrout}
\definecolor{shadecolor}{rgb}{0.969, 0.969, 0.969}\color{fgcolor}\begin{kframe}
\begin{alltt}
\hlkwd{library}\hlstd{(ReacTran)}
\hlstd{out} \hlkwb{<-} \hlkwd{steady.1D}\hlstd{(}\hlkwc{func} \hlstd{= advModel,} \hlkwc{y} \hlstd{=} \hlkwd{runif}\hlstd{(}\hlnum{25}\hlstd{),} \hlkwc{params} \hlstd{= parms,} \hlkwc{nspace}\hlstd{=} \hlnum{1}\hlstd{,} \hlkwc{positive} \hlstd{=} \hlnum{TRUE}\hlstd{)}
\end{alltt}
\end{kframe}
\end{knitrout}

We can use R to solve the advection-diffusion-reaction equation. The following code uses the `deSolve` package to solve the advection-diffusion-reaction equation for a simple one-dimensional case.

\begin{knitrout}
\definecolor{shadecolor}{rgb}{0.969, 0.969, 0.969}\color{fgcolor}\begin{kframe}
\begin{alltt}
\hlcom{# Load the deSolve package}
\hlkwd{library}\hlstd{(deSolve)}

\hlcom{# Define the advection-diffusion-reaction equation}
\hlstd{advection_diffusion_reaction} \hlkwb{<-} \hlkwa{function}\hlstd{(}\hlkwc{t}\hlstd{,} \hlkwc{C}\hlstd{,} \hlkwc{parms}\hlstd{) \{}
  \hlkwd{with}\hlstd{(}\hlkwd{as.list}\hlstd{(parms), \{}
    \hlstd{dC} \hlkwb{<-} \hlstd{D} \hlopt{*} \hlstd{(}\hlkwd{diff}\hlstd{(C,} \hlkwc{lag} \hlstd{=} \hlnum{2}\hlstd{)} \hlopt{-} \hlnum{2} \hlopt{*} \hlkwd{diff}\hlstd{(C,} \hlkwc{lag} \hlstd{=} \hlnum{1}\hlstd{)} \hlopt{+} \hlkwd{diff}\hlstd{(C,} \hlkwc{lag} \hlstd{=} \hlnum{0}\hlstd{))} \hlopt{/} \hlstd{dx}\hlopt{^}\hlnum{2} \hlopt{-} \hlstd{k} \hlopt{*} \hlstd{C}
    \hlstd{dC[}\hlnum{1}\hlstd{]} \hlkwb{<-} \hlnum{0}
    \hlstd{dC[n]} \hlkwb{<-} \hlnum{0}
    \hlkwd{list}\hlstd{(dC)}
  \hlstd{\})}
\hlstd{\}}

\hlcom{# Set the parameters}
\hlstd{parms} \hlkwb{<-} \hlkwd{list}\hlstd{(}
  \hlkwc{D} \hlstd{=} \hlnum{0.1}\hlstd{,}  \hlcom{# Diffusion coefficient}
  \hlkwc{dx} \hlstd{=} \hlnum{0.1}\hlstd{,}  \hlcom{# Spatial step}
  \hlkwc{k} \hlstd{=} \hlnum{0.01}  \hlcom{# Reaction rate}
\hlstd{)}

\hlcom{# Set the initial conditions}
\hlstd{C0} \hlkwb{<-} \hlkwd{c}\hlstd{(}\hlnum{0}\hlstd{,} \hlkwd{rep}\hlstd{(}\hlnum{0}\hlstd{,} \hlnum{98}\hlstd{),} \hlnum{1}\hlstd{,} \hlkwd{rep}\hlstd{(}\hlnum{0}\hlstd{,} \hlnum{98}\hlstd{),} \hlnum{0}\hlstd{)}

\hlcom{# Set the times at which to evaluate the solution}
\hlstd{times} \hlkwb{<-} \hlkwd{seq}\hlstd{(}\hlnum{0}\hlstd{,} \hlnum{100}\hlstd{,} \hlkwc{by} \hlstd{=} \hlnum{1}\hlstd{)}

\hlcom{# Solve the advection-diffusion-reaction equation}
\hlstd{out} \hlkwb{<-} \hlkwd{ode}\hlstd{(}\hlkwc{y} \hlstd{= C0,} \hlkwc{times} \hlstd{= times,} \hlkwc{func} \hlstd{= advection_diffusion_reaction,} \hlkwc{parms} \hlstd{= parms)}

\hlcom{# Plot the solution}
\hlkwd{plot}\hlstd{(out,} \hlkwc{xlab} \hlstd{=} \hlstr{"Distance"}\hlstd{,} \hlkwc{ylab} \hlstd{=} \hlstr{"Concentration"}\hlstd{,} \hlkwc{type} \hlstd{=} \hlstr{"l"}\hlstd{)}
\end{alltt}
\end{kframe}
\end{knitrout}

\subsection{1-D Reaction-Transport Model}
\begin{knitrout}
\definecolor{shadecolor}{rgb}{0.969, 0.969, 0.969}\color{fgcolor}\begin{kframe}
\begin{alltt}
\hlstd{parms} \hlkwb{<-} \hlkwd{c}\hlstd{(}\hlkwc{F0} \hlstd{=} \hlnum{1}\hlstd{,} \hlkwc{v}\hlstd{=}\hlnum{1}\hlstd{,} \hlkwc{k} \hlstd{=} \hlnum{0.1}\hlstd{,} \hlkwc{D} \hlstd{=} \hlnum{0}\hlstd{,} \hlkwc{dx} \hlstd{=} \hlnum{1}\hlstd{)}

\hlstd{advModel} \hlkwb{<-} \hlkwa{function}\hlstd{(}\hlkwc{t}\hlstd{,} \hlkwc{C}\hlstd{,} \hlkwc{parms}\hlstd{) \{}
  \hlkwd{with}\hlstd{(}\hlkwd{as.list}\hlstd{(parms), \{}
    \hlstd{Tran} \hlkwb{<-} \hlkwd{tran.1D}\hlstd{(}\hlkwc{C} \hlstd{= C,} \hlkwc{D} \hlstd{= D,} \hlkwc{flux.up} \hlstd{= F0,} \hlkwc{v} \hlstd{= v,} \hlkwc{dx} \hlstd{= dx)}
    \hlstd{Consumption} \hlkwb{=} \hlstd{k} \hlopt{*} \hlstd{C}
    \hlstd{dC} \hlkwb{<-} \hlstd{Tran}\hlopt{$}\hlstd{dC} \hlopt{-} \hlstd{Consumption}

    \hlkwd{return}\hlstd{(}\hlkwd{list}\hlstd{(}\hlkwc{dC} \hlstd{= dC,} \hlkwc{Consumption} \hlstd{= Consumption,} \hlkwc{flux.up} \hlstd{= Tran}\hlopt{$}\hlstd{flux.up,} \hlkwc{flux.down} \hlstd{= Tran}\hlopt{$}\hlstd{flux.down))}
  \hlstd{\})}
\hlstd{\}}

\hlstd{out} \hlkwb{<-} \hlkwd{steady.1D}\hlstd{(}\hlkwc{func} \hlstd{= advModel,} \hlkwc{y} \hlstd{=} \hlkwd{runif}\hlstd{(}\hlnum{25}\hlstd{),} \hlkwc{parms} \hlstd{= parms,} \hlkwc{nspec}\hlstd{=} \hlnum{1}\hlstd{,} \hlkwc{positive} \hlstd{=} \hlnum{TRUE}\hlstd{)}

\hlstd{parms} \hlkwb{<-} \hlkwd{c}\hlstd{(}\hlkwc{F0} \hlstd{=} \hlnum{1}\hlstd{,} \hlkwc{v}\hlstd{=}\hlnum{1}\hlstd{,} \hlkwc{k} \hlstd{=} \hlnum{0.5}\hlstd{,} \hlkwc{D}\hlstd{=}\hlnum{0}\hlstd{,} \hlkwc{dx} \hlstd{=} \hlnum{1}\hlstd{)}
\hlstd{out2} \hlkwb{<-} \hlkwd{steady.1D}\hlstd{(}\hlkwc{func} \hlstd{= advModel,} \hlkwc{y} \hlstd{=} \hlkwd{runif}\hlstd{(}\hlnum{25}\hlstd{),} \hlkwc{parms} \hlstd{= parms,} \hlkwc{nspec}\hlstd{=} \hlnum{1}\hlstd{,} \hlkwc{positive} \hlstd{=} \hlnum{TRUE}\hlstd{)}

\hlstd{parms} \hlkwb{<-} \hlkwd{c}\hlstd{(}\hlkwc{F0} \hlstd{=} \hlnum{1}\hlstd{,} \hlkwc{v}\hlstd{=}\hlnum{1}\hlstd{,} \hlkwc{k} \hlstd{=} \hlnum{0.5}\hlstd{,} \hlkwc{D}\hlstd{=}\hlnum{50}\hlstd{,} \hlkwc{dx} \hlstd{=} \hlnum{1}\hlstd{)}
\hlstd{out3} \hlkwb{<-} \hlkwd{steady.1D}\hlstd{(}\hlkwc{func} \hlstd{= advModel,} \hlkwc{y} \hlstd{=} \hlkwd{runif}\hlstd{(}\hlnum{25}\hlstd{),} \hlkwc{parms} \hlstd{= parms,} \hlkwc{nspec}\hlstd{=} \hlnum{1}\hlstd{,} \hlkwc{positive} \hlstd{=} \hlnum{TRUE}\hlstd{)}
\end{alltt}
\end{kframe}
\end{knitrout}

We can look at the output, using a simple call of the object, but without more information, it's not clear what we are looking at. 
\begin{knitrout}
\definecolor{shadecolor}{rgb}{0.969, 0.969, 0.969}\color{fgcolor}\begin{kframe}
\begin{alltt}
\hlstd{out}
\end{alltt}
\begin{verbatim}
## $y
##  [1] 0.9090909 0.8264463 0.7513148 0.6830135 0.6209213 0.5644739 0.5131581
##  [8] 0.4665074 0.4240976 0.3855433 0.3504939 0.3186308 0.2896644 0.2633313
## [15] 0.2393921 0.2176291 0.1978447 0.1798588 0.1635080 0.1486436 0.1351306
## [22] 0.1228460 0.1116782 0.1015256 0.0922960
## 
## $Consumption
##  [1] 0.09090909 0.08264463 0.07513148 0.06830135 0.06209213 0.05644739
##  [7] 0.05131581 0.04665074 0.04240976 0.03855433 0.03504939 0.03186308
## [13] 0.02896644 0.02633313 0.02393921 0.02176291 0.01978447 0.01798588
## [19] 0.01635080 0.01486436 0.01351306 0.01228460 0.01116782 0.01015256
## [25] 0.00922960
## 
## $flux.up
## [1] 1
## 
## $flux.down
## [1] 0.092296
## 
## attr(,"precis")
## [1] 4.537303e-01 2.595270e-09
## attr(,"steady")
## [1] TRUE
## attr(,"class")
## [1] "steady1D"  "rootSolve" "list"     
## attr(,"dimens")
## [1] 25
## attr(,"nspec")
## [1] 1
\end{verbatim}
\end{kframe}
\end{knitrout}

Thus, we might be better off plotting the output. I am not sure why the plot functions are ignoring my par() call, perhaps this will be fixed by version 0.9!

\begin{figure*}
\begin{knitrout}
\definecolor{shadecolor}{rgb}{0.969, 0.969, 0.969}\color{fgcolor}\begin{kframe}
\begin{alltt}
\hlkwd{par}\hlstd{(}\hlkwc{mfrow}\hlstd{=}\hlkwd{c}\hlstd{(}\hlnum{1}\hlstd{,}\hlnum{3}\hlstd{))}
\hlkwd{plot}\hlstd{(out,} \hlkwc{xlab} \hlstd{=} \hlstr{"X"}\hlstd{,} \hlkwc{ylab} \hlstd{=} \hlstr{"Concentration"}\hlstd{,} \hlkwc{las}\hlstd{=}\hlnum{1}\hlstd{,} \hlkwc{main}\hlstd{=}\hlstr{"Advection-Reaction: Steady State Solution"}\hlstd{)}
\end{alltt}
\end{kframe}
\includegraphics[width=\maxwidth]{figure/unnamed-chunk-7-1} 
\begin{kframe}\begin{alltt}
\hlkwd{plot}\hlstd{(out2,} \hlkwc{xlab} \hlstd{=} \hlstr{"X"}\hlstd{,} \hlkwc{ylab} \hlstd{=} \hlstr{"Concentration"}\hlstd{,} \hlkwc{las}\hlstd{=}\hlnum{1}\hlstd{,} \hlkwc{main}\hlstd{=}\hlstr{"Advection-Reaction: Steady State Solution"}\hlstd{)}
\end{alltt}
\end{kframe}
\includegraphics[width=\maxwidth]{figure/unnamed-chunk-7-2} 
\begin{kframe}\begin{alltt}
\hlkwd{plot}\hlstd{(out3,} \hlkwc{xlab} \hlstd{=} \hlstr{"X"}\hlstd{,} \hlkwc{ylab} \hlstd{=} \hlstr{"Concentration"}\hlstd{,} \hlkwc{las}\hlstd{=}\hlnum{1}\hlstd{,} \hlkwc{main}\hlstd{=}\hlstr{"Advection-Reaction: Steady State Solution"}\hlstd{)}
\end{alltt}
\end{kframe}
\includegraphics[width=\maxwidth]{figure/unnamed-chunk-7-3} 
\end{knitrout}
\end{figure*}

\subsection{Oxygen Consumption Porous Spherical Particle}

We will be modeling the consumption of oxygen in a "sand-sized" porous spherical particle. The model is based on the following equation:

\[ \frac{\partial C}{\partial t} = -v \frac{\partial C}{\partial x} - k(C) \]

where \( C \) is the concentration of oxygen, \( D \) is the diffusion coefficient, \( v \) is the velocity of the fluid, and \( k(C) \) is the rate of oxygen consumption.

At this scale the velocity will be zero. Thus, we will rely on diffusion to for the oxygen movement to where it is consumed. 



We start with defining the size and porosity of the particle and use R to create a grid to solve the advection-diffusion-reaction equation.

\begin{table}[h]
\caption{Chararacteristics of the Particle}
\centering
\begin{tabular}{|l|l|c|c|} \hline
Parameter & Description & Typical Range &  Modelled Value \\ \hline\hline
\( R \) & Radius of the particle & \( 0.005 - 0.2 \) cm &  0.025 cm \\
Porosity & Proportion of void space & \(0.005 -- 0.7\) &  0.7 \\ \hline
\end{tabular}
\end{table}

We will create a grid to model the particle with Radius \( R \) and \( N \) (= 100) grid points. We will also define the properties of the particle such as porosity, diffusion coefficient (D = 400), and the rate of oxygen consumption, R$_{02}$ = \ensuremath{10^{6}}.

Although we are modeling a one-dimensional system, we will need to create a grid surface as a circle to effectively model the particle surface area changes as O2 diffuses into the particle and is consumed by the reactions in the particle.

\begin{knitrout}
\definecolor{shadecolor}{rgb}{0.969, 0.969, 0.969}\color{fgcolor}\begin{kframe}
\begin{alltt}
\hlstd{grid} \hlkwb{<-} \hlkwd{setup.grid.1D}\hlstd{(}\hlkwc{x.up}\hlstd{=}\hlnum{0}\hlstd{,} \hlkwc{L} \hlstd{= R,} \hlkwc{N} \hlstd{= N)}

\hlstd{por.grid} \hlkwb{<-} \hlkwd{setup.prop.1D}\hlstd{(}\hlkwc{value}\hlstd{=por,} \hlkwc{grid}\hlstd{=grid)}
\hlstd{D.grid} \hlkwb{<-} \hlkwd{setup.prop.1D}\hlstd{(}\hlkwc{value}\hlstd{=D,} \hlkwc{grid}\hlstd{=grid)}

\hlstd{sphere.surf} \hlkwb{<-} \hlkwa{function}\hlstd{(}\hlkwc{x}\hlstd{)} \hlnum{4}\hlopt{*}\hlstd{pi}\hlopt{*}\hlstd{x}\hlopt{^}\hlnum{2}
\hlstd{A.grid} \hlkwb{<-} \hlkwd{setup.prop.1D}\hlstd{(}\hlkwc{func}\hlstd{=sphere.surf,}  \hlkwc{grid}\hlstd{=grid)}
\end{alltt}
\end{kframe}
\end{knitrout}

Finally, we need to define the O2 concentration at the surface of the particle and the O2 consumption rate of the particle. 

\begin{table}[h]
\caption{Oxygen Consumption in the Particle}
\centering
\begin{tabular}{|l|l|c|c|} \hline
Parameter & Description & Typical Range &  Modelled Value \\ \hline\hline
\( C_{ow} \) & Concentration of O2 in Water & \( 0.1 - 0.3 \) \numolspercm &  0.25 \numolspercm \\
\( R_{02} \) & Rate of oxygen consumption & \( 10^5 - 10^6 \) \numolspercm~/year &  \ensuremath{10^{6}} \numolspercm/year \\
\( K_s \) & O2 saturation & \( 0.001 - 0.01 \) \numolspercm &  0.005 \numolspercm \\ \hline
\end{tabular}
\end{table}

Note, we often measure oxygen using ppm (parts per million), but the model uses \numolspercm. The conversion is 1 ppm = 0.0224 \numolspercm, thus, we are modeling \(Ks \) within a range of 2.24 - 6.72 ppm, using Ks = 0.22 \numolspercm.


Next we create a function to model the oxygen consumption in the, particle that relies on We will use the \texttt{tran.1D} function to solve the advection-diffusion equation and the \texttt{steady.1D} function to solve the steady state solution of the advection-diffusion-reaction equation.

\begin{knitrout}
\definecolor{shadecolor}{rgb}{0.969, 0.969, 0.969}\color{fgcolor}\begin{kframe}
\begin{alltt}
\hlstd{Aggregate.Model} \hlkwb{<-} \hlkwa{function}\hlstd{(}\hlkwc{time}\hlstd{,} \hlkwc{O2}\hlstd{,} \hlkwc{pars}\hlstd{) \{}
  \hlstd{tran} \hlkwb{<-} \hlkwd{tran.1D}\hlstd{(}\hlkwc{C} \hlstd{= O2,} \hlkwc{C.down} \hlstd{= C.ow.02,} \hlkwc{D} \hlstd{= D.grid,}
          \hlkwc{A}\hlstd{=A.grid,} \hlkwc{VF} \hlstd{= por.grid,} \hlkwc{dx} \hlstd{= grid)}
    \hlstd{reac} \hlkwb{<-} \hlopt{-} \hlstd{R.02} \hlopt{*} \hlstd{(O2} \hlopt{/}\hlstd{(Ks} \hlopt{+} \hlstd{O2))}
    \hlkwd{return}\hlstd{(}\hlkwd{list}\hlstd{(}\hlkwc{dCdt}\hlstd{= tran}\hlopt{$}\hlstd{dC} \hlopt{+} \hlstd{reac,} \hlkwc{reac} \hlstd{= reac,}
                \hlkwc{flux.up}\hlstd{=tran}\hlopt{$}\hlstd{flux.up,} \hlkwc{flux.down}\hlstd{=tran}\hlopt{$}\hlstd{flux.down))}
\hlstd{\}}


\hlstd{O2.agg} \hlkwb{<-} \hlkwd{steady.1D}\hlstd{(}\hlkwc{y} \hlstd{=} \hlkwd{runif}\hlstd{(N),} \hlkwc{func}\hlstd{=Aggregate.Model,}
                    \hlkwc{nspec}\hlstd{=}\hlnum{1}\hlstd{,} \hlkwc{positive}\hlstd{=}\hlnum{TRUE}\hlstd{,} \hlkwc{atol} \hlstd{=} \hlnum{1e-10}\hlstd{)}
\end{alltt}
\end{kframe}
\end{knitrout}


\begin{figure}[h]
\centering
\caption{Oxygen Consumption in a Porous Sphere}
\begin{knitrout}
\definecolor{shadecolor}{rgb}{0.969, 0.969, 0.969}\color{fgcolor}
\includegraphics[width=\maxwidth]{figure/plotO2agg-1} 
\end{knitrout}
\end{figure}

The plot shows the oxygen concentration in the particle. The concentration is highest at the surface and decreases as it moves into the particle. The concentration is zero at the center of the particle.

\begin{figure}[h]
\begin{knitrout}
\definecolor{shadecolor}{rgb}{0.969, 0.969, 0.969}\color{fgcolor}
\includegraphics[width=\maxwidth]{figure/circleplot-1} 
\end{knitrout}
\end{figure}







%\printclassoptions

% Setting up the margins, etc for R





\begin{knitrout}
\definecolor{shadecolor}{rgb}{0.969, 0.969, 0.969}\color{fgcolor}\begin{kframe}
\begin{alltt}
\hlstd{diffusion2D} \hlkwb{<-} \hlkwa{function}\hlstd{(}\hlkwc{t}\hlstd{,}\hlkwc{conc}\hlstd{,}\hlkwc{par}\hlstd{)\{}
\hlstd{Conc} \hlkwb{<-} \hlkwd{matrix}\hlstd{(}\hlkwc{nr}\hlstd{=n,}\hlkwc{nc}\hlstd{=n,}\hlkwc{data}\hlstd{=conc)} \hlcom{# vector to 2-D matrix}
\hlstd{dConc} \hlkwb{<-} \hlopt{-}\hlstd{r}\hlopt{*}\hlstd{Conc}\hlopt{*}\hlstd{Conc} \hlcom{# consumption}
\hlstd{BND} \hlkwb{<-} \hlkwd{rep}\hlstd{(}\hlnum{1}\hlstd{,n)} \hlcom{# boundary concentration}

\hlcom{# constant production in certain cells}
\hlstd{dConc[ii]}\hlkwb{<-} \hlstd{dConc[ii]}\hlopt{+} \hlstd{p}

\hlcom{#diffusion in X-direction; boundaries=imposed concentration}

\hlstd{Flux} \hlkwb{<-} \hlopt{-}\hlstd{Dx} \hlopt{*} \hlkwd{rbind}\hlstd{(}\hlkwd{rep}\hlstd{(}\hlnum{0}\hlstd{,n),(Conc[}\hlnum{2}\hlopt{:}\hlstd{n,]}\hlopt{-}\hlstd{Conc[}\hlnum{1}\hlopt{:}\hlstd{(n}\hlopt{-}\hlnum{1}\hlstd{),]),}\hlkwd{rep}\hlstd{(}\hlnum{0}\hlstd{,n))}\hlopt{/}\hlstd{dx}
\hlstd{dConc} \hlkwb{<-} \hlstd{dConc} \hlopt{-} \hlstd{(Flux[}\hlnum{2}\hlopt{:}\hlstd{(n}\hlopt{+}\hlnum{1}\hlstd{),]}\hlopt{-}\hlstd{Flux[}\hlnum{1}\hlopt{:}\hlstd{n,])}\hlopt{/}\hlstd{dx}

\hlcom{#diffusion in Y-direction}
\hlstd{Flux} \hlkwb{<-} \hlopt{-}\hlstd{Dy} \hlopt{*} \hlkwd{cbind}\hlstd{(}\hlkwd{rep}\hlstd{(}\hlnum{0}\hlstd{,n),(Conc[,}\hlnum{2}\hlopt{:}\hlstd{n]}\hlopt{-}\hlstd{Conc[,}\hlnum{1}\hlopt{:}\hlstd{(n}\hlopt{-}\hlnum{1}\hlstd{)]),}\hlkwd{rep}\hlstd{(}\hlnum{0}\hlstd{,n))}\hlopt{/}\hlstd{dy}
\hlstd{dConc} \hlkwb{<-} \hlstd{dConc} \hlopt{-} \hlstd{(Flux[,}\hlnum{2}\hlopt{:}\hlstd{(n}\hlopt{+}\hlnum{1}\hlstd{)]}\hlopt{-}\hlstd{Flux[,}\hlnum{1}\hlopt{:}\hlstd{n])}\hlopt{/}\hlstd{dy}

\hlkwd{return}\hlstd{(}\hlkwd{list}\hlstd{(}\hlkwd{as.vector}\hlstd{(dConc)))}
\hlstd{\}}
\end{alltt}
\end{kframe}
\end{knitrout}

After specifying the values of the parameters, 10 cells on the 2-D grid where there will be
substance produced are randomly selected (ii).

14 Package rootSolve : roots, gradients and steady-states in R
0.0 0.2 0.4 0.6 0.8 1.0
0.0 0.2 0.4 0.6 0.8 1.0
2-D diffusion+production
x
y
Figure 5: Steady-state solution of the nonlinear 2-Dimensional model
\begin{knitrout}
\definecolor{shadecolor}{rgb}{0.969, 0.969, 0.969}\color{fgcolor}\begin{kframe}
\begin{alltt}
\hlcom{# parameters}
\hlstd{dy} \hlkwb{<-} \hlstd{dx} \hlkwb{<-} \hlnum{1} \hlcom{# grid size}
\hlstd{Dy} \hlkwb{<-} \hlstd{Dx} \hlkwb{<-} \hlnum{1.5} \hlcom{# diffusion coeff, X- and Y-direction}
\hlstd{r} \hlkwb{<-} \hlnum{0.01} \hlcom{# 2-nd-order consumption rate (/time)}
\hlstd{p} \hlkwb{<-} \hlnum{20} \hlcom{# 0-th order production rate (CONC/t)}
\hlstd{n} \hlkwb{<-} \hlnum{100}
\hlcom{# 10 random cells where substance is produced at rate p}
\hlstd{ii} \hlkwb{<-} \hlkwd{trunc}\hlstd{(}\hlkwd{cbind}\hlstd{(}\hlkwd{runif}\hlstd{(}\hlnum{10}\hlstd{)}\hlopt{*}\hlstd{n}\hlopt{+}\hlnum{1}\hlstd{,}\hlkwd{runif}\hlstd{(}\hlnum{10}\hlstd{)}\hlopt{*}\hlstd{n}\hlopt{+}\hlnum{1}\hlstd{))}
\end{alltt}
\end{kframe}
\end{knitrout}
The steady-state is found using function steady.2D. It takes as arguments a.o. the dimensionality
of the problem (dimens) and lrw=1000000, the length of the work array needed by
the solver. If this value is set too small, the solver will return with the size needed.
It takes about 0.5 second to solve this 10000 state variable model.
\begin{knitrout}
\definecolor{shadecolor}{rgb}{0.969, 0.969, 0.969}\color{fgcolor}\begin{kframe}
\begin{alltt}
\hlstd{Conc0} \hlkwb{<-} \hlkwd{matrix}\hlstd{(}\hlkwc{nr}\hlstd{=n,}\hlkwc{nc}\hlstd{=n,}\hlnum{10.}\hlstd{)}
\hlkwd{print}\hlstd{(}\hlkwd{system.time}\hlstd{(}
\hlstd{ST3} \hlkwb{<-} \hlkwd{steady.2D}\hlstd{(Conc0,}\hlkwc{func}\hlstd{=diffusion2D,}\hlkwc{parms}\hlstd{=}\hlkwa{NULL}\hlstd{,}\hlkwc{pos}\hlstd{=}\hlnum{TRUE}\hlstd{,}\hlkwc{dimens}\hlstd{=}\hlkwd{c}\hlstd{(n,n),}
\hlkwc{lrw}\hlstd{=}\hlnum{1000000}\hlstd{,}\hlkwc{atol}\hlstd{=}\hlnum{1e-10}\hlstd{,}\hlkwc{rtol}\hlstd{=}\hlnum{1e-10}\hlstd{,}\hlkwc{ctol}\hlstd{=}\hlnum{1e-10}\hlstd{)}
\hlstd{))}
\end{alltt}
\begin{verbatim}
##    user  system elapsed 
##   0.242   0.023   0.267
\end{verbatim}
\end{kframe}
\end{knitrout}
user system elapsed
1.044 0.032 1.076
The S3 image method is used to generate the steady-state plot.

\begin{knitrout}
\definecolor{shadecolor}{rgb}{0.969, 0.969, 0.969}\color{fgcolor}\begin{kframe}
\begin{alltt}
\hlkwd{image}\hlstd{(ST3,}\hlkwc{main}\hlstd{=}\hlstr{"2-D diffusion+production"}\hlstd{,} \hlkwc{xlab}\hlstd{=}\hlstr{"x"}\hlstd{,} \hlkwc{ylab}\hlstd{=}\hlstr{"y"}\hlstd{)}
\end{alltt}
\end{kframe}
\includegraphics[width=\maxwidth]{figure/unnamed-chunk-15-1} 
\end{knitrout}


\section{Considering 2D Models}

2.4. Steady-state solution of 2-D PDEs
Function steady.2D effciently snds the steady-state of 2-dimensional problems.
Karline Soetaert 13
In the following model
@C
@t
= Dx 
@2C
@x2 + Dy 
@2C
@y2
.. r  C2 + pxy
a substance C is consumed at a quadratic rate (r C2), while dispersing in X- and Y-direction.
At certain positions (x,y) the substance is produced (rate p).
The model is solved on a square (100*100) grid. There are zero-
ux boundary conditions at
the 4 boundaries.
The term Dx  @2C
@x2 is in fact shorthand for:
..
@Flux
@x
where
Flux = ..Dx 
@C
@x
i.e. it is the negative of the 
ux gradient, where the 
ux is due to diffusion.
In the numerical approximation fo the 
ux, the concentration gradient is approximated as the
subtraction of two matrices, with the columns or rows shifted (e.g. Conc[2:n,]-Conc[1:(n-1),]).
The 
ux gradient is then also approximated by subtracting entire matrices
(e.g. Flux[2:(n+1),]-Flux[1:(n),]). This is very fast. The zero-
ux at the boundaries is
imposed by binding a column or row with 0-s.

\section{Conclusion}


% bibiliography section here-------------------------------------------
%\clearpage

\bibliographystyle{apalike}
%\renewcommand\bibname{References}{}
\bibliography{/home/mwl04747/RTricks/references}%	\addcontentsline{toc}{chapter}{References}


\end{document}
