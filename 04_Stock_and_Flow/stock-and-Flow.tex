% Options for packages loaded elsewhere
\PassOptionsToPackage{unicode}{hyperref}
\PassOptionsToPackage{hyphens}{url}
%
\documentclass[
]{article}
\usepackage{amsmath,amssymb}
\usepackage{iftex}
\ifPDFTeX
  \usepackage[T1]{fontenc}
  \usepackage[utf8]{inputenc}
  \usepackage{textcomp} % provide euro and other symbols
\else % if luatex or xetex
  \usepackage{unicode-math} % this also loads fontspec
  \defaultfontfeatures{Scale=MatchLowercase}
  \defaultfontfeatures[\rmfamily]{Ligatures=TeX,Scale=1}
\fi
\usepackage{lmodern}
\ifPDFTeX\else
  % xetex/luatex font selection
\fi
% Use upquote if available, for straight quotes in verbatim environments
\IfFileExists{upquote.sty}{\usepackage{upquote}}{}
\IfFileExists{microtype.sty}{% use microtype if available
  \usepackage[]{microtype}
  \UseMicrotypeSet[protrusion]{basicmath} % disable protrusion for tt fonts
}{}
\makeatletter
\@ifundefined{KOMAClassName}{% if non-KOMA class
  \IfFileExists{parskip.sty}{%
    \usepackage{parskip}
  }{% else
    \setlength{\parindent}{0pt}
    \setlength{\parskip}{6pt plus 2pt minus 1pt}}
}{% if KOMA class
  \KOMAoptions{parskip=half}}
\makeatother
\usepackage{xcolor}
\usepackage[margin=1in]{geometry}
\usepackage{color}
\usepackage{fancyvrb}
\newcommand{\VerbBar}{|}
\newcommand{\VERB}{\Verb[commandchars=\\\{\}]}
\DefineVerbatimEnvironment{Highlighting}{Verbatim}{commandchars=\\\{\}}
% Add ',fontsize=\small' for more characters per line
\usepackage{framed}
\definecolor{shadecolor}{RGB}{248,248,248}
\newenvironment{Shaded}{\begin{snugshade}}{\end{snugshade}}
\newcommand{\AlertTok}[1]{\textcolor[rgb]{0.94,0.16,0.16}{#1}}
\newcommand{\AnnotationTok}[1]{\textcolor[rgb]{0.56,0.35,0.01}{\textbf{\textit{#1}}}}
\newcommand{\AttributeTok}[1]{\textcolor[rgb]{0.13,0.29,0.53}{#1}}
\newcommand{\BaseNTok}[1]{\textcolor[rgb]{0.00,0.00,0.81}{#1}}
\newcommand{\BuiltInTok}[1]{#1}
\newcommand{\CharTok}[1]{\textcolor[rgb]{0.31,0.60,0.02}{#1}}
\newcommand{\CommentTok}[1]{\textcolor[rgb]{0.56,0.35,0.01}{\textit{#1}}}
\newcommand{\CommentVarTok}[1]{\textcolor[rgb]{0.56,0.35,0.01}{\textbf{\textit{#1}}}}
\newcommand{\ConstantTok}[1]{\textcolor[rgb]{0.56,0.35,0.01}{#1}}
\newcommand{\ControlFlowTok}[1]{\textcolor[rgb]{0.13,0.29,0.53}{\textbf{#1}}}
\newcommand{\DataTypeTok}[1]{\textcolor[rgb]{0.13,0.29,0.53}{#1}}
\newcommand{\DecValTok}[1]{\textcolor[rgb]{0.00,0.00,0.81}{#1}}
\newcommand{\DocumentationTok}[1]{\textcolor[rgb]{0.56,0.35,0.01}{\textbf{\textit{#1}}}}
\newcommand{\ErrorTok}[1]{\textcolor[rgb]{0.64,0.00,0.00}{\textbf{#1}}}
\newcommand{\ExtensionTok}[1]{#1}
\newcommand{\FloatTok}[1]{\textcolor[rgb]{0.00,0.00,0.81}{#1}}
\newcommand{\FunctionTok}[1]{\textcolor[rgb]{0.13,0.29,0.53}{\textbf{#1}}}
\newcommand{\ImportTok}[1]{#1}
\newcommand{\InformationTok}[1]{\textcolor[rgb]{0.56,0.35,0.01}{\textbf{\textit{#1}}}}
\newcommand{\KeywordTok}[1]{\textcolor[rgb]{0.13,0.29,0.53}{\textbf{#1}}}
\newcommand{\NormalTok}[1]{#1}
\newcommand{\OperatorTok}[1]{\textcolor[rgb]{0.81,0.36,0.00}{\textbf{#1}}}
\newcommand{\OtherTok}[1]{\textcolor[rgb]{0.56,0.35,0.01}{#1}}
\newcommand{\PreprocessorTok}[1]{\textcolor[rgb]{0.56,0.35,0.01}{\textit{#1}}}
\newcommand{\RegionMarkerTok}[1]{#1}
\newcommand{\SpecialCharTok}[1]{\textcolor[rgb]{0.81,0.36,0.00}{\textbf{#1}}}
\newcommand{\SpecialStringTok}[1]{\textcolor[rgb]{0.31,0.60,0.02}{#1}}
\newcommand{\StringTok}[1]{\textcolor[rgb]{0.31,0.60,0.02}{#1}}
\newcommand{\VariableTok}[1]{\textcolor[rgb]{0.00,0.00,0.00}{#1}}
\newcommand{\VerbatimStringTok}[1]{\textcolor[rgb]{0.31,0.60,0.02}{#1}}
\newcommand{\WarningTok}[1]{\textcolor[rgb]{0.56,0.35,0.01}{\textbf{\textit{#1}}}}
\usepackage{graphicx}
\makeatletter
\def\maxwidth{\ifdim\Gin@nat@width>\linewidth\linewidth\else\Gin@nat@width\fi}
\def\maxheight{\ifdim\Gin@nat@height>\textheight\textheight\else\Gin@nat@height\fi}
\makeatother
% Scale images if necessary, so that they will not overflow the page
% margins by default, and it is still possible to overwrite the defaults
% using explicit options in \includegraphics[width, height, ...]{}
\setkeys{Gin}{width=\maxwidth,height=\maxheight,keepaspectratio}
% Set default figure placement to htbp
\makeatletter
\def\fps@figure{htbp}
\makeatother
\setlength{\emergencystretch}{3em} % prevent overfull lines
\providecommand{\tightlist}{%
  \setlength{\itemsep}{0pt}\setlength{\parskip}{0pt}}
\setcounter{secnumdepth}{-\maxdimen} % remove section numbering
\ifLuaTeX
  \usepackage{selnolig}  % disable illegal ligatures
\fi
\usepackage{bookmark}
\IfFileExists{xurl.sty}{\usepackage{xurl}}{} % add URL line breaks if available
\urlstyle{same}
\hypersetup{
  pdftitle={Stock and Flow},
  pdfauthor={Marc Los Huertos},
  hidelinks,
  pdfcreator={LaTeX via pandoc}}

\title{Stock and Flow}
\author{Marc Los Huertos}
\date{2025-01-07}

\begin{document}
\maketitle

\subsection{R Markdown}\label{r-markdown}

\begin{Shaded}
\begin{Highlighting}[]
\FunctionTok{library}\NormalTok{(deSolve) }\CommentTok{\#supports numerical integration using a range of numerical methods}
\FunctionTok{library}\NormalTok{(ggplot2) }\CommentTok{\#supports visualization of layered graphics}
\FunctionTok{library}\NormalTok{(gridExtra) }\CommentTok{\#supports visualization of multiple plots \& graphs}
\FunctionTok{library}\NormalTok{(plyr) }\CommentTok{\#supports data frame merging}
\FunctionTok{library}\NormalTok{(magrittr) }\CommentTok{\#supports pipe function}
\FunctionTok{library}\NormalTok{(grid) }\CommentTok{\#supports additional graphing capability}
\end{Highlighting}
\end{Shaded}

\begin{Shaded}
\begin{Highlighting}[]
\CommentTok{\# Set time period and step}
\NormalTok{START }\OtherTok{\textless{}{-}} \DecValTok{2015}
\NormalTok{FINISH }\OtherTok{\textless{}{-}} \DecValTok{2030}
\NormalTok{STEP }\OtherTok{\textless{}{-}} \FloatTok{0.25}
\NormalTok{simtime }\OtherTok{\textless{}{-}} \FunctionTok{seq}\NormalTok{(START, FINISH, }\AttributeTok{by =}\NormalTok{ STEP)}

\FunctionTok{head}\NormalTok{(simtime);}\FunctionTok{tail}\NormalTok{(simtime)}
\end{Highlighting}
\end{Shaded}

\begin{verbatim}
## [1] 2015.00 2015.25 2015.50 2015.75 2016.00 2016.25
\end{verbatim}

\begin{verbatim}
## [1] 2028.75 2029.00 2029.25 2029.50 2029.75 2030.00
\end{verbatim}

\begin{Shaded}
\begin{Highlighting}[]
\CommentTok{\# Set stock capacity, growth and decline rates}
\NormalTok{stocks }\OtherTok{\textless{}{-}} \FunctionTok{c}\NormalTok{(}\AttributeTok{sCustomers =} \DecValTok{10000}\NormalTok{)}
\NormalTok{auxs }\OtherTok{\textless{}{-}} \FunctionTok{c}\NormalTok{(}\AttributeTok{aGrowthFraction =} \FloatTok{0.08}\NormalTok{, }\AttributeTok{aDeclineFraction =} \FloatTok{0.03}\NormalTok{)}
\end{Highlighting}
\end{Shaded}

\begin{Shaded}
\begin{Highlighting}[]
\CommentTok{\# Create the function and returns a list}
\NormalTok{model }\OtherTok{\textless{}{-}} \ControlFlowTok{function}\NormalTok{(time, stocks, auxs)\{}
  \FunctionTok{with}\NormalTok{(}\FunctionTok{as.list}\NormalTok{(}\FunctionTok{c}\NormalTok{(stocks, auxs)), \{}
\NormalTok{    fRecruits }\OtherTok{\textless{}{-}}\NormalTok{ sCustomers }\SpecialCharTok{*}\NormalTok{ aGrowthFraction}
\NormalTok{    fLosses }\OtherTok{\textless{}{-}}\NormalTok{ sCustomers }\SpecialCharTok{*}\NormalTok{ aDeclineFraction}
\NormalTok{    dC\_dt }\OtherTok{\textless{}{-}}\NormalTok{ fRecruits }\SpecialCharTok{{-}}\NormalTok{ fLosses}
    \FunctionTok{return}\NormalTok{ (}\FunctionTok{list}\NormalTok{(}\FunctionTok{c}\NormalTok{(dC\_dt),}
                 \AttributeTok{Recruits =}\NormalTok{ fRecruits, }\AttributeTok{Losses =}\NormalTok{ fLosses,}
                 \AttributeTok{GF =}\NormalTok{ aGrowthFraction, }\AttributeTok{DF =}\NormalTok{ aDeclineFraction))}
\NormalTok{  \}) }
\NormalTok{\}}

\CommentTok{\# Create the data frame using the \textasciigrave{}ode\textasciigrave{} function}
\NormalTok{o }\OtherTok{\textless{}{-}} \FunctionTok{data.frame}\NormalTok{(}\FunctionTok{ode}\NormalTok{(}\AttributeTok{y =}\NormalTok{ stocks, }\AttributeTok{times =}\NormalTok{ simtime, }\AttributeTok{func =}\NormalTok{ model,}
                    \AttributeTok{parms =}\NormalTok{ auxs, }\AttributeTok{method =} \StringTok{"euler"}\NormalTok{))}

\FunctionTok{head}\NormalTok{(o)}
\end{Highlighting}
\end{Shaded}

\begin{verbatim}
##      time sCustomers Recruits   Losses   GF   DF
## 1 2015.00   10000.00 800.0000 300.0000 0.08 0.03
## 2 2015.25   10125.00 810.0000 303.7500 0.08 0.03
## 3 2015.50   10251.56 820.1250 307.5469 0.08 0.03
## 4 2015.75   10379.71 830.3766 311.3912 0.08 0.03
## 5 2016.00   10509.45 840.7563 315.2836 0.08 0.03
## 6 2016.25   10640.82 851.2657 319.2246 0.08 0.03
\end{verbatim}

\begin{Shaded}
\begin{Highlighting}[]
\FunctionTok{summary}\NormalTok{(o[,}\SpecialCharTok{{-}}\FunctionTok{c}\NormalTok{(}\DecValTok{1}\NormalTok{, }\DecValTok{5}\NormalTok{, }\DecValTok{6}\NormalTok{)])}
\end{Highlighting}
\end{Shaded}

\begin{verbatim}
##    sCustomers       Recruits          Losses     
##  Min.   :10000   Min.   : 800.0   Min.   :300.0  
##  1st Qu.:12048   1st Qu.: 963.9   1st Qu.:361.4  
##  Median :14516   Median :1161.3   Median :435.5  
##  Mean   :14866   Mean   :1189.3   Mean   :446.0  
##  3rd Qu.:17489   3rd Qu.:1399.2   3rd Qu.:524.7  
##  Max.   :21072   Max.   :1685.7   Max.   :632.2
\end{verbatim}

4 Limits to Growth It is necessary to introduce a formulation method
before presenting the limits to growth model. Formulation models allows
developers to create equations robust enough to model the effect of a
variable on another. They are useful for a wide variety of system
dynamics models especially those with influencing stocks in different
systems.

We begin with modeling causal relationships using effects. This is
defined as

Y=Y∗∗Effect(X1onY)∗\ldots∗Effect(XnonY)

where, based on the assumptions:

Y = dependent variable of a causal relationship and is a function of n
independent variables (X1,X2,\ldots,Xn)

Y∗ = reference value, which is the normal value the variable Y takes on.
This reference value is multiplied by a sequence of effect functions
that are calculated based on the normalized ratio of input term (Xi/X∗i)
where X∗i is the reference input value, and Xi is the actual input
value.

The effect function has the normalized ratio (X/X∗) on its x-axis, and
always contains the point (1, 1). This point (1, 1) is important for the
following reason: if x equals its reference value X∗ , then the effect
function will be 1, and therefore y will then equal its reference value
Y∗ .

\subsection{New Version}\label{new-version}

\begin{Shaded}
\begin{Highlighting}[]
\CommentTok{\# Load necessary library}
\FunctionTok{library}\NormalTok{(dplyr)}
\end{Highlighting}
\end{Shaded}

\begin{verbatim}
## 
## Attaching package: 'dplyr'
\end{verbatim}

\begin{verbatim}
## The following objects are masked from 'package:plyr':
## 
##     arrange, count, desc, failwith, id, mutate, rename, summarise,
##     summarize
\end{verbatim}

\begin{verbatim}
## The following object is masked from 'package:gridExtra':
## 
##     combine
\end{verbatim}

\begin{verbatim}
## The following objects are masked from 'package:stats':
## 
##     filter, lag
\end{verbatim}

\begin{verbatim}
## The following objects are masked from 'package:base':
## 
##     intersect, setdiff, setequal, union
\end{verbatim}

\begin{Shaded}
\begin{Highlighting}[]
\FunctionTok{library}\NormalTok{(tidyr)}
\end{Highlighting}
\end{Shaded}

\begin{verbatim}
## 
## Attaching package: 'tidyr'
\end{verbatim}

\begin{verbatim}
## The following object is masked from 'package:magrittr':
## 
##     extract
\end{verbatim}

\begin{Shaded}
\begin{Highlighting}[]
\CommentTok{\# Example dataset: Emissions inventories for USA, China, and India over 10 years}
\NormalTok{emissions\_data }\OtherTok{\textless{}{-}} \FunctionTok{data.frame}\NormalTok{(}
  \AttributeTok{Year =} \FunctionTok{rep}\NormalTok{(}\DecValTok{2015}\SpecialCharTok{:}\DecValTok{2024}\NormalTok{, }\AttributeTok{each =} \DecValTok{3}\NormalTok{), }\CommentTok{\# 10 years}
  \AttributeTok{Country =} \FunctionTok{rep}\NormalTok{(}\FunctionTok{c}\NormalTok{(}\StringTok{"USA"}\NormalTok{, }\StringTok{"China"}\NormalTok{, }\StringTok{"India"}\NormalTok{), }\AttributeTok{times =} \DecValTok{10}\NormalTok{),}
  \AttributeTok{CO2\_emissions =} \FunctionTok{c}\NormalTok{(}
    \FunctionTok{runif}\NormalTok{(}\DecValTok{10}\NormalTok{, }\DecValTok{4500}\NormalTok{, }\DecValTok{5000}\NormalTok{), }\FunctionTok{runif}\NormalTok{(}\DecValTok{10}\NormalTok{, }\DecValTok{9000}\NormalTok{, }\DecValTok{10000}\NormalTok{), }\FunctionTok{runif}\NormalTok{(}\DecValTok{10}\NormalTok{, }\DecValTok{2200}\NormalTok{, }\DecValTok{2500}\NormalTok{) }\CommentTok{\# million metric tons}
\NormalTok{  ),}
  \AttributeTok{CH4\_emissions =} \FunctionTok{c}\NormalTok{(}
    \FunctionTok{runif}\NormalTok{(}\DecValTok{10}\NormalTok{, }\DecValTok{250}\NormalTok{, }\DecValTok{300}\NormalTok{), }\FunctionTok{runif}\NormalTok{(}\DecValTok{10}\NormalTok{, }\DecValTok{400}\NormalTok{, }\DecValTok{500}\NormalTok{), }\FunctionTok{runif}\NormalTok{(}\DecValTok{10}\NormalTok{, }\DecValTok{150}\NormalTok{, }\DecValTok{200}\NormalTok{) }\CommentTok{\# million metric tons}
\NormalTok{  ),}
  \AttributeTok{N2O\_emissions =} \FunctionTok{c}\NormalTok{(}
    \FunctionTok{runif}\NormalTok{(}\DecValTok{10}\NormalTok{, }\DecValTok{100}\NormalTok{, }\DecValTok{120}\NormalTok{), }\FunctionTok{runif}\NormalTok{(}\DecValTok{10}\NormalTok{, }\DecValTok{180}\NormalTok{, }\DecValTok{200}\NormalTok{), }\FunctionTok{runif}\NormalTok{(}\DecValTok{10}\NormalTok{, }\DecValTok{80}\NormalTok{, }\DecValTok{100}\NormalTok{) }\CommentTok{\# million metric tons}
\NormalTok{  )}
\NormalTok{)}

\CommentTok{\# Global Warming Potential (GWP) values for 100{-}year horizon}
\NormalTok{GWP\_CO2 }\OtherTok{\textless{}{-}} \DecValTok{1}
\NormalTok{GWP\_CH4 }\OtherTok{\textless{}{-}} \DecValTok{28}
\NormalTok{GWP\_N2O }\OtherTok{\textless{}{-}} \DecValTok{265}

\CommentTok{\# Calculate yearly GWP{-}weighted emissions}
\NormalTok{emissions\_data }\OtherTok{\textless{}{-}}\NormalTok{ emissions\_data }\SpecialCharTok{\%\textgreater{}\%}
  \FunctionTok{mutate}\NormalTok{(}
    \AttributeTok{GWP\_CO2 =}\NormalTok{ CO2\_emissions }\SpecialCharTok{*}\NormalTok{ GWP\_CO2,}
    \AttributeTok{GWP\_CH4 =}\NormalTok{ CH4\_emissions }\SpecialCharTok{*}\NormalTok{ GWP\_CH4,}
    \AttributeTok{GWP\_N2O =}\NormalTok{ N2O\_emissions }\SpecialCharTok{*}\NormalTok{ GWP\_N2O,}
    \AttributeTok{Total\_GWP =}\NormalTok{ GWP\_CO2 }\SpecialCharTok{+}\NormalTok{ GWP\_CH4 }\SpecialCharTok{+}\NormalTok{ GWP\_N2O}
\NormalTok{  )}

\CommentTok{\# Stock{-}and{-}flow model: Cumulative GWP}
\NormalTok{cumulative\_emissions }\OtherTok{\textless{}{-}}\NormalTok{ emissions\_data }\SpecialCharTok{\%\textgreater{}\%}
  \FunctionTok{group\_by}\NormalTok{(Country) }\SpecialCharTok{\%\textgreater{}\%}
  \FunctionTok{arrange}\NormalTok{(Country, Year) }\SpecialCharTok{\%\textgreater{}\%}
  \FunctionTok{mutate}\NormalTok{(}
    \AttributeTok{Cumulative\_GWP =} \FunctionTok{cumsum}\NormalTok{(Total\_GWP) }\CommentTok{\# Accumulate total GWP over time}
\NormalTok{  )}

\CommentTok{\# View cumulative emissions data}
\FunctionTok{print}\NormalTok{(cumulative\_emissions)}
\end{Highlighting}
\end{Shaded}

\begin{verbatim}
## # A tibble: 30 x 10
## # Groups:   Country [3]
##     Year Country CO2_emissions CH4_emissions N2O_emissions GWP_CO2 GWP_CH4
##    <int> <chr>           <dbl>         <dbl>         <dbl>   <dbl>   <dbl>
##  1  2015 China           4692.          260.         107.    4692.   7290.
##  2  2016 China           4659.          281.         115.    4659.   7854.
##  3  2017 China           4715.          278.         110.    4715.   7783.
##  4  2018 China           9222.          470.         187.    9222.  13172.
##  5  2019 China           9840.          455.         198.    9840.  12752.
##  6  2020 China           9135.          463.         181.    9135.  12973.
##  7  2021 China           9369.          493.         181.    9369.  13817.
##  8  2022 China           2238.          184.          80.2   2238.   5149.
##  9  2023 China           2284.          164.          95.2   2284.   4588.
## 10  2024 China           2492.          173.          91.0   2492.   4843.
## # i 20 more rows
## # i 3 more variables: GWP_N2O <dbl>, Total_GWP <dbl>, Cumulative_GWP <dbl>
\end{verbatim}

\begin{Shaded}
\begin{Highlighting}[]
\CommentTok{\# Summarize total GWP and cumulative GWP per country for 10 years}
\NormalTok{summary\_table }\OtherTok{\textless{}{-}}\NormalTok{ cumulative\_emissions }\SpecialCharTok{\%\textgreater{}\%}
  \FunctionTok{group\_by}\NormalTok{(Country) }\SpecialCharTok{\%\textgreater{}\%}
  \FunctionTok{summarize}\NormalTok{(}
    \AttributeTok{Total\_10\_Years\_GWP =} \FunctionTok{sum}\NormalTok{(Total\_GWP),}
    \AttributeTok{Final\_Cumulative\_GWP =} \FunctionTok{max}\NormalTok{(Cumulative\_GWP)}
\NormalTok{  )}

\CommentTok{\# Print summary results}
\FunctionTok{print}\NormalTok{(summary\_table)}
\end{Highlighting}
\end{Shaded}

\begin{verbatim}
## # A tibble: 3 x 3
##   Country Total_10_Years_GWP Final_Cumulative_GWP
##   <chr>                <dbl>                <dbl>
## 1 China              505676.              505676.
## 2 India              469769.              469769.
## 3 USA                488546.              488546.
\end{verbatim}

\begin{Shaded}
\begin{Highlighting}[]
\CommentTok{\# Save detailed and summary results to CSV}
\FunctionTok{write.csv}\NormalTok{(cumulative\_emissions, }\StringTok{"cumulative\_emissions\_gwp\_3\_countries.csv"}\NormalTok{, }\AttributeTok{row.names =} \ConstantTok{FALSE}\NormalTok{)}
\FunctionTok{write.csv}\NormalTok{(summary\_table, }\StringTok{"summary\_gwp\_3\_countries.csv"}\NormalTok{, }\AttributeTok{row.names =} \ConstantTok{FALSE}\NormalTok{)}
\end{Highlighting}
\end{Shaded}

Stocks and Flows Julio Huato 5/8/2020 1 Introduction

As in the physical sciences, all basic magnitudes in economics are
specified either at a point in time or over a period of time. The
distinction between these two types of magnitudes is one of the most
important ones in accounting, finance, and economics. Much confusion can
be avoided by understanding the distinction, and maintaining it
consistently when conducting economic analysis. Let us define the basic
concepts. 2 Definitions

Stock magnitude

A stock magnitude (a stock, for short) is the value of a variable at a
given point in time.

Flow magnitude

A flow magnitude (or, simply, a flow) is the value of a variable for a
given period of time.

Once these basic magnitudes are defined, a host of other magnitudes can
be derived as well---such as ratios of flows to flows, stocks to stocks,
flows to stocks, and stocks1 to flows.

As an analogy, consider a vessel or container with water, with one pipe
delivering additional water into the vessel, and another pipe leaking it
out. A physicist or an engineer would call this a simple ``hydrodynamic
system.''

A simple hydrodynamic system.

Stock and flow magnitudes can be used to measure the performance of this
system. A stock measure would be the amount of water in the container at
given points in time, such as Sundays at noon or on the last day of each
month at 4 PM. Alternatively, a flow measure would be one of the amount
of water flowing into, through, or out of the vessel over given periods
of time, such as a day, week, or month.

Then, by combining both forms of measurement, we would get a more
complete picture of the functioning of the system. The stock measures
may help us correct errors in the measurement of flows and, vice versa.

Using the upper-case letters denote stocks and the lower-case ones
denote flows, we now let Xt be the amount of water in the vessel at
point in time t, xit the amount of water flowing into the vessel during
the month, and xot the amount of water flowing out of the vessel during
the month. Then, the net amount of water flowing through the vessel in
the month is given by xnt=xit−xot, and the amount of water in the
container at point in time t+1 is: Xt+1=Xt+xit−xot=Xt+xnt In words, the
amount of water in the container at the end of the period is equal to
the amount of water in the container at the beginning of the period,
plus the water that got in it during period from t through t+1

, minus the water that leaked out of it during the same period.2

Example: A container On November 1 (t=0 ), the amount of water in the
container is 10 gallons (X0=10). The water flowing into the vessel
during November is 40 gallons (xit=40) and the water flowing out of the
vessel is 39 gallons (xot=39). The amount of water in the container on
December 1: X1=X0+xi1−xo1=10+40−39=11.

That is, there are eleven gallons of water in the container at the end
of November.

If the stock of water on December 1 is any different from (greater than
or less than) the result above, we have made errors in our measurements.
The new measure of the water level can help us correct our records.

Example: Peter's checking account On November 1 (t=0 ), the balance of
Peter's checking account is 100 dollars (X0=100). During November, Peter
deposited checks his grandmother and father sent him that totalled 400
dollars (xit=400). Also in the month, he made electronic payments, wrote
checks, and withdrew cash from the ATM machine for a total of 390
dollars (xot=390). The balance of his account on December 1 is
determined as follows: X1=X0+xi1−xo1=100+400−390=110.

That is, Peter held a balance of 110 dollars at the end of November.

Example: A town's finances On January 1, 2010 (t=0 ), the town's assets
were estimated in 10,000 dollars (X0=10). In 2010, the town received
local taxes and grants from the central government totalling 40,000
dollars (xit=40). Also in 2010, the town spent in its administration,
public works, and local public programs a total of 39,000 dollars
(xot=39). The value of the town's assets at the end of 2010 is given by:
X1=X0+xi1−xo1=10+40−39=11.

That is, the town began 2011 with assets worth 11,000 dollars.

Example: An economy As of January 1, 2010 (t=0 ), the total wealth of an
economy is estimated in 10 billion dollars (X0=10). During 2010, the
economy produced output estimated in 40 billion dollars (xit=40). ALso,
households in the economy consumed 39 billion dollars during the year
(xot=390). The wealth of this economy as of January 1, 2011 is:
X1=X0+xi1−xo1=10+40−39=11.

That is, the outstanding wealth of the economy increased by 10 percent
to 11 billion dollars.

Note that the mathematical structure of the four examples above is
essentially the same.

Exercises

R code 3 Accounting principles

The two basic financial statements that accountants produce are the
balance sheet and the income statement. These financial statements
provide a detailed picture of the ongoing financial performance of a
business or organization.

Balance sheet: The balance sheet of any organization (such as a
business) is a report that describes the financial condition of the
organization at a given point in time.

The balance sheet has two sides. The left-hand side reports the value of
all the organization's assets at a given point in time, typically at the
end of a year. The assets are the resources under the management of the
organization measured at a point in time. The right-hand side reports
the source of the value of the assets listed on the left-hand side. They
are either owed to others, in which case they are called liabilities, or
they are owned by the legal owners, in which case they are called
equity, net worth, or (in the case of banks) capital.

In usual practice, organizations report a balance sheet every year,
quarter, and---sometimes---month. Less usual is every week or day. In
principle, it could be determined at each point in time. This has become
increasingly possible with modern day computers.

The fundamental equation of the balance sheet is: At=Lt+Et

where t indicates a point in time (e.g.~the last day of the year), A is
total assets, L is total liabilities, and E

is total equity.3

Example: A typical balance sheet. See table 1.

Table 1. ABC, Inc.'s Balance Sheet as of 12/31/2010 Item Amount Item
Amount Cash and liquid securities \$10 Payables \$20 Inventories 50
Other short-term debt 40 Receivables 60 Mortgages 80 Trucks (net) 25
Other long-term debt 250 Office equipment (net) 10 Total liabilities 390
Machinery (net) 45 Total equity 120 Buildings (net) 220\\
Other fixed assets (net) 90\\
Total assets \$510 Total liabilities plus equity \$510 :

All items in the balance sheet are stock measures. So, whenever you
consider a balance sheet, think of ``water in a container'' measured at
a point in time.

In the balance sheet, by convention, asset items are viewed as positive
stock magnitudes while liabilities and equity items are negative stock
magnitudes.

Income statement: The income statement (or result statement or
profit-loss statement) of any organization (such as a business) is a
report that describes the financial activity of the organization over a
given period time.

The income statement reports on its top line the total flow of gross
income (e.g.~sales revenues) received by the organization during a
period of time (e.g.~a year) as a positive number. The next lines report
the various expenses that the activity of the organization incurred
during the period to sustain its gross income. These expenses---sorted
out as production costs, operating expenses, financial expenses, and
taxes---are deducted or subtracted from the top line. In other words,
they are regarded as negative numbers. Finally, the bottom line of the
income statement indicates the flow of net income or net profit (if the
net income is positive) or net loss (if the net income is negative)
during the period.

The fundamental equation of the income statement is:
NIt=GIt−PCt−OEt−FEt−Tt where t is the period of time from point in time
t−1 to point in time t, NI is the residual income or net income (net
profit or net loss), GI is the gross income (typically, revenues from
sales, though it may include rental income and interest income the
organization may receive during the period), PC is the total cost of
goods sold (such as cost of raw materials, storage costs, wages and
benefits of factory-floor workers), OE are the operating expenses (sales
and administrative expenses, including salaries and commissions of
administrative and sales personnel and depreciation of fixed assets), FE
is financial expenses (interest paid on outstanding liabilities), and T

is taxes paid.

Example: A typical income statement. See table 2.

Table 2. ABC, Inc.'s Income Statement from 1/1/2010 through 12/31/2010
Item Amount (+ ) Sales revenues \$200 (− ) Cost of goods sold 90 Gross
profit 110 (− ) Operating expenses (includes depreciation) 40 Operating
profit 70 (− ) Interest paid 10 Taxable profit 60 (− ) Taxes 10 Net
profit \$50 :

All items in the income statement are flow measures. Think of them as
``water that flows in or out of the container'' over a period of time.

In the income statement, the convention is that revenues (from sales)
are regarded as positive flows and all costs and expenses as negative
flows.

The balance sheet and the income statement are related in multiple ways.

Again, it is useful to think that, every time an organization conducts
an operation or transaction, every time a business takes raw materials
from its inventories and have its workers process them on the factory
floor, every time its sales people sell a batch of goods or its
administrative personnel orders a shipment from its suppliers, every
time a payment is made or received, etc. there is ``water flowing'' from
one balance-sheet ``container'' into another one. At the end of the
given period (and beginning of the next period), the balance sheet
reports the adjusted levels of ``water'' in each''container'' at that
point in time.

Also at the end of the given period (beginning of the next), each spurt
of ``water'' that flowed from ``container'' to ``container'' during the
period is added up (aggregated) into its respective category and
recorded in the income statement. The legal owners of the organization
(if a corporation, the legal owners are called stockholders or
shareholders) pay most attention to the level of ``water'' in their
equity ``container.''

The examples above should make it clear that the principles of
accounting are the same whether they are applied to one household,
business, entity, or organization, including a national economy or the
entire global economy for that matter.

The Generally Accepted Accounting Principles (GAAP)---with rules set by
the Financial Accounting Standards Board (FASB), a private sector
organization---that are used in business practice, are a set of detailed
conventions based on these fundamental principles. The national income
and product accounts (NIPA) and the balance of payments (BoP) accounts
kept by the Bureau of Economic Analysis (BEA) of the U.S. Department of
Commerce, the flow of funds (FoF) accounts kept by the Federal Reserve
(``Fed''), and the data on prices, employment, pay, and productivity
kept by Bureau of Labor Statistics, all of them statistical records
intended to measure the performance of the U.S. economy, are based on
these very principles. Other national statistics agencies, central
banks, and international organizations---such as the International
Monetary Fund (IMF), the World Bank (WB), the World Trade Organization
(WTO), the Organization for Economic Cooperation and Development (OECD),
and the United Nations (UN)---apply the same principles to their
statistical collection efforts.

Further details are the subject matter of formal courses in financial
accounting, macroeconomics, and other particular fields.

Exercises

R code

\begin{verbatim}
The term stock has several meanings in economics and finance. It also refers to the equity or residual wealth claimed by the legal owners of a company traded in the market. The context should make it clear when the term is used in one or another sense.↩︎

Assume no evaporation or, alternatively, that the water leaked out in the period includes evaporated water.↩︎

To separate an organization from its individual owners, it may be convenient to state the equation as A = L, i.e. assets equal liabilities. They are liabilities to either others or to the individual “owners” of the organization. In this interpretation, the equity of the legal owners of, e.g., a business is considered a special type of liability.↩︎
\end{verbatim}

\end{document}
