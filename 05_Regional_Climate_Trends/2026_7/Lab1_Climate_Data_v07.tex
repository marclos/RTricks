\documentclass{article}\usepackage[]{graphicx}\usepackage[]{xcolor}
% maxwidth is the original width if it is less than linewidth
% otherwise use linewidth (to make sure the graphics do not exceed the margin)
\makeatletter
\def\maxwidth{ %
  \ifdim\Gin@nat@width>\linewidth
    \linewidth
  \else
    \Gin@nat@width
  \fi
}
\makeatother

\definecolor{fgcolor}{rgb}{0.345, 0.345, 0.345}
\newcommand{\hlnum}[1]{\textcolor[rgb]{0.686,0.059,0.569}{#1}}%
\newcommand{\hlstr}[1]{\textcolor[rgb]{0.192,0.494,0.8}{#1}}%
\newcommand{\hlcom}[1]{\textcolor[rgb]{0.678,0.584,0.686}{\textit{#1}}}%
\newcommand{\hlopt}[1]{\textcolor[rgb]{0,0,0}{#1}}%
\newcommand{\hlstd}[1]{\textcolor[rgb]{0.345,0.345,0.345}{#1}}%
\newcommand{\hlkwa}[1]{\textcolor[rgb]{0.161,0.373,0.58}{\textbf{#1}}}%
\newcommand{\hlkwb}[1]{\textcolor[rgb]{0.69,0.353,0.396}{#1}}%
\newcommand{\hlkwc}[1]{\textcolor[rgb]{0.333,0.667,0.333}{#1}}%
\newcommand{\hlkwd}[1]{\textcolor[rgb]{0.737,0.353,0.396}{\textbf{#1}}}%
\let\hlipl\hlkwb

\usepackage{framed}
\makeatletter
\newenvironment{kframe}{%
 \def\at@end@of@kframe{}%
 \ifinner\ifhmode%
  \def\at@end@of@kframe{\end{minipage}}%
  \begin{minipage}{\columnwidth}%
 \fi\fi%
 \def\FrameCommand##1{\hskip\@totalleftmargin \hskip-\fboxsep
 \colorbox{shadecolor}{##1}\hskip-\fboxsep
     % There is no \\@totalrightmargin, so:
     \hskip-\linewidth \hskip-\@totalleftmargin \hskip\columnwidth}%
 \MakeFramed {\advance\hsize-\width
   \@totalleftmargin\z@ \linewidth\hsize
   \@setminipage}}%
 {\par\unskip\endMakeFramed%
 \at@end@of@kframe}
\makeatother

\definecolor{shadecolor}{rgb}{.97, .97, .97}
\definecolor{messagecolor}{rgb}{0, 0, 0}
\definecolor{warningcolor}{rgb}{1, 0, 1}
\definecolor{errorcolor}{rgb}{1, 0, 0}
\newenvironment{knitrout}{}{} % an empty environment to be redefined in TeX

\usepackage{alltt}
\usepackage{hyperref}
\hypersetup{colorlinks=true, linkcolor=blue, urlcolor=blue}
\usepackage{graphicx}
\usepackage{fancyvrb}
\usepackage{xcolor}
\usepackage{framed}

% Define a shaded box for code examples
\definecolor{shadecolor}{RGB}{248,248,248}
\definecolor{codegreen}{RGB}{0,128,0}

\title{Lab 1: Climate Data Collection and Processing\\
\large{Regional Climate Trends Project}\\
\normalsize{Learning R Through Climate Science}}
\author{EA050}
\date{\today~v. 0.07}
\IfFileExists{upquote.sty}{\usepackage{upquote}}{}
\begin{document}
\maketitle

\tableofcontents
\newpage

\section{Introduction}

Welcome to Lab 1 of the Regional Climate Trends Project! This lab is designed to teach you fundamental R programming concepts while analyzing real climate data. You will learn:

\begin{itemize}
  \item How R handles dates and time
  \item Data manipulation with subsetting and aggregation
  \item Creating basic plots
  \item Understanding data structures (data frames, lists)
  \item Working with NOAA climate data
\end{itemize}

\subsection{Learning Objectives}

By the end of this lab, you will be able to:
\begin{enumerate}
  \item Convert between date formats in R
  \item Subset data frames by conditions
  \item Aggregate data using the \texttt{aggregate()} function
  \item Create basic time series plots
  \item Understand anomalies and their calculation
\end{enumerate}

\section{Setup}

First, we need to set up our R environment. The following code loads all necessary packages and functions.

\begin{knitrout}
\definecolor{shadecolor}{rgb}{0.969, 0.969, 0.969}\color{fgcolor}\begin{kframe}
\begin{verbatim}
## 
## ===========================================================
##  Climate Narratives Functions v7.0 Loaded Successfully!
## ===========================================================
##  Run check_packages() to verify dependencies
## ===========================================================
## 
##  QUICK START:
##  1. setup_project('CA')
##  2. select_stations_for_analysis(n_stations = 50)
##  3. download_stations()
##  4. load_and_save_stations(cleanup = TRUE)
##  5. process_all_stations_for_spatial()
##  6. create_spatial_objects(all_station_trends)
## 
##  NEW IN v7.0:
##  - Fixed figuresfolder variable handling
##  - Improved error messages
##  - Better documentation for teaching
## 
## ===========================================================
\end{verbatim}
\end{kframe}
\end{knitrout}

\begin{shaded}
\begin{Verbatim}[fontsize=\small]
# Set working directory to your project folder
# IMPORTANT: Change this path to match your computer!
setwd("/path/to/your/project/folder/")

# Load the consolidated functions file
source("ClimateNarrativesFunctions_v07.R")

# Check that all required packages are installed
check_packages()
\end{Verbatim}
\end{shaded}

\textbf{R Concept: Working Directory}

The \texttt{setwd()} function sets your ``working directory'' -- the folder where R looks for files by default. Always set this at the beginning of your script to ensure file paths work correctly.

\section{Project Initialization}

Let's set up the project structure and download station information.

\begin{knitrout}
\definecolor{shadecolor}{rgb}{0.969, 0.969, 0.969}\color{fgcolor}\begin{kframe}
\begin{verbatim}
## 
## ===========================================================
##   Climate Narratives Project Setup v7.0
## ===========================================================
##   Enhanced for improved spatial analysis
## ===========================================================
##   Project path: /home/mwl04747/RTricks/05_Regional_Climate_Trends/2026_7 
## ===========================================================
## 
## Creating directories...
##   [OK] Exists: /home/mwl04747/RTricks/05_Regional_Climate_Trends/2026_7/Data 
##   [OK] Exists: /home/mwl04747/RTricks/05_Regional_Climate_Trends/2026_7/Output 
##   [OK] Exists: /home/mwl04747/RTricks/05_Regional_Climate_Trends/2026_7/Figures 
## 
## [OK] Set folder variables:
##      datafolder    = /home/mwl04747/RTricks/05_Regional_Climate_Trends/2026_7/Data/ 
##      figuresfolder = /home/mwl04747/RTricks/05_Regional_Climate_Trends/2026_7/Figures/ 
## 
## [OK] Station inventory already exists
## 
## [OK] Found 614 potential stations for CA 
## 
## ===========================================================
##   Setup Complete!
## ===========================================================
##   Variables created in global environment:
##     * my.state       = CA 
##     * my.inventory   = 614 potential stations
##     * datafolder     = /home/mwl04747/RTricks/05_Regional_Climate_Trends/2026_7/Data/ 
##     * figuresfolder  = /home/mwl04747/RTricks/05_Regional_Climate_Trends/2026_7/Figures/ 
## ===========================================================
##   Next step:
##     select_stations_for_analysis(n_stations = 50)
## ===========================================================
\end{verbatim}
\end{kframe}
\end{knitrout}

\begin{shaded}
\begin{Verbatim}[fontsize=\small]
# Set your state (change this to your assigned state!)
my.state <- "CA"

# Set up project folders and download station inventory
setup_project(my.state)
\end{Verbatim}
\end{shaded}

\textbf{R Concept: Variable Assignment}

In R, we use \texttt{<-} to assign values to variables. The variable name goes on the left, and the value goes on the right. You can also use \texttt{=}, but \texttt{<-} is the traditional R style.

The \texttt{setup\_project()} function creates several important variables in your R environment:
\begin{itemize}
  \item \texttt{my.state} -- Your two-letter state code
  \item \texttt{my.inventory} -- Data frame of available weather stations
  \item \texttt{datafolder} -- Path to the Data folder
  \item \texttt{figuresfolder} -- Path to the Figures folder
\end{itemize}

\section{Station Selection}

Now we select high-quality stations with long data records.

\begin{knitrout}
\definecolor{shadecolor}{rgb}{0.969, 0.969, 0.969}\color{fgcolor}\begin{kframe}
\begin{verbatim}
## ===========================================================
##   Selecting Stations for Analysis v7.0
## ===========================================================
## Quality filters:
##   Minimum record length: 50 years
##   Must have data through: 2020 
##   Target number of stations: 50 
## 
## Stations meeting quality criteria: 223 
## Selected top 50 stations by record length
## ===========================================================
## Selected Station Summary:
## ===========================================================
##   Number of stations: 50 
##   Oldest station start: 1870 
##   Newest station start: 1906 
##   Average record length: 129.4 years
##   Median record length: 129 years
## ===========================================================
## [OK] Updated my.inventory with 50 selected stations
## [OK] Saved to: /home/mwl04747/RTricks/05_Regional_Climate_Trends/2026_7/Data/selected_inventory_CA.csv
## 
## Next step: download_stations()
\end{verbatim}
\end{kframe}
\end{knitrout}

\begin{shaded}
\begin{Verbatim}[fontsize=\small]
# Select 50 stations with quality filters
selected_inventory <- select_stations_for_analysis(
  n_stations = 50,    # Number of stations to select
  min_years = 50,     # Minimum years of data required
  min_last_year = 2020 # Must have data through this year
)
\end{Verbatim}
\end{shaded}

\textbf{R Concept: Function Arguments}

Functions in R can have multiple \textbf{arguments} (inputs). You can specify them by name (like \texttt{n\_stations = 50}) or by position. Using names makes your code clearer and helps avoid mistakes.

Let's examine the selected stations:

\begin{knitrout}
\definecolor{shadecolor}{rgb}{0.969, 0.969, 0.969}\color{fgcolor}\begin{kframe}
\begin{verbatim}
##            ID                 NAME FIRSTYEAR LASTYEAR RECORD_LENGTH
## 1 USC00043157           FT BIDWELL      1870     2026           157
## 2 USW00023271     SACRAMENTO 5 ESE      1877     2025           149
## 3 USC00042294 DAVIS 2 WSW EXP FARM      1893     2026           134
## 4 USC00046074  NAPA STATE HOSPITAL      1893     2026           134
## 5 USC00046136          NEVADA CITY      1893     2026           134
## 6 USC00046719             PASADENA      1893     2026           134
\end{verbatim}
\end{kframe}
\end{knitrout}

\begin{shaded}
\begin{Verbatim}[fontsize=\small]
# View the first 6 rows of our station inventory
head(selected_inventory[, c("ID", "NAME", "FIRSTYEAR", 
                            "LASTYEAR", "RECORD_LENGTH")])
\end{Verbatim}
\end{shaded}

\textbf{R Concept: Subsetting Data Frames}

Data frames are like spreadsheets -- they have rows and columns. We access specific parts using square brackets: \texttt{df[rows, columns]}.

\begin{itemize}
  \item \texttt{df[1, ]} -- First row, all columns
  \item \texttt{df[, 1]} -- All rows, first column
  \item \texttt{df[1:5, c("col1", "col2")]} -- Rows 1-5, specific columns
  \item \texttt{head(df)} -- First 6 rows (convenient shortcut)
\end{itemize}

\section{Download Station Data}

\textbf{Note:} This step downloads data from NOAA and takes 10-30 minutes. Run it once, then the data is saved locally.

\begin{knitrout}
\definecolor{shadecolor}{rgb}{0.969, 0.969, 0.969}\color{fgcolor}\begin{kframe}
\begin{alltt}
\hlcom{# Download data from NOAA (run once, takes 10-30 minutes)}
\hlkwd{download_stations}\hlstd{()}
\end{alltt}
\end{kframe}
\end{knitrout}

\begin{shaded}
\begin{Verbatim}[fontsize=\small]
# Download data from NOAA (run once, takes 10-30 minutes)
download_stations()
\end{Verbatim}
\end{shaded}



\section{Load and Save Data}

After downloading, we load the data into R and save it in an efficient format.

\begin{knitrout}
\definecolor{shadecolor}{rgb}{0.969, 0.969, 0.969}\color{fgcolor}\begin{kframe}
\begin{verbatim}
## ===========================================================
##   Loading and Processing Station Data
## ===========================================================
## [1/50] USC00043157 (228,582 records)
## [2/50] USW00023271 (196,575 records)
## [3/50] USC00042294 (364,733 records)
## [4/50] USC00046074 (211,071 records)
## [5/50] USC00046136 (245,445 records)
## [6/50] USC00046719 (218,343 records)
## [7/50] USC00046826 (200,734 records)
## [8/50] USC00047821 (195,308 records)
## [9/50] USC00047902 (231,585 records)
## [10/50] USC00049866 (219,335 records)
## [11/50] USC00040693 (220,819 records)
## [12/50] USC00044412 (73,202 records)
## [13/50] USC00048351 (150,489 records)
## [14/50] USC00041614 (216,044 records)
## [15/50] USC00046730 (211,996 records)
## [16/50] USC00047880 (150,982 records)
## [17/50] USC00044259 (227,308 records)
## [18/50] USC00049073 (147,356 records)
## [19/50] USC00043161 (182,043 records)
## [20/50] USC00047195 (212,706 records)
## [21/50] USC00048702 (188,081 records)
## [22/50] USW00023157 (315,977 records)
## [23/50] USC00040943 (134,213 records)
## [24/50] USC00043875 (205,387 records)
## [25/50] USC00042805 (196,827 records)
## [26/50] USC00043191 (174,331 records)
## [27/50] USC00048839 (191,648 records)
## [28/50] USC00049367 (228,008 records)
## [29/50] USC00043747 (221,740 records)
## [30/50] USC00044890 (215,181 records)
## [31/50] USC00042239 (216,876 records)
## [32/50] USC00045532 (208,262 records)
## [33/50] USC00049452 (207,388 records)
## [34/50] USC00041018 (191,794 records)
## [35/50] USC00044223 (200,261 records)
## [36/50] USC00044500 (209,634 records)
## [37/50] USC00047965 (200,730 records)
## [38/50] USC00046168 (205,158 records)
## [39/50] USC00044997 (210,787 records)
## [40/50] USC00046508 (200,300 records)
## [41/50] USC00046506 (214,291 records)
## [42/50] USC00046252 (186,788 records)
## [43/50] USC00040383 (197,382 records)
## [44/50] USC00041912 (201,996 records)
## [45/50] USC00046399 (204,277 records)
## [46/50] USC00049855 (204,181 records)
## [47/50] USC00048353 (218,776 records)
## [48/50] USC00049699 (215,085 records)
## [49/50] USW00093134 (305,096 records)
## [50/50] USC00041072 (157,593 records)
## 
## [OK] Saved 50 stations to: /home/mwl04747/RTricks/05_Regional_Climate_Trends/2026_7/Data/all_stations_raw.RData 
##      File size: 16 MB
## 
## Cleaning up temporary files...
##   Removing 100 .csv and .gz files
##   Freed up 498 MB of disk space
## ===========================================================
## Loading Summary:
## ===========================================================
##   Successfully loaded: 50 stations
##   Failed to load: 0 stations
##   Saved to RData: /home/mwl04747/RTricks/05_Regional_Climate_Trends/2026_7/Data/all_stations_raw.RData 
##   Cleaned up temporary files: YES
## ===========================================================
## Next step: process_all_stations()
\end{verbatim}
\end{kframe}
\end{knitrout}

\begin{shaded}
\begin{Verbatim}[fontsize=\small]
# Load all CSV files and save as RData format
# cleanup = TRUE removes temporary files to save disk space
load_and_save_stations(cleanup = TRUE)
\end{Verbatim}
\end{shaded}

\textbf{R Concept: RData Files}

RData files (\texttt{.RData}) are R's native format for saving data. They:
\begin{itemize}
  \item Are compressed (smaller file size)
  \item Load faster than CSV files
  \item Preserve R data types (dates, factors, etc.)
  \item Can store multiple objects in one file
\end{itemize}

\section{Learning R: Working with Dates}

One of the most important skills in data analysis is handling dates properly. Let's learn how R handles dates using our climate data.

\subsection{The Problem: NOAA Date Format}

NOAA stores dates as 8-digit integers. For example:
\begin{itemize}
  \item \texttt{20230715} means July 15, 2023
  \item \texttt{19610101} means January 1, 1961
\end{itemize}

R doesn't automatically recognize this format. We need to convert it.



\begin{knitrout}
\definecolor{shadecolor}{rgb}{0.969, 0.969, 0.969}\color{fgcolor}\begin{kframe}
\begin{verbatim}
## Raw DATE values from NOAA:
##  [1] 18670601 18670602 18670603 18670604 18670605 18670606 18670607 18670608
##  [9] 18670609 18670610
## 
## Data type: integer
\end{verbatim}
\end{kframe}
\end{knitrout}

\begin{shaded}
\begin{Verbatim}[fontsize=\small]
# Look at the raw date format
cat("Raw DATE values from NOAA:\n")
print(head(example_station$DATE, 10))
cat("\nData type:", class(example_station$DATE), "\n")
\end{Verbatim}
\end{shaded}

\subsection{Converting Dates with as.Date()}

The \texttt{as.Date()} function converts text to proper date objects.

\begin{knitrout}
\definecolor{shadecolor}{rgb}{0.969, 0.969, 0.969}\color{fgcolor}\begin{kframe}
\begin{verbatim}
## Original integer: 20230715
## As character:     20230715
## As Date object:   2023-07-15
## Data type now:    Date
\end{verbatim}
\end{kframe}
\end{knitrout}

\begin{shaded}
\begin{Verbatim}[fontsize=\small]
# Convert integer to character, then to Date
# The format string tells R how to interpret the text:
#   %Y = 4-digit year
#   %m = 2-digit month  
#   %d = 2-digit day

example_date <- 20230715
date_as_char <- as.character(example_date)
date_proper <- as.Date(date_as_char, format = "%Y%m%d")

cat("Original integer:", example_date, "\n")
cat("As character:    ", date_as_char, "\n")
cat("As Date object:  ", as.character(date_proper), "\n")
\end{Verbatim}
\end{shaded}

\textbf{R Concept: Date Format Codes}

Common format codes for dates:
\begin{itemize}
  \item \texttt{\%Y} -- 4-digit year (2023)
  \item \texttt{\%y} -- 2-digit year (23)
  \item \texttt{\%m} -- Month as number (07)
  \item \texttt{\%d} -- Day of month (15)
  \item \texttt{\%b} -- Abbreviated month name (Jul)
  \item \texttt{\%B} -- Full month name (July)
\end{itemize}

\subsection{Extracting Month and Year}

Once we have a Date object, we can extract components:

\begin{knitrout}
\definecolor{shadecolor}{rgb}{0.969, 0.969, 0.969}\color{fgcolor}\begin{kframe}
\begin{verbatim}
## Date: 2023-07-15
## Month: 7
## Year: 2023
## Day: 15
\end{verbatim}
\end{kframe}
\end{knitrout}

\begin{shaded}
\begin{Verbatim}[fontsize=\small]
# Extract month and year from a date
my_date <- as.Date("2023-07-15")

# Use format() to extract parts
month_num <- as.numeric(format(my_date, "%m"))
year_num <- as.numeric(format(my_date, "%Y"))
day_num <- as.numeric(format(my_date, "%d"))

cat("Date:", as.character(my_date), "\n")
cat("Month:", month_num, "\n")
cat("Year:", year_num, "\n")
cat("Day:", day_num, "\n")
\end{Verbatim}
\end{shaded}

\subsection{The fixDates.fun() Function}

Our \texttt{fixDates.fun()} function does all of this for station data:

\begin{knitrout}
\definecolor{shadecolor}{rgb}{0.969, 0.969, 0.969}\color{fgcolor}\begin{kframe}
\begin{verbatim}
## New columns added:
##   Ymd (Date): Date
##   MONTH: numeric - range: 1 to 12
##   YEAR: numeric - range: 1867 to 2026
\end{verbatim}
\end{kframe}
\end{knitrout}

\begin{shaded}
\begin{Verbatim}[fontsize=\small]
# The fixDates.fun() function adds three new columns:
#   Ymd   - proper Date object
#   MONTH - month number (1-12)
#   YEAR  - 4-digit year

station_a <- fixDates.fun(example_station)

# Check the new columns
head(station_a[, c("DATE", "Ymd", "MONTH", "YEAR")])
\end{Verbatim}
\end{shaded}

\section{Learning R: Unit Conversions}

NOAA stores values in scaled units to save storage space. We need to convert them.

\subsection{Understanding NOAA Units}

\begin{itemize}
  \item \textbf{Temperature (TMAX, TMIN):} Stored in tenths of degrees Celsius
    \begin{itemize}
      \item Value of 235 = 23.5°C
      \item Value of -50 = -5.0°C
    \end{itemize}
  \item \textbf{Precipitation (PRCP):} Stored in tenths of millimeters
    \begin{itemize}
      \item Value of 50 = 5.0 mm
    \end{itemize}
\end{itemize}

\begin{knitrout}
\definecolor{shadecolor}{rgb}{0.969, 0.969, 0.969}\color{fgcolor}\begin{kframe}
\begin{verbatim}
## Raw TMAX values (tenths of degrees C):
## [1] -6 11 17 22 -6
## 
## These need to be divided by 10 to get actual temperatures.
\end{verbatim}
\end{kframe}
\end{knitrout}

\begin{shaded}
\begin{Verbatim}[fontsize=\small]
# Look at raw values before conversion
tmax_raw <- subset(station_a, ELEMENT == "TMAX")
cat("Raw TMAX values (tenths of degrees C):\n")
print(head(tmax_raw$VALUE, 5))
\end{Verbatim}
\end{shaded}

\subsection{The fixValues.fun() Function}

\begin{knitrout}
\definecolor{shadecolor}{rgb}{0.969, 0.969, 0.969}\color{fgcolor}\begin{kframe}
\begin{verbatim}
## After conversion (degrees C):
## [1] -0.6  1.1  1.7  2.2 -0.6
\end{verbatim}
\end{kframe}
\end{knitrout}

\begin{shaded}
\begin{Verbatim}[fontsize=\small]
# Apply unit conversion
station_b <- fixValues.fun(station_a)

# The function divides temperature by 10
# and precipitation by 10 to get proper units
\end{Verbatim}
\end{shaded}

\section{Learning R: Data Subsetting}

Subsetting is one of the most important skills in R. Let's practice with our climate data.

\subsection{Using subset()}

\begin{knitrout}
\definecolor{shadecolor}{rgb}{0.969, 0.969, 0.969}\color{fgcolor}\begin{kframe}
\begin{verbatim}
## TMAX records: 48982
## Summer records: 56751
## Summer TMAX records: 12388
\end{verbatim}
\end{kframe}
\end{knitrout}

\begin{shaded}
\begin{Verbatim}[fontsize=\small]
# Subset for just TMAX data
tmax_data <- subset(station_b, ELEMENT == "TMAX")

# Subset for summer months (June, July, August)
# The %in% operator checks if values are in a list
summer_data <- subset(station_b, MONTH %in% c(6, 7, 8))

# Combine conditions with & (AND)
hot_summer <- subset(station_b, 
                     ELEMENT == "TMAX" & MONTH %in% c(6, 7, 8))

# Use | for OR conditions
# Use != for "not equal"
\end{Verbatim}
\end{shaded}

\textbf{R Concept: Logical Operators}

\begin{itemize}
  \item \texttt{==} Equal to
  \item \texttt{!=} Not equal to
  \item \texttt{>}, \texttt{<} Greater/less than
  \item \texttt{>=}, \texttt{<=} Greater/less than or equal
  \item \texttt{\&} AND (both conditions must be true)
  \item \texttt{|} OR (either condition can be true)
  \item \texttt{\%in\%} Is the value in this list?
\end{itemize}

\section{Learning R: Creating Plots}

Visualization is crucial for understanding climate data. Let's create some basic plots.

\subsection{Basic Time Series Plot}

\begin{knitrout}
\definecolor{shadecolor}{rgb}{0.969, 0.969, 0.969}\color{fgcolor}
\includegraphics[width=\maxwidth]{figure/plot-basics-1} 
\end{knitrout}

\begin{shaded}
\begin{Verbatim}[fontsize=\small]
# Get TMAX data
tmax_data <- subset(station_b, ELEMENT == "TMAX")

# Basic plot with plot()
plot(VALUE ~ Ymd, data = tmax_data,
     type = "l",          # "l" for lines, "p" for points
     col = "red",         # color
     main = "Daily Maximum Temperature",  # title
     xlab = "Date",       # x-axis label
     ylab = "Temperature (°C)")  # y-axis label
\end{Verbatim}
\end{shaded}

\textbf{R Concept: The Formula Syntax}

In R, \texttt{y \textasciitilde{} x} means ``y as a function of x''. This is called \textbf{formula syntax} and is used throughout R for:
\begin{itemize}
  \item Plotting: \texttt{plot(y \textasciitilde{} x)}
  \item Regression: \texttt{lm(y \textasciitilde{} x)}
  \item Aggregation: \texttt{aggregate(y \textasciitilde{} group, ...)}
\end{itemize}

\subsection{Multiple Panels}

\begin{knitrout}
\definecolor{shadecolor}{rgb}{0.969, 0.969, 0.969}\color{fgcolor}
\includegraphics[width=\maxwidth]{figure/plot-multipanel-1} 
\end{knitrout}

\begin{shaded}
\begin{Verbatim}[fontsize=\small]
# Set up 2x2 panel of plots
par(mfrow = c(2, 2),    # 2 rows, 2 columns
    mar = c(4, 4, 3, 1)) # margins: bottom, left, top, right

# Plot 1: TMAX time series
plot(VALUE ~ Ymd, data = tmax_data, type = "l", col = "red",
     main = "Max Temperature", xlab = "Date", ylab = "°C")

# Plot 2: TMIN time series  
plot(VALUE ~ Ymd, data = tmin_data, type = "l", col = "blue",
     main = "Min Temperature", xlab = "Date", ylab = "°C")

# Plot 3: Precipitation (type = "h" for histogram-like bars)
plot(VALUE ~ Ymd, data = prcp_data, type = "h", col = "green",
     main = "Precipitation", xlab = "Date", ylab = "mm")

# Plot 4: Histogram
hist(tmax_data$VALUE, breaks = 30, col = "coral",
     main = "TMAX Distribution", xlab = "Temperature (°C)")

par(mfrow = c(1, 1))  # Reset to single panel
\end{Verbatim}
\end{shaded}

\section{Learning R: Data Aggregation}

Aggregation means summarizing data by groups. For climate analysis, we aggregate daily data to monthly values.

\subsection{Using aggregate()}

\begin{knitrout}
\definecolor{shadecolor}{rgb}{0.969, 0.969, 0.969}\color{fgcolor}\begin{kframe}
\begin{verbatim}
## Monthly TMAX data (first 12 rows):
##    MONTH YEAR      TMAX
## 1      1 1870  4.506452
## 2      2 1870  7.021429
## 3      3 1870  7.141935
## 4      4 1870 16.060000
## 5      5 1870 20.641935
## 6      6 1870 25.926667
## 7      7 1870 31.161290
## 8      8 1870 29.464516
## 9      9 1870 24.350000
## 10    10 1870 16.787097
## 11    11 1870 10.700000
## 12    12 1870 -1.319355
\end{verbatim}
\end{kframe}
\end{knitrout}

\begin{shaded}
\begin{Verbatim}[fontsize=\small]
# Calculate monthly mean TMAX
tmax_data <- subset(station_b, ELEMENT == "TMAX")

# aggregate() syntax:
#   VALUE ~ MONTH + YEAR means: 
#     "aggregate VALUE by MONTH and YEAR"
#   FUN = mean means: calculate the mean
#   na.rm = TRUE means: ignore missing values

monthly_tmax <- aggregate(VALUE ~ MONTH + YEAR, 
                          data = tmax_data,
                          FUN = mean,    # or sum, median, sd, etc.
                          na.rm = TRUE)

# Rename the result column
names(monthly_tmax)[3] <- "TMAX"
\end{Verbatim}
\end{shaded}

\textbf{R Concept: Aggregation Functions}

Common functions to use with \texttt{aggregate()}:
\begin{itemize}
  \item \texttt{mean} -- Average
  \item \texttt{sum} -- Total
  \item \texttt{median} -- Middle value
  \item \texttt{sd} -- Standard deviation
  \item \texttt{min}, \texttt{max} -- Extremes
  \item \texttt{length} -- Count of observations
\end{itemize}

\section{Understanding Climate Anomalies}

\subsection{What is an Anomaly?}

An \textbf{anomaly} is the difference between an observed value and a reference value (the ``normal''):

\[ \text{Anomaly} = \text{Observed} - \text{Normal} \]

\begin{itemize}
  \item Positive anomaly = warmer/wetter than normal
  \item Negative anomaly = cooler/drier than normal
\end{itemize}

\subsection{Climate Normals (1961-1990)}

The standard reference period is 1961-1990, called the ``climate normal.'' We calculate the average for each month during this period.

\begin{knitrout}
\definecolor{shadecolor}{rgb}{0.969, 0.969, 0.969}\color{fgcolor}\begin{kframe}
\begin{verbatim}
## Climate Normals (1961-1990) for TMAX:
##    MONTH    NORMAL
## 1      1  4.380645
## 2      2  7.723009
## 3      3 11.184409
## 4      4 15.546042
## 5      5 20.561123
## 6      6 25.246000
## 7      7 29.835699
## 8      8 29.369032
## 9      9 25.148562
## 10    10 19.123763
## 11    11  9.909540
## 12    12  5.003656
\end{verbatim}
\end{kframe}
\end{knitrout}

\begin{shaded}
\begin{Verbatim}[fontsize=\small]
# Subset data to the normal period (1961-1990)
normal_period <- subset(station_b, 
                        Ymd >= as.Date("1961-01-01") &
                        Ymd <= as.Date("1990-12-31") &
                        ELEMENT == "TMAX")

# Calculate normals: mean TMAX for each month
tmax_normals <- aggregate(VALUE ~ MONTH, 
                          data = normal_period,
                          FUN = mean, na.rm = TRUE)
names(tmax_normals) <- c("MONTH", "NORMAL")
\end{Verbatim}
\end{shaded}

\subsection{Calculating Anomalies}

\begin{knitrout}
\definecolor{shadecolor}{rgb}{0.969, 0.969, 0.969}\color{fgcolor}\begin{kframe}
\begin{verbatim}
## Sample of anomaly data:
##     MONTH YEAR      TMAX    NORMAL     ANOMALY
## 1       1 1870  4.506452  4.380645  0.12580645
## 270     2 1870  7.021429  7.723009 -0.70158028
## 369     3 1870  7.141935 11.184409 -4.04247312
## 492     4 1870 16.060000 15.546042  0.51395764
## 627     5 1870 20.641935 20.561123  0.08081237
## 749     6 1870 25.926667 25.246000  0.68066667
\end{verbatim}
\end{kframe}
\end{knitrout}

\begin{shaded}
\begin{Verbatim}[fontsize=\small]
# Merge monthly values with normals by MONTH
monthly_with_normals <- merge(monthly_tmax, tmax_normals, 
                              by = "MONTH")

# Calculate anomaly = observed - normal
monthly_with_normals$ANOMALY <- monthly_with_normals$TMAX - 
                                 monthly_with_normals$NORMAL
\end{Verbatim}
\end{shaded}

\subsection{Visualizing Anomalies}

\begin{knitrout}
\definecolor{shadecolor}{rgb}{0.969, 0.969, 0.969}\color{fgcolor}
\includegraphics[width=\maxwidth]{figure/plot-anomalies-1} 
\end{knitrout}

\begin{shaded}
\begin{Verbatim}[fontsize=\small]
# Plot anomalies with colors based on sign
plot(ANOMALY ~ Date, data = monthly_with_normals,
     pch = 19, cex = 0.5,
     col = ifelse(ANOMALY > 0, "red", "blue"),  # red=warm, blue=cool
     main = "Monthly TMAX Anomalies",
     xlab = "Date", ylab = "Temperature Anomaly (°C)")

# Add reference line at zero
abline(h = 0, lty = 2, col = "gray")

# Add trend line using linear regression
trend <- lm(ANOMALY ~ Date, data = monthly_with_normals)
abline(trend, col = "black", lwd = 2)
\end{Verbatim}
\end{shaded}

\section{Process All Stations}

Now we apply these techniques to all 50 stations.

\begin{knitrout}
\definecolor{shadecolor}{rgb}{0.969, 0.969, 0.969}\color{fgcolor}\begin{kframe}
\begin{verbatim}
## Loading station data from RData file...
## ===========================================================
##   Processing All Stations for Spatial Analysis
## ===========================================================
## Processing 50 stations...
## 
## [1/50] USC00043157 ... OK
## [2/50] USW00023271 ... OK
## [3/50] USC00042294 ... OK
## [4/50] USC00046074 ... OK
## [5/50] USC00046136 ... OK
## [6/50] USC00046719 ... OK
## [7/50] USC00046826 ... OK
## [8/50] USC00047821 ... OK
## [9/50] USC00047902 ... OK
## [10/50] USC00049866 ... OK
## [11/50] USC00040693 ... OK
## [12/50] USC00044412 ... OK
## [13/50] USC00048351 ... OK
## [14/50] USC00041614 ... OK
## [15/50] USC00046730 ... OK
## [16/50] USC00047880 ... OK
## [17/50] USC00044259 ... OK
## [18/50] USC00049073 ... OK
## [19/50] USC00043161 ... OK
## [20/50] USC00047195 ... OK
## [21/50] USC00048702 ... OK
## [22/50] USW00023157 ... OK
## [23/50] USC00040943 ... OK
## [24/50] USC00043875 ... OK
## [25/50] USC00042805 ... OK
## [26/50] USC00043191 ... OK
## [27/50] USC00048839 ... OK
## [28/50] USC00049367 ... OK
## [29/50] USC00043747 ... OK
## [30/50] USC00044890 ... OK
## [31/50] USC00042239 ... OK
## [32/50] USC00045532 ... OK
## [33/50] USC00049452 ... OK
## [34/50] USC00041018 ... OK
## [35/50] USC00044223 ... OK
## [36/50] USC00044500 ... OK
## [37/50] USC00047965 ... OK
## [38/50] USC00046168 ... OK
## [39/50] USC00044997 ... OK
## [40/50] USC00046508 ... OK
## [41/50] USC00046506 ... OK
## [42/50] USC00046252 ... OK
## [43/50] USC00040383 ... OK
## [44/50] USC00041912 ... OK
## [45/50] USC00046399 ... OK
## [46/50] USC00049855 ... OK
## [47/50] USC00048353 ... OK
## [48/50] USC00049699 ... OK
## [49/50] USW00093134 ... OK
## [50/50] USC00041072 ... OK
## ===========================================================
## Processing Summary:
## ===========================================================
##   Successfully processed: 50 stations
##   Time elapsed: 0.7 minutes
## ===========================================================
## [OK] Saved spatial trends data
\end{verbatim}
\end{kframe}
\end{knitrout}

\begin{shaded}
\begin{Verbatim}[fontsize=\small]
# Process all stations at once
# This function:
#   1. Fixes dates for each station
#   2. Converts units
#   3. Calculates monthly values
#   4. Calculates normals
#   5. Calculates anomalies
#   6. Fits trend lines
#   7. Saves results

all_station_trends <- process_all_stations_for_spatial(verbose = TRUE)
\end{Verbatim}
\end{shaded}

\section{Create Spatial Objects}

Finally, we convert our trend data to spatial format for mapping in Lab 3.

\begin{knitrout}
\definecolor{shadecolor}{rgb}{0.969, 0.969, 0.969}\color{fgcolor}\begin{kframe}
\begin{verbatim}
## [OK] Created spatial objects:
##      trends_sf: 50 features
##      trends_sp: 50 features
\end{verbatim}
\end{kframe}
\end{knitrout}

\begin{shaded}
\begin{Verbatim}[fontsize=\small]
# Create spatial objects for mapping
# This creates:
#   trends_sf - Simple Features format (modern)
#   trends_sp - Spatial Points format (legacy, for kriging)

spatial_objects <- create_spatial_objects(all_station_trends)
\end{Verbatim}
\end{shaded}

\section{Summary Statistics}

\begin{knitrout}
\definecolor{shadecolor}{rgb}{0.969, 0.969, 0.969}\color{fgcolor}\begin{kframe}
\begin{verbatim}
## ===========================================================
## CLIMATE TRENDS SUMMARY FOR CA
## ===========================================================
## 
## Stations analyzed: 50
## 
## Mean Annual Trends:
##   TMAX: 0.56 °C/century
##   TMIN: 1.49 °C/century
##   PRCP: -3.3 mm/century
## ===========================================================
\end{verbatim}
\end{kframe}
\end{knitrout}

\section{Lab 1 Summary: R Concepts Learned}

In this lab, you learned:

\begin{enumerate}
  \item \textbf{Date handling}
  \begin{itemize}
    \item \texttt{as.Date()} converts text to dates
    \item Format codes like \texttt{\%Y\%m\%d}
    \item Extracting parts with \texttt{format()}
  \end{itemize}
  
  \item \textbf{Data subsetting}
  \begin{itemize}
    \item \texttt{subset()} for filtering rows
    \item Logical operators (\texttt{==}, \texttt{\&}, \texttt{|}, \texttt{\%in\%})
    \item Square brackets for rows and columns
  \end{itemize}
  
  \item \textbf{Data aggregation}
  \begin{itemize}
    \item \texttt{aggregate()} for group summaries
    \item Formula syntax (\texttt{y \textasciitilde{} x})
    \item Common functions (mean, sum, etc.)
  \end{itemize}
  
  \item \textbf{Basic plotting}
  \begin{itemize}
    \item \texttt{plot()} for scatter/line plots
    \item \texttt{hist()} for histograms
    \item \texttt{par()} for multiple panels
    \item Adding elements with \texttt{abline()}
  \end{itemize}
  
  \item \textbf{Climate concepts}
  \begin{itemize}
    \item Climate normals (1961-1990 baseline)
    \item Anomalies (deviation from normal)
    \item Trends (change over time)
  \end{itemize}
\end{enumerate}

\section{Next Steps}

You're ready for Lab 2, where you will:
\begin{itemize}
  \item Analyze monthly and seasonal trend patterns
  \item Learn statistical significance testing
  \item Compare warming patterns across your state
  \item Create more advanced visualizations
\end{itemize}

\end{document}
