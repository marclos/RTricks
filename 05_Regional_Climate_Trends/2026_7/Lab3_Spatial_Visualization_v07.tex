\documentclass{article}\usepackage[]{graphicx}\usepackage[]{xcolor}
% maxwidth is the original width if it is less than linewidth
% otherwise use linewidth (to make sure the graphics do not exceed the margin)
\makeatletter
\def\maxwidth{ %
  \ifdim\Gin@nat@width>\linewidth
    \linewidth
  \else
    \Gin@nat@width
  \fi
}
\makeatother

\definecolor{fgcolor}{rgb}{0.345, 0.345, 0.345}
\newcommand{\hlnum}[1]{\textcolor[rgb]{0.686,0.059,0.569}{#1}}%
\newcommand{\hlstr}[1]{\textcolor[rgb]{0.192,0.494,0.8}{#1}}%
\newcommand{\hlcom}[1]{\textcolor[rgb]{0.678,0.584,0.686}{\textit{#1}}}%
\newcommand{\hlopt}[1]{\textcolor[rgb]{0,0,0}{#1}}%
\newcommand{\hlstd}[1]{\textcolor[rgb]{0.345,0.345,0.345}{#1}}%
\newcommand{\hlkwa}[1]{\textcolor[rgb]{0.161,0.373,0.58}{\textbf{#1}}}%
\newcommand{\hlkwb}[1]{\textcolor[rgb]{0.69,0.353,0.396}{#1}}%
\newcommand{\hlkwc}[1]{\textcolor[rgb]{0.333,0.667,0.333}{#1}}%
\newcommand{\hlkwd}[1]{\textcolor[rgb]{0.737,0.353,0.396}{\textbf{#1}}}%
\let\hlipl\hlkwb

\usepackage{framed}
\makeatletter
\newenvironment{kframe}{%
 \def\at@end@of@kframe{}%
 \ifinner\ifhmode%
  \def\at@end@of@kframe{\end{minipage}}%
  \begin{minipage}{\columnwidth}%
 \fi\fi%
 \def\FrameCommand##1{\hskip\@totalleftmargin \hskip-\fboxsep
 \colorbox{shadecolor}{##1}\hskip-\fboxsep
     % There is no \\@totalrightmargin, so:
     \hskip-\linewidth \hskip-\@totalleftmargin \hskip\columnwidth}%
 \MakeFramed {\advance\hsize-\width
   \@totalleftmargin\z@ \linewidth\hsize
   \@setminipage}}%
 {\par\unskip\endMakeFramed%
 \at@end@of@kframe}
\makeatother

\definecolor{shadecolor}{rgb}{.97, .97, .97}
\definecolor{messagecolor}{rgb}{0, 0, 0}
\definecolor{warningcolor}{rgb}{1, 0, 1}
\definecolor{errorcolor}{rgb}{1, 0, 0}
\newenvironment{knitrout}{}{} % an empty environment to be redefined in TeX

\usepackage{alltt}
\usepackage{hyperref}
\hypersetup{colorlinks=true, linkcolor=blue, urlcolor=blue}
\usepackage{graphicx}
\usepackage{fancyvrb}
\usepackage{xcolor}
\usepackage{framed}

\definecolor{shadecolor}{RGB}{248,248,248}

\title{Lab 3: Spatial Visualization with Heat Mapping\\
\large{Regional Climate Trends Project}\\
\normalsize{Learning Spatial Analysis in R}}
\author{EA030}
\date{\today~v. 0.07}
\IfFileExists{upquote.sty}{\usepackage{upquote}}{}
\begin{document}
\maketitle

\tableofcontents
\newpage

\section{Introduction}

In Lab 3, you will create professional spatial visualizations showing climate trends across your state. You will learn:

\begin{itemize}
  \item Spatial data structures in R (sf and sp)
  \item Kriging interpolation concepts
  \item Creating heat maps with ggplot2
  \item Combining multiple maps
  \item Exporting publication-quality figures
\end{itemize}

\subsection{Why Heat Maps?}

Heat maps transform scattered point data into continuous surfaces, making spatial patterns easier to see and communicate.

\section{Setup}

\begin{knitrout}
\definecolor{shadecolor}{rgb}{0.969, 0.969, 0.969}\color{fgcolor}\begin{kframe}
\begin{verbatim}
## 
## ===========================================================
##  Climate Narratives Functions v7.0 Loaded Successfully!
## ===========================================================
##  Run check_packages() to verify dependencies
## ===========================================================
## 
##  QUICK START:
##  1. setup_project('CA')
##  2. select_stations_for_analysis(n_stations = 50)
##  3. download_stations()
##  4. load_and_save_stations(cleanup = TRUE)
##  5. process_all_stations_for_spatial()
##  6. create_spatial_objects(all_station_trends)
## 
##  NEW IN v7.0:
##  - Fixed figuresfolder variable handling
##  - Improved error messages
##  - Better documentation for teaching
## 
## ===========================================================
\end{verbatim}
\end{kframe}
\end{knitrout}

\begin{shaded}
\begin{Verbatim}[fontsize=\small]
# Set working directory (CHANGE to your path!)
setwd("/path/to/your/project/folder/")

# Load functions and packages
source("ClimateNarrativesFunctions_v07.R")
check_packages()
\end{Verbatim}
\end{shaded}

\section{Load Data}

\begin{knitrout}
\definecolor{shadecolor}{rgb}{0.969, 0.969, 0.969}\color{fgcolor}\begin{kframe}
\begin{verbatim}
## Loaded variables: all_station_trends
## [OK] Created spatial objects:
##      trends_sf: 50 features
##      trends_sp: 50 features
## 
## [OK] Loaded spatial data for CA
##      Stations: 50
##      Longitude: -123.76 to -115.57
##      Latitude: 32.85 to 41.87
\end{verbatim}
\end{kframe}
\end{knitrout}

\begin{shaded}
\begin{Verbatim}[fontsize=\small]
# Set your state
my.state <- "CA"

# IMPORTANT: Define folder paths before saving anything
datafolder <- "Data/"
figuresfolder <- "Figures/"

# Create Figures folder if needed
if (!dir.exists(figuresfolder)) {
  dir.create(figuresfolder, recursive = TRUE)
}

# Load the processed data
rdata_file <- paste0(datafolder, "spatial_trends_", my.state, ".RData")
load(rdata_file)

# Create spatial objects for mapping
spatial_objects <- create_spatial_objects(all_station_trends)
\end{Verbatim}
\end{shaded}

\textbf{R Concept: Directory Creation}

\texttt{dir.exists()} checks if a folder exists, and \texttt{dir.create()} creates it. The \texttt{recursive = TRUE} option creates parent folders if needed.

\section{Learning R: Spatial Data}

R has two main systems for spatial data:
\begin{itemize}
  \item \textbf{sf} (Simple Features): Modern, tidy approach
  \item \textbf{sp} (Spatial): Legacy system, still used by some packages
\end{itemize}

\subsection{The sf Package}

\begin{knitrout}
\definecolor{shadecolor}{rgb}{0.969, 0.969, 0.969}\color{fgcolor}\begin{kframe}
\begin{verbatim}
## Structure of trends_sf (Simple Features):
## ==========================================
## Class:     sf
## Features:  50
## CRS:       4326 (WGS84 lat/lon)
## 
## First few rows:
##            ID LATITUDE LONGITUDE annual_trend_TMAX
## 1 USC00043157  41.8714 -120.1575        0.16117902
## 2 USW00023271  38.5553 -121.4183        2.53534598
## 3 USC00042294  38.5350 -121.7761       -0.04812917
## 4 USC00046074  38.2778 -122.2647        0.78330318
## 5 USC00046136  39.2467 -121.0008       -0.97059534
## 6 USC00046719  34.1483 -118.1447        2.38718896
\end{verbatim}
\end{kframe}
\end{knitrout}

\begin{shaded}
\begin{Verbatim}[fontsize=\small]
# sf objects look like data frames with a geometry column
head(trends_sf)

# Check the coordinate reference system
st_crs(trends_sf)

# Get the bounding box
st_bbox(trends_sf)
\end{Verbatim}
\end{shaded}

\subsection{Converting Between sf and sp}

\begin{knitrout}
\definecolor{shadecolor}{rgb}{0.969, 0.969, 0.969}\color{fgcolor}\begin{kframe}
\begin{verbatim}
## Converting between sf and sp:
## ==============================
## trends_sf class: sf
## trends_sp class: SpatialPointsDataFrame
\end{verbatim}
\end{kframe}
\end{knitrout}

\begin{shaded}
\begin{Verbatim}[fontsize=\small]
# sf to sp (for packages that need sp)
trends_sp <- as(trends_sf, "Spatial")

# sp to sf
trends_sf <- st_as_sf(trends_sp)
\end{Verbatim}
\end{shaded}

\section{Understanding Kriging}

\textbf{Kriging} is a geostatistical interpolation method that:
\begin{enumerate}
  \item Assumes nearby points have similar values
  \item Uses a \textbf{variogram} to model spatial correlation
  \item Predicts values at unsampled locations
  \item Provides prediction uncertainty estimates
\end{enumerate}

\subsection{The Variogram}

A variogram describes how similar values are based on distance:

\begin{knitrout}
\definecolor{shadecolor}{rgb}{0.969, 0.969, 0.969}\color{fgcolor}
\includegraphics[width=\maxwidth]{figure/variogram-example-1} 
\end{knitrout}

\begin{shaded}
\begin{Verbatim}[fontsize=\small]
# Calculate empirical variogram
v <- variogram(annual_trend_TMAX ~ 1, trends_sp)

# Plot it
plot(v, main = "Variogram for Annual TMAX Trend")

# The variogram shows:
#   - Points close together are similar (low semivariance)
#   - Points far apart are different (high semivariance)
#   - The "range" is where the curve levels off
\end{Verbatim}
\end{shaded}

\section{Station Location Map}

Let's first see where our stations are located.

\begin{knitrout}
\definecolor{shadecolor}{rgb}{0.969, 0.969, 0.969}\color{fgcolor}
\includegraphics[width=\maxwidth]{figure/station-map-1} 
\end{knitrout}

\begin{shaded}
\begin{Verbatim}[fontsize=\small]
# Get state boundary
us_states <- map_data("state")
state_name <- tolower(state.name[state.abb == my.state])
state_boundary <- us_states %>% filter(region == state_name)

# Create map with ggplot2
ggplot() +
  # State polygon
  geom_polygon(data = state_boundary,
               aes(x = long, y = lat, group = group),
               fill = "gray95", color = "gray50") +
  # Station points colored by trend
  geom_sf(data = trends_sf,
          aes(color = annual_trend_TMAX),
          size = 3) +
  # Color scale
  scale_color_viridis_c(name = "Trend")
\end{Verbatim}
\end{shaded}

\section{Creating Heat Maps}

Now let's create interpolated heat maps using kriging.

\subsection{Annual Maximum Temperature (TMAX)}

\begin{knitrout}
\definecolor{shadecolor}{rgb}{0.969, 0.969, 0.969}\color{fgcolor}\begin{kframe}
\begin{verbatim}
## Fitting variogram for annual_trend_TMAX ...
## Performing kriging interpolation...
## [using ordinary kriging]
\end{verbatim}
\end{kframe}
\includegraphics[width=\maxwidth]{figure/tmax-annual-heatmap-1} 
\begin{kframe}\begin{verbatim}
## 
## [OK] Saved: Heatmap_TMAX_annual_CA.png
\end{verbatim}
\end{kframe}
\end{knitrout}

\begin{shaded}
\begin{Verbatim}[fontsize=\small]
# Create heat map using our function
map_tmax_annual <- create_heatmap(
  trends_sp,                    # Spatial data
  trend_var = "annual_trend_TMAX",  # Variable to map
  title = paste(my.state, "- Annual TMAX Trend"),
  subtitle = "Change in deg C per 100 years",
  state = my.state,
  colors = "temp",              # Color scheme
  resolution = 0.1              # Grid resolution
)

# Display it
print(map_tmax_annual)

# Save high-resolution version
ggsave(paste0(figuresfolder, "Heatmap_TMAX_annual_", my.state, ".png"),
       map_tmax_annual,
       width = 10, height = 8, dpi = 300)
\end{Verbatim}
\end{shaded}

\textbf{R Concept: ggsave()}

\texttt{ggsave()} saves ggplot figures with professional quality:
\begin{itemize}
  \item \texttt{width}, \texttt{height}: Size in inches
  \item \texttt{dpi = 300}: Publication quality resolution
  \item File format determined by extension (.png, .pdf, .jpg)
\end{itemize}

\subsection{Annual Minimum Temperature (TMIN)}

\begin{knitrout}
\definecolor{shadecolor}{rgb}{0.969, 0.969, 0.969}\color{fgcolor}\begin{kframe}
\begin{verbatim}
## Fitting variogram for annual_trend_TMIN ...
## Performing kriging interpolation...
## [using ordinary kriging]
\end{verbatim}
\end{kframe}
\includegraphics[width=\maxwidth]{figure/tmin-annual-heatmap-1} 
\begin{kframe}\begin{verbatim}
## [OK] Saved: Heatmap_TMIN_annual_CA.png
\end{verbatim}
\end{kframe}
\end{knitrout}

\section{Seasonal Heat Maps}

Let's compare trends across seasons.

\subsection{Summer vs Winter TMAX}

\begin{knitrout}
\definecolor{shadecolor}{rgb}{0.969, 0.969, 0.969}\color{fgcolor}\begin{kframe}
\begin{verbatim}
## Fitting variogram for summer_trend_TMAX ...
## Performing kriging interpolation...
## [using ordinary kriging]
\end{verbatim}
\end{kframe}
\includegraphics[width=\maxwidth]{figure/seasonal-heatmaps-1} 
\end{knitrout}

\begin{knitrout}
\definecolor{shadecolor}{rgb}{0.969, 0.969, 0.969}\color{fgcolor}\begin{kframe}
\begin{verbatim}
## Fitting variogram for winter_trend_TMAX ...
## Performing kriging interpolation...
## [using ordinary kriging]
\end{verbatim}
\end{kframe}
\includegraphics[width=\maxwidth]{figure/winter-heatmap-1} 
\end{knitrout}

\section{Learning R: Combining Plots}

The \texttt{patchwork} package makes it easy to combine multiple plots.

\begin{knitrout}
\definecolor{shadecolor}{rgb}{0.969, 0.969, 0.969}\color{fgcolor}\begin{kframe}
\begin{verbatim}
## Fitting variogram for winter_trend_TMAX ...
## Performing kriging interpolation...
## [using ordinary kriging]
## Fitting variogram for spring_trend_TMAX ...
## Performing kriging interpolation...
## [using ordinary kriging]
## Fitting variogram for summer_trend_TMAX ...
## Performing kriging interpolation...
## [using ordinary kriging]
## Fitting variogram for fall_trend_TMAX ...
## Performing kriging interpolation...
## [using ordinary kriging]
\end{verbatim}
\end{kframe}
\includegraphics[width=\maxwidth]{figure/seasonal-panel-1} 
\begin{kframe}\begin{verbatim}
## 
## [OK] Saved: Seasonal_TMAX_comparison_CA.png
\end{verbatim}
\end{kframe}
\end{knitrout}

\begin{shaded}
\begin{Verbatim}[fontsize=\small]
library(patchwork)

# patchwork operators:
#   |   = side by side
#   /   = stacked vertically
#   +   = add annotation

seasonal_comparison <- (winter | spring) / 
                       (summer | fall) +
  plot_annotation(title = "Seasonal Comparison")

print(seasonal_comparison)
\end{Verbatim}
\end{shaded}

\section{Precipitation Heat Map}

\begin{knitrout}
\definecolor{shadecolor}{rgb}{0.969, 0.969, 0.969}\color{fgcolor}\begin{kframe}
\begin{verbatim}
## Fitting variogram for annual_trend_PRCP ...
## Performing kriging interpolation...
## [using ordinary kriging]
\end{verbatim}
\end{kframe}
\includegraphics[width=\maxwidth]{figure/prcp-heatmap-1} 
\begin{kframe}\begin{verbatim}
## [OK] Saved: Heatmap_PRCP_annual_CA.png
\end{verbatim}
\end{kframe}
\end{knitrout}

\section{Regional Summary}

\begin{knitrout}
\definecolor{shadecolor}{rgb}{0.969, 0.969, 0.969}\color{fgcolor}\begin{kframe}
\begin{verbatim}
## ===========================================================
## REGIONAL SUMMARY - ANNUAL TMAX TRENDS
## ===========================================================
## # A tibble: 4 x 3
##   region        n mean_trend
##   <chr>     <int>      <dbl>
## 1 Northeast     7       1.01
## 2 Northwest    18       0.37
## 3 Southeast    18       0.38
## 4 Southwest     7       1.08
## ===========================================================
\end{verbatim}
\end{kframe}
\end{knitrout}

\section{Export Summary}

\begin{knitrout}
\definecolor{shadecolor}{rgb}{0.969, 0.969, 0.969}\color{fgcolor}\begin{kframe}
\begin{verbatim}
## ===========================================================
##               EXPORTED HEAT MAPS
## ===========================================================
## Total figures created: 6
## -----------------------------------------------------------
##  * Heatmap_PRCP_annual_CA.png 
##  * Heatmap_Summer_TMAX_CA.png 
##  * Heatmap_TMAX_annual_CA.png 
##  * Heatmap_TMIN_annual_CA.png 
##  * Heatmap_Winter_TMAX_CA.png 
##  * Seasonal_TMAX_comparison_CA.png
## -----------------------------------------------------------
## Location: Figures/
## Resolution: 300 dpi (publication quality)
## ===========================================================
\end{verbatim}
\end{kframe}
\end{knitrout}

\section{Interpreting Your Maps}

When examining heat maps, consider:

\begin{enumerate}
  \item \textbf{Spatial Patterns}
  \begin{itemize}
    \item Are there distinct regional differences?
    \item Where are the hotspots (strongest warming)?
    \item Do coastal vs. inland areas differ?
  \end{itemize}
  
  \item \textbf{Seasonal Variation}
  \begin{itemize}
    \item Which season shows strongest trends?
    \item Do patterns differ by season?
  \end{itemize}
  
  \item \textbf{Temperature vs. Precipitation}
  \begin{itemize}
    \item Do patterns align?
    \item Are warming areas also drying/wetting?
  \end{itemize}
\end{enumerate}

\section{Lab 3 Summary}

\subsection{What You Accomplished}

\begin{itemize}
  \item Created interpolated heat maps using kriging
  \item Visualized annual and seasonal temperature trends
  \item Mapped precipitation changes
  \item Identified regional climate hotspots
  \item Generated 6 publication-quality figures
\end{itemize}

\subsection{R Concepts Learned}

\begin{enumerate}
  \item \textbf{Spatial Data:} sf and sp objects
  \item \textbf{Kriging:} Variograms, interpolation
  \item \textbf{ggplot2:} Layers, scales, themes
  \item \textbf{patchwork:} Combining plots
  \item \textbf{ggsave:} Exporting figures
\end{enumerate}

\subsection{Key Results for CA}

\begin{itemize}
  \item Mean TMAX trend: \textbf{0.56 °C/century}
  \item Spatial range: \textbf{-1.68 to 2.59 °C/century}
\end{itemize}

\subsection{Using These Maps in Your Video}

\begin{enumerate}
  \item Start with annual TMAX (overview)
  \item Show seasonal variation for nuance
  \item Zoom into specific hotspots
  \item Compare temperature and precipitation
  \item Connect to community impacts
\end{enumerate}

\textbf{You now have everything needed to create a compelling climate narrative video!}

\end{document}
