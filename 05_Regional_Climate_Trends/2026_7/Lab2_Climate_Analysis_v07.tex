\documentclass{article}\usepackage[]{graphicx}\usepackage[]{xcolor}
% maxwidth is the original width if it is less than linewidth
% otherwise use linewidth (to make sure the graphics do not exceed the margin)
\makeatletter
\def\maxwidth{ %
  \ifdim\Gin@nat@width>\linewidth
    \linewidth
  \else
    \Gin@nat@width
  \fi
}
\makeatother

\definecolor{fgcolor}{rgb}{0.345, 0.345, 0.345}
\newcommand{\hlnum}[1]{\textcolor[rgb]{0.686,0.059,0.569}{#1}}%
\newcommand{\hlstr}[1]{\textcolor[rgb]{0.192,0.494,0.8}{#1}}%
\newcommand{\hlcom}[1]{\textcolor[rgb]{0.678,0.584,0.686}{\textit{#1}}}%
\newcommand{\hlopt}[1]{\textcolor[rgb]{0,0,0}{#1}}%
\newcommand{\hlstd}[1]{\textcolor[rgb]{0.345,0.345,0.345}{#1}}%
\newcommand{\hlkwa}[1]{\textcolor[rgb]{0.161,0.373,0.58}{\textbf{#1}}}%
\newcommand{\hlkwb}[1]{\textcolor[rgb]{0.69,0.353,0.396}{#1}}%
\newcommand{\hlkwc}[1]{\textcolor[rgb]{0.333,0.667,0.333}{#1}}%
\newcommand{\hlkwd}[1]{\textcolor[rgb]{0.737,0.353,0.396}{\textbf{#1}}}%
\let\hlipl\hlkwb

\usepackage{framed}
\makeatletter
\newenvironment{kframe}{%
 \def\at@end@of@kframe{}%
 \ifinner\ifhmode%
  \def\at@end@of@kframe{\end{minipage}}%
  \begin{minipage}{\columnwidth}%
 \fi\fi%
 \def\FrameCommand##1{\hskip\@totalleftmargin \hskip-\fboxsep
 \colorbox{shadecolor}{##1}\hskip-\fboxsep
     % There is no \\@totalrightmargin, so:
     \hskip-\linewidth \hskip-\@totalleftmargin \hskip\columnwidth}%
 \MakeFramed {\advance\hsize-\width
   \@totalleftmargin\z@ \linewidth\hsize
   \@setminipage}}%
 {\par\unskip\endMakeFramed%
 \at@end@of@kframe}
\makeatother

\definecolor{shadecolor}{rgb}{.97, .97, .97}
\definecolor{messagecolor}{rgb}{0, 0, 0}
\definecolor{warningcolor}{rgb}{1, 0, 1}
\definecolor{errorcolor}{rgb}{1, 0, 0}
\newenvironment{knitrout}{}{} % an empty environment to be redefined in TeX

\usepackage{alltt}
\usepackage{hyperref}
\hypersetup{colorlinks=true, linkcolor=blue, urlcolor=blue}
\usepackage{graphicx}
\usepackage{booktabs}
\usepackage{fancyvrb}
\usepackage{xcolor}
\usepackage{framed}

\definecolor{shadecolor}{RGB}{248,248,248}

\title{Lab 2: Climate Trend Analysis\\
\large{Regional Climate Trends Project}\\
\normalsize{Learning Statistical Analysis in R}}
\author{EA030}
\date{\today~v. 0.07}
\IfFileExists{upquote.sty}{\usepackage{upquote}}{}
\begin{document}
\maketitle

\tableofcontents
\newpage

\section{Introduction}

In Lab 2, you will learn statistical analysis of climate trends. Building on Lab 1's data preparation, we focus on:

\begin{itemize}
  \item Descriptive statistics in R
  \item Histograms and distributions
  \item Linear regression for trend analysis
  \item Interpreting p-values and significance
  \item Comparing groups (seasons, regions)
\end{itemize}

\section{Setup}

\begin{knitrout}
\definecolor{shadecolor}{rgb}{0.969, 0.969, 0.969}\color{fgcolor}\begin{kframe}
\begin{verbatim}
## 
## ===========================================================
##  Climate Narratives Functions v7.0 Loaded Successfully!
## ===========================================================
##  Run check_packages() to verify dependencies
## ===========================================================
## 
##  QUICK START:
##  1. setup_project('CA')
##  2. select_stations_for_analysis(n_stations = 50)
##  3. download_stations()
##  4. load_and_save_stations(cleanup = TRUE)
##  5. process_all_stations_for_spatial()
##  6. create_spatial_objects(all_station_trends)
## 
##  NEW IN v7.0:
##  - Fixed figuresfolder variable handling
##  - Improved error messages
##  - Better documentation for teaching
## 
## ===========================================================
\end{verbatim}
\end{kframe}
\end{knitrout}

\begin{shaded}
\begin{Verbatim}[fontsize=\small]
setwd("/path/to/your/project/folder/")
source("ClimateNarrativesFunctions_v07.R")
check_packages()
\end{Verbatim}
\end{shaded}

\section{Load Data from Lab 1}

\begin{knitrout}
\definecolor{shadecolor}{rgb}{0.969, 0.969, 0.969}\color{fgcolor}\begin{kframe}
\begin{verbatim}
## [OK] Loaded data for CA : 50 stations
\end{verbatim}
\end{kframe}
\end{knitrout}

\begin{shaded}
\begin{Verbatim}[fontsize=\small]
my.state <- "CA"
datafolder <- "Data/"
figuresfolder <- "Figures/"

rdata_file <- paste0(datafolder, "spatial_trends_", my.state, ".RData")
load(rdata_file)
\end{Verbatim}
\end{shaded}

\section{Learning R: Summary Statistics}

\subsection{Basic Summary Functions}

\begin{knitrout}
\definecolor{shadecolor}{rgb}{0.969, 0.969, 0.969}\color{fgcolor}\begin{kframe}
\begin{verbatim}
## Statistics for Annual TMAX Trend (°C/century):
## ========================================
## Mean:     0.56
## Median:   0.4
## Std Dev:  1.02
## Min:      -1.68
## Max:      2.59
\end{verbatim}
\end{kframe}
\end{knitrout}

\begin{shaded}
\begin{Verbatim}[fontsize=\small]
tmax_trends <- all_station_trends$annual_trend_TMAX

mean(tmax_trends, na.rm = TRUE)     # Average
median(tmax_trends, na.rm = TRUE)   # Middle value
sd(tmax_trends, na.rm = TRUE)       # Standard deviation
min(tmax_trends, na.rm = TRUE)      # Minimum
max(tmax_trends, na.rm = TRUE)      # Maximum
\end{Verbatim}
\end{shaded}

\textbf{R Concept: na.rm = TRUE}

Many R functions return \texttt{NA} if any values are missing. Adding \texttt{na.rm = TRUE} removes missing values before calculating.

\section{Learning R: Histograms}

Histograms show the distribution of data.

\begin{knitrout}
\definecolor{shadecolor}{rgb}{0.969, 0.969, 0.969}\color{fgcolor}
\includegraphics[width=\maxwidth]{figure/histogram-basic-1} 
\end{knitrout}

\begin{shaded}
\begin{Verbatim}[fontsize=\small]
hist(tmax_trends, breaks = 20, col = "coral",
     main = "Distribution of Annual TMAX Trends",
     xlab = "Trend (°C/century)")
abline(v = 0, col = "gray", lty = 2)
abline(v = mean(tmax_trends, na.rm = TRUE), col = "red", lwd = 2)
\end{Verbatim}
\end{shaded}

\section{Learning R: Linear Regression}

Linear regression finds the best-fit line: $y = a + bx$

\begin{knitrout}
\definecolor{shadecolor}{rgb}{0.969, 0.969, 0.969}\color{fgcolor}\begin{kframe}
\begin{verbatim}
## Linear Regression Results:
##                 Estimate  Std. Error   t value     Pr(>|t|)
## (Intercept) -30.35504722 8.250718385 -3.679079 0.0005083376
## YEAR          0.01539644 0.004145927  3.713631 0.0004552597
## 
## Trend: 1.54 °C/century
## p-value: 0.000455
## R-squared: 0.189
\end{verbatim}
\end{kframe}
\end{knitrout}

\begin{shaded}
\begin{Verbatim}[fontsize=\small]
# Fit linear model
model <- lm(ANOMALY ~ YEAR, data = demo_data)

# Get summary
summary(model)

# Extract slope (trend per year)
coef(model)[2]

# Convert to per century
coef(model)[2] * 100
\end{Verbatim}
\end{shaded}

\textbf{Understanding p-values:}
\begin{itemize}
  \item p $<$ 0.05: Statistically significant
  \item p $<$ 0.01: Highly significant
  \item p $<$ 0.001: Very highly significant
\end{itemize}

\begin{knitrout}
\definecolor{shadecolor}{rgb}{0.969, 0.969, 0.969}\color{fgcolor}
\includegraphics[width=\maxwidth]{figure/trend-plot-1} 
\end{knitrout}

\section{Seasonal Comparison}

\begin{knitrout}
\definecolor{shadecolor}{rgb}{0.969, 0.969, 0.969}\color{fgcolor}
\includegraphics[width=\maxwidth]{figure/boxplot-seasonal-1} 
\end{knitrout}

\begin{shaded}
\begin{Verbatim}[fontsize=\small]
seasonal_data <- data.frame(
  Winter = all_station_trends$winter_trend_TMAX,
  Spring = all_station_trends$spring_trend_TMAX,
  Summer = all_station_trends$summer_trend_TMAX,
  Fall = all_station_trends$fall_trend_TMAX
)

boxplot(seasonal_data, col = c("lightblue", "lightgreen", "coral", "orange"),
        main = "TMAX Trends by Season", ylab = "Trend (°C/century)")
\end{Verbatim}
\end{shaded}

\section{Geographic Patterns}

\begin{knitrout}
\definecolor{shadecolor}{rgb}{0.969, 0.969, 0.969}\color{fgcolor}
\includegraphics[width=\maxwidth]{figure/trend-latitude-1} 
\end{knitrout}

\section{TMAX vs TMIN Comparison}

\begin{knitrout}
\definecolor{shadecolor}{rgb}{0.969, 0.969, 0.969}\color{fgcolor}
\includegraphics[width=\maxwidth]{figure/tmax-tmin-compare-1} 
\end{knitrout}

\begin{knitrout}
\definecolor{shadecolor}{rgb}{0.969, 0.969, 0.969}\color{fgcolor}\begin{kframe}
\begin{verbatim}
## 
## ===========================================================
## DAYTIME vs NIGHTTIME WARMING
## ===========================================================
## Mean TMAX trend: 0.56 °C/century
## Mean TMIN trend: 1.49 °C/century
## Correlation:     -0.309
## ===========================================================
\end{verbatim}
\end{kframe}
\end{knitrout}

\section{Key Findings Summary}

\begin{knitrout}
\definecolor{shadecolor}{rgb}{0.969, 0.969, 0.969}\color{fgcolor}\begin{kframe}
\begin{verbatim}
## ===========================================================
##               KEY FINDINGS SUMMARY
## ===========================================================
## State: CA | Stations: 50
## -----------------------------------------------------------
## ANNUAL TRENDS:
##   TMAX: +0.56 °C/century
##   TMIN: +1.49 °C/century
##   PRCP: -3.3 mm/century
## -----------------------------------------------------------
## Stations warming: 39 / 50 (78%)
## ===========================================================
\end{verbatim}
\end{kframe}
\end{knitrout}

\section{Lab 2 Summary: R Concepts Learned}

\begin{enumerate}
  \item \textbf{Summary Statistics:} \texttt{mean()}, \texttt{median()}, \texttt{sd()}, \texttt{summary()}
  \item \textbf{Histograms:} \texttt{hist()}, \texttt{breaks}, \texttt{abline()}
  \item \textbf{Linear Regression:} \texttt{lm()}, \texttt{coef()}, \texttt{summary()}
  \item \textbf{Statistical Concepts:} p-value, R², correlation
  \item \textbf{Boxplots:} Comparing distributions across groups
\end{enumerate}

\section{Next Steps}

Lab 3: Create spatial heat maps, visualize regional patterns, generate publication-quality figures.

\end{document}
