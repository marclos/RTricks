\documentclass{article}\usepackage[]{graphicx}\usepackage[]{xcolor}
% maxwidth is the original width if it is less than linewidth
% otherwise use linewidth (to make sure the graphics do not exceed the margin)
\makeatletter
\def\maxwidth{ %
  \ifdim\Gin@nat@width>\linewidth
    \linewidth
  \else
    \Gin@nat@width
  \fi
}
\makeatother

\definecolor{fgcolor}{rgb}{0.345, 0.345, 0.345}
\newcommand{\hlnum}[1]{\textcolor[rgb]{0.686,0.059,0.569}{#1}}%
\newcommand{\hlstr}[1]{\textcolor[rgb]{0.192,0.494,0.8}{#1}}%
\newcommand{\hlcom}[1]{\textcolor[rgb]{0.678,0.584,0.686}{\textit{#1}}}%
\newcommand{\hlopt}[1]{\textcolor[rgb]{0,0,0}{#1}}%
\newcommand{\hlstd}[1]{\textcolor[rgb]{0.345,0.345,0.345}{#1}}%
\newcommand{\hlkwa}[1]{\textcolor[rgb]{0.161,0.373,0.58}{\textbf{#1}}}%
\newcommand{\hlkwb}[1]{\textcolor[rgb]{0.69,0.353,0.396}{#1}}%
\newcommand{\hlkwc}[1]{\textcolor[rgb]{0.333,0.667,0.333}{#1}}%
\newcommand{\hlkwd}[1]{\textcolor[rgb]{0.737,0.353,0.396}{\textbf{#1}}}%
\let\hlipl\hlkwb

\usepackage{framed}
\makeatletter
\newenvironment{kframe}{%
 \def\at@end@of@kframe{}%
 \ifinner\ifhmode%
  \def\at@end@of@kframe{\end{minipage}}%
  \begin{minipage}{\columnwidth}%
 \fi\fi%
 \def\FrameCommand##1{\hskip\@totalleftmargin \hskip-\fboxsep
 \colorbox{shadecolor}{##1}\hskip-\fboxsep
     % There is no \\@totalrightmargin, so:
     \hskip-\linewidth \hskip-\@totalleftmargin \hskip\columnwidth}%
 \MakeFramed {\advance\hsize-\width
   \@totalleftmargin\z@ \linewidth\hsize
   \@setminipage}}%
 {\par\unskip\endMakeFramed%
 \at@end@of@kframe}
\makeatother

\definecolor{shadecolor}{rgb}{.97, .97, .97}
\definecolor{messagecolor}{rgb}{0, 0, 0}
\definecolor{warningcolor}{rgb}{1, 0, 1}
\definecolor{errorcolor}{rgb}{1, 0, 0}
\newenvironment{knitrout}{}{} % an empty environment to be redefined in TeX

\usepackage{alltt}

\title{Obtaining Climate Data from NOAA}
\author{Marc Los Huertos}
\date{\today~(ver. 0.4)}
\IfFileExists{upquote.sty}{\usepackage{upquote}}{}
\begin{document}
\maketitle

\section{Introduction}

\subsection{Read Data}

\begin{knitrout}
\definecolor{shadecolor}{rgb}{0.969, 0.969, 0.969}\color{fgcolor}\begin{kframe}
\begin{alltt}
\hlkwd{library}\hlstd{(here)}
\end{alltt}


{\ttfamily\noindent\itshape\color{messagecolor}{\#\# here() starts at /home/mwl04747/RTricks}}\begin{alltt}
\hlkwd{library}\hlstd{(xtable)}

\hlstd{stations.active.oldest} \hlkwb{=} \hlkwd{read.csv}\hlstd{(}\hlkwd{here}\hlstd{(}\hlstr{"04_Regional_Climate_Trends"}\hlstd{,} \hlstr{"stations.active.oldest.csv"}\hlstd{))}
\end{alltt}
\end{kframe}
\end{knitrout}

\subsection{Select and Evaluate State Data}

\begin{knitrout}
\definecolor{shadecolor}{rgb}{0.969, 0.969, 0.969}\color{fgcolor}\begin{kframe}
\begin{alltt}
\hlstd{stations.unique} \hlkwb{=} \hlkwd{unique}\hlstd{(stations.active.oldest[,}\hlkwd{c}\hlstd{(}\hlstr{"STATE"}\hlstd{,} \hlstr{"STATE_NAME"}\hlstd{)])}

\hlstd{xtab} \hlkwb{=} \hlkwd{xtable}\hlstd{(stations.unique)}
\end{alltt}
\end{kframe}
\end{knitrout}


\begin{knitrout}
\definecolor{shadecolor}{rgb}{0.969, 0.969, 0.969}\color{fgcolor}\begin{kframe}
\begin{alltt}
\hlstd{my.state} \hlkwb{=} \hlstr{"CA"}
\end{alltt}
\end{kframe}
\end{knitrout}

\section{Download Data from NOAA}

\subsection{Function to Download Data}


\begin{knitrout}
\definecolor{shadecolor}{rgb}{0.969, 0.969, 0.969}\color{fgcolor}\begin{kframe}
\begin{alltt}
\hlcom{# Select Stations in State}
\hlstd{my.stations} \hlkwb{=} \hlkwd{subset}\hlstd{(stations.active.oldest, STATE} \hlopt{==} \hlstd{my.state)}

\hlcom{# Download Updated Station Data}
\hlstd{i}\hlkwb{=}\hlnum{1}
\hlstd{here}\hlopt{::}\hlkwd{here}\hlstd{(}\hlstr{"04_Regional_Climate_Trends"}\hlstd{, my.stations}\hlopt{$}\hlstd{ID[i])}
\end{alltt}
\begin{verbatim}
## [1] "/home/mwl04747/RTricks/04_Regional_Climate_Trends/USC00043157"
\end{verbatim}
\begin{alltt}
\hlcom{#station = data.frame(NULL)}
\hlkwa{for}\hlstd{(i} \hlkwa{in} \hlnum{1}\hlopt{:}\hlkwd{nrow}\hlstd{(my.stations))\{}
  \hlstd{url} \hlkwb{=} \hlkwd{paste0}\hlstd{(}\hlstr{"https://www.ncei.noaa.gov/pub/data/ghcn/daily/by_station/"}\hlstd{, my.stations}\hlopt{$}\hlstd{ID[i],} \hlstr{".csv.gz"}\hlstd{)}

\hlkwd{print}\hlstd{(i)} \hlcom{# Print Index Number}
\hlkwd{download.file}\hlstd{(url,} \hlkwd{paste0}\hlstd{(here}\hlopt{::}\hlkwd{here}\hlstd{(}\hlstr{"04_Regional_Climate_Trends"}\hlstd{,} \hlstr{"Data"}\hlstd{,} \hlstr{"SP24/"}\hlstd{), my.stations}\hlopt{$}\hlstd{ID[i],} \hlstr{".csv.gz"}\hlstd{),} \hlkwc{quiet} \hlstd{=} \hlnum{FALSE}\hlstd{,} \hlkwc{mode} \hlstd{=} \hlstr{"w"}\hlstd{,} \hlkwc{cacheOK} \hlstd{=} \hlnum{TRUE}\hlstd{)}

\hlkwd{assign}\hlstd{(}\hlkwd{paste0}\hlstd{(}\hlstr{"station"}\hlstd{, i),} \hlkwd{read.csv}\hlstd{(}\hlkwd{gzfile}\hlstd{(}\hlkwd{paste0}\hlstd{(here}\hlopt{::}\hlkwd{here}\hlstd{(}\hlstr{"04_Regional_Climate_Trends"}\hlstd{,} \hlstr{"Data"}\hlstd{,} \hlstr{"SP24/"}\hlstd{),my.stations}\hlopt{$}\hlstd{ID[i],} \hlstr{".csv.gz"}\hlstd{)),} \hlkwc{header}\hlstd{=}\hlnum{FALSE}\hlstd{))}

\hlcom{# can't get the header named in loop!}
\hlcom{#names(paste0("station",i)) <- c("ID", "DATE", "ELEMENT", "VALUE", "M-FLAG", "Q-FLAG", "S-FLAG", "OBS-TIME")}

\hlstd{\}}
\end{alltt}
\begin{verbatim}
## [1] 1
## [1] 2
## [1] 3
## [1] 4
## [1] 5
\end{verbatim}
\begin{alltt}
\hlkwd{names}\hlstd{(station1)} \hlkwb{<-} \hlkwd{c}\hlstd{(}\hlstr{"ID"}\hlstd{,} \hlstr{"DATE"}\hlstd{,} \hlstr{"ELEMENT"}\hlstd{,} \hlstr{"VALUE"}\hlstd{,} \hlstr{"M-FLAG"}\hlstd{,} \hlstr{"Q-FLAG"}\hlstd{,} \hlstr{"S-FLAG"}\hlstd{,} \hlstr{"OBS-TIME"}\hlstd{)}
\hlkwd{names}\hlstd{(station3)} \hlkwb{<-} \hlkwd{names}\hlstd{(station2)} \hlkwb{<-} \hlkwd{names}\hlstd{(station1)}
\hlkwd{names}\hlstd{(station5)} \hlkwb{<-} \hlkwd{names}\hlstd{(station4)} \hlkwb{<-} \hlkwd{names}\hlstd{(station1)}

\hlcom{# NAMES OF VARIABLES ARE INCORRECT??}

  \hlcom{#ID = 11 character station identification code}
  \hlcom{#YEAR/MONTH/DAY = 8 character date in YYYYMMDD format (e.g. 19860529 = May 29, 1986)}
  \hlcom{#ELEMENT = 4 character indicator of element type }
  \hlcom{#DATA VALUE = 5 character data value for ELEMENT }
  \hlcom{#M-FLAG = 1 character Measurement Flag }
  \hlcom{#Q-FLAG = 1 character Quality Flag }
  \hlcom{#S-FLAG = 1 character Source Flag }
  \hlcom{#OBS-TIME = 4-character time of observation in hour-minute format (i.e. 0700 =7:00 am); if no ob time information }
 \hlcom{#is available, the field is left empty}
\end{alltt}
\end{kframe}
\end{knitrout}

\section{Process and Clean Data}

\subsection{Correct Values Units}

\begin{knitrout}
\definecolor{shadecolor}{rgb}{0.969, 0.969, 0.969}\color{fgcolor}\begin{kframe}
\begin{alltt}
\hlstd{station1}\hlopt{$}\hlstd{VALUE} \hlkwb{=} \hlstd{station1}\hlopt{$}\hlstd{VALUE}\hlopt{/}\hlnum{10}
\end{alltt}
\end{kframe}
\end{knitrout}

\subsection{Fix Dates!}
\begin{knitrout}
\definecolor{shadecolor}{rgb}{0.969, 0.969, 0.969}\color{fgcolor}\begin{kframe}
\begin{alltt}
\hlcom{#fix Dates}
\hlstd{station1}\hlopt{$}\hlstd{Ymd} \hlkwb{=} \hlkwd{as.Date}\hlstd{(}\hlkwd{as.character}\hlstd{(station1}\hlopt{$}\hlstd{DATE),} \hlkwc{format} \hlstd{=} \hlstr{"%Y%m%d"}\hlstd{)}
\hlkwd{str}\hlstd{(station1)}
\end{alltt}
\begin{verbatim}
## 'data.frame':	224921 obs. of  9 variables:
##  $ ID      : chr  "USC00043157" "USC00043157" "USC00043157" "USC00043157" ...
##  $ DATE    : int  18670601 18670602 18670603 18670604 18670605 18670606 18670607 18670608 18670609 18670610 ...
##  $ ELEMENT : chr  "PRCP" "PRCP" "PRCP" "PRCP" ...
##  $ VALUE   : num  0 0 0 0 0 0 0 0 0 0 ...
##  $ M-FLAG  : chr  "" "" "" "" ...
##  $ Q-FLAG  : chr  "" "" "" "" ...
##  $ S-FLAG  : chr  "F" "F" "F" "F" ...
##  $ OBS-TIME: int  NA NA NA NA NA NA NA NA NA NA ...
##  $ Ymd     : Date, format: "1867-06-01" "1867-06-02" ...
\end{verbatim}
\begin{alltt}
\hlcom{# Baseline Period 1961-1990}

\hlstd{station1.baseline} \hlkwb{=} \hlkwd{subset}\hlstd{(station1, Ymd} \hlopt{>=} \hlstr{"1961-01-01"} \hlopt{&} \hlstd{Ymd} \hlopt{<=} \hlstr{"1990-12-31"}\hlstd{)}
\hlstd{a}
\end{alltt}


{\ttfamily\noindent\bfseries\color{errorcolor}{\#\# Error in eval(expr, envir, enclos): object 'a' not found}}\begin{alltt}
\hlcom{# Monthly Averages}
\hlstd{station1}\hlopt{$}\hlstd{MONTH} \hlkwb{=} \hlkwd{as.numeric}\hlstd{(}\hlkwd{format}\hlstd{(station1}\hlopt{$}\hlstd{Ymd,} \hlstr{"%m"}\hlstd{))}
\hlstd{station1}\hlopt{$}\hlstd{YEAR} \hlkwb{=} \hlkwd{as.numeric}\hlstd{(}\hlkwd{format}\hlstd{(station1}\hlopt{$}\hlstd{Ymd,} \hlstr{"%Y"}\hlstd{))}
\hlstd{station1.monthly} \hlkwb{=} \hlkwd{aggregate}\hlstd{(VALUE} \hlopt{~} \hlstd{MONTH} \hlopt{+} \hlstd{YEAR,}
                             \hlkwc{data} \hlstd{=} \hlkwd{subset}\hlstd{(station1, ELEMENT} \hlopt{==} \hlstr{"TMAX"}\hlstd{), mean)}
\hlkwd{str}\hlstd{(station1.monthly)}
\end{alltt}
\begin{verbatim}
## 'data.frame':	1601 obs. of  3 variables:
##  $ MONTH: num  1 2 3 4 5 6 7 8 9 10 ...
##  $ YEAR : num  1870 1870 1870 1870 1870 1870 1870 1870 1870 1870 ...
##  $ VALUE: num  4.51 7.02 7.14 16.06 20.64 ...
\end{verbatim}
\begin{alltt}
\hlcom{# lm}
\hlkwa{for}\hlstd{(i} \hlkwa{in} \hlkwd{unique}\hlstd{(station1.monthly}\hlopt{$}\hlstd{MONTH))\{}
  \hlstd{temp.lm} \hlkwb{<-} \hlkwd{lm}\hlstd{(VALUE} \hlopt{~} \hlstd{YEAR,} \hlkwc{data} \hlstd{=} \hlkwd{subset}\hlstd{(station1.monthly, MONTH}\hlopt{==}\hlstd{i))}

\hlstd{\}}

\hlkwd{plot}\hlstd{(VALUE} \hlopt{~} \hlstd{YEAR,} \hlkwc{data} \hlstd{=} \hlkwd{subset}\hlstd{(station1.monthly, MONTH} \hlopt{==} \hlnum{1}\hlstd{),}
     \hlkwc{las}\hlstd{=}\hlnum{1}\hlstd{,} \hlkwc{pch}\hlstd{=}\hlnum{19}\hlstd{,} \hlkwc{col} \hlstd{=} \hlstr{"blue"}\hlstd{,} \hlkwc{cex}\hlstd{=}\hlnum{.5}\hlstd{)}
\hlkwd{abline}\hlstd{(}\hlkwd{coef}\hlstd{(temp.lm),} \hlkwc{col} \hlstd{=} \hlstr{"red"}\hlstd{)}
\end{alltt}
\end{kframe}
\includegraphics[width=\maxwidth]{figure/unnamed-chunk-6-1} 
\end{knitrout}




\end{document}
