\documentclass[11pt]{article}\usepackage[]{graphicx}\usepackage[]{xcolor}
% maxwidth is the original width if it is less than linewidth
% otherwise use linewidth (to make sure the graphics do not exceed the margin)
\makeatletter
\def\maxwidth{ %
  \ifdim\Gin@nat@width>\linewidth
    \linewidth
  \else
    \Gin@nat@width
  \fi
}
\makeatother

\definecolor{fgcolor}{rgb}{0.345, 0.345, 0.345}
\newcommand{\hlnum}[1]{\textcolor[rgb]{0.686,0.059,0.569}{#1}}%
\newcommand{\hlstr}[1]{\textcolor[rgb]{0.192,0.494,0.8}{#1}}%
\newcommand{\hlcom}[1]{\textcolor[rgb]{0.678,0.584,0.686}{\textit{#1}}}%
\newcommand{\hlopt}[1]{\textcolor[rgb]{0,0,0}{#1}}%
\newcommand{\hlstd}[1]{\textcolor[rgb]{0.345,0.345,0.345}{#1}}%
\newcommand{\hlkwa}[1]{\textcolor[rgb]{0.161,0.373,0.58}{\textbf{#1}}}%
\newcommand{\hlkwb}[1]{\textcolor[rgb]{0.69,0.353,0.396}{#1}}%
\newcommand{\hlkwc}[1]{\textcolor[rgb]{0.333,0.667,0.333}{#1}}%
\newcommand{\hlkwd}[1]{\textcolor[rgb]{0.737,0.353,0.396}{\textbf{#1}}}%
\let\hlipl\hlkwb

\usepackage{framed}
\makeatletter
\newenvironment{kframe}{%
 \def\at@end@of@kframe{}%
 \ifinner\ifhmode%
  \def\at@end@of@kframe{\end{minipage}}%
  \begin{minipage}{\columnwidth}%
 \fi\fi%
 \def\FrameCommand##1{\hskip\@totalleftmargin \hskip-\fboxsep
 \colorbox{shadecolor}{##1}\hskip-\fboxsep
     % There is no \\@totalrightmargin, so:
     \hskip-\linewidth \hskip-\@totalleftmargin \hskip\columnwidth}%
 \MakeFramed {\advance\hsize-\width
   \@totalleftmargin\z@ \linewidth\hsize
   \@setminipage}}%
 {\par\unskip\endMakeFramed%
 \at@end@of@kframe}
\makeatother

\definecolor{shadecolor}{rgb}{.97, .97, .97}
\definecolor{messagecolor}{rgb}{0, 0, 0}
\definecolor{warningcolor}{rgb}{1, 0, 1}
\definecolor{errorcolor}{rgb}{1, 0, 0}
\newenvironment{knitrout}{}{} % an empty environment to be redefined in TeX

\usepackage{alltt}
\usepackage{graphicx}
\usepackage{amsmath}
\usepackage{booktabs}
\usepackage{hyperref}

\title{Algal Growth Survival Analysis \& Event Modeling: A Detailed Handout}
\author{Marc Los Huertos}
\date{\today}
\IfFileExists{upquote.sty}{\usepackage{upquote}}{}
\begin{document}
\maketitle
\tableofcontents

\section{Goal and Justification}

The purpose of this handout is to compare the time it takes for algal cultures under three different treatments to reach a specific growth milestone (an absorbance threshold). Survival analysis is used because it correctly handles cultures that never reach the threshold (right-censoring) and provides robust, interpretable comparisons of growth rates over time.

\subsection{Why Survival Analysis is an important way to go!}

In simple terms, survival analysis focuses on speed. Instead of just looking at OD at a single time point, we look at the time-to-event.

\begin{itemize}
    \item The Event: Our chosen growth milestone (e.g., OD $\geq 0.6$).
    \item The Time: The hours it took to hit that milestone.
    \item The Magic (Censoring): If a culture is too slow and never reaches the threshold by the end of the 72-hour experiment, we don't throw it out. We simply record that the event did not occur (status = 0) by the time of the last observation. This avoids bias, giving us an honest assessment of growth rates.
\end{itemize}

\begin{center}
\begin{tabular}{lp{4cm}p{4cm}}
\toprule
Survival Term & Algae Equivalent & Statistical Role \\
\midrule
Event & Reaching OD threshold (e.g., OD = 0.6) & Defines the failure point we are tracking \\
Time & Hours until OD $\geq$ threshold & The variable we are modeling \\
Censoring & Not reaching threshold by experiment end & Allows us to use incomplete data \\
Hazard Ratio & Relative rate of reaching threshold & HR $>$ 1 means faster growth (speed) \\
\bottomrule
\end{tabular}
\end{center}

\section{Packages and Data Setup}

We first load the required packages. These provide tools for data manipulation (\texttt{tidyverse}), survival analysis (\texttt{survival}), visualization (\texttt{survminer}), and tidying model output (\texttt{broom}).

\begin{knitrout}
\definecolor{shadecolor}{rgb}{0.969, 0.969, 0.969}\color{fgcolor}\begin{kframe}
\begin{alltt}
\hlkwd{library}\hlstd{(tidyverse)}
\end{alltt}


{\ttfamily\noindent\itshape\color{messagecolor}{\#\# -- Attaching core tidyverse packages ------------------------ tidyverse 2.0.0 --\\\#\# v dplyr \ \ \ \ 1.1.4 \ \ \ \ v readr \ \ \ \ 2.1.5\\\#\# v forcats \ \ 1.0.0 \ \ \ \ v stringr \ \ 1.5.1\\\#\# v ggplot2 \ \ 3.4.4 \ \ \ \ v tibble \ \ \ 3.2.1\\\#\# v lubridate 1.9.3 \ \ \ \ v tidyr \ \ \ \ 1.3.1\\\#\# v purrr \ \ \ \ 1.0.2 \ \ \ \ \\\#\# -- Conflicts ------------------------------------------ tidyverse\_conflicts() --\\\#\# x dplyr::filter() masks stats::filter()\\\#\# x dplyr::lag() \ \ \ masks stats::lag()\\\#\# i Use the conflicted package (<http://conflicted.r-lib.org/>) to force all conflicts to become errors}}\begin{alltt}
\hlkwd{library}\hlstd{(survival)}
\hlkwd{library}\hlstd{(survminer)}
\end{alltt}


{\ttfamily\noindent\itshape\color{messagecolor}{\#\# Loading required package: ggpubr\\\#\# \\\#\# Attaching package: 'survminer'\\\#\# \\\#\# The following object is masked from 'package:survival':\\\#\# \\\#\# \ \ \ \ myeloma}}\begin{alltt}
\hlkwd{library}\hlstd{(broom)}
\end{alltt}
\end{kframe}
\end{knitrout}

\subsection{Data Simulation}

We simulate growth data for three treatments. The function \texttt{make\_growth} generates noisy logistic growth curves.

\begin{knitrout}
\definecolor{shadecolor}{rgb}{0.969, 0.969, 0.969}\color{fgcolor}\begin{kframe}
\begin{alltt}
\hlkwd{set.seed}\hlstd{(}\hlnum{2025}\hlstd{)}
\hlstd{n_rep} \hlkwb{<-} \hlnum{10}
\hlstd{times} \hlkwb{<-} \hlkwd{seq}\hlstd{(}\hlnum{0}\hlstd{,} \hlnum{72}\hlstd{,} \hlkwc{by} \hlstd{=} \hlnum{8}\hlstd{)}

\hlstd{make_growth} \hlkwb{<-} \hlkwa{function}\hlstd{(}\hlkwc{n}\hlstd{,} \hlkwc{treatment_label}\hlstd{,} \hlkwc{mu_time_to_mid} \hlstd{=} \hlnum{30}\hlstd{,} \hlkwc{sd_time} \hlstd{=} \hlnum{6}\hlstd{,} \hlkwc{maxOD} \hlstd{=} \hlnum{1.2}\hlstd{,} \hlkwc{prop_no_reach} \hlstd{=} \hlnum{0.1}\hlstd{)\{}
    \hlkwd{tibble}\hlstd{(}\hlkwc{sample} \hlstd{=} \hlkwd{paste0}\hlstd{(treatment_label,} \hlstr{"_"}\hlstd{,} \hlkwd{seq_len}\hlstd{(n)))} \hlopt
    \hlkwd{rowwise}\hlstd{()} \hlopt
    \hlkwd{mutate}\hlstd{(}
      \hlkwc{t_mid} \hlstd{=} \hlkwd{rnorm}\hlstd{(}\hlnum{1}\hlstd{, mu_time_to_mid, sd_time) |>} \hlkwd{pmax}\hlstd{(}\hlnum{6}\hlstd{),}
      \hlkwc{slope} \hlstd{= (maxOD)} \hlopt{/} \hlstd{(t_mid} \hlopt{+} \hlnum{0.1}\hlstd{),}
      \hlkwc{never} \hlstd{=} \hlkwd{runif}\hlstd{(}\hlnum{1}\hlstd{)} \hlopt{<} \hlstd{prop_no_reach}
    \hlstd{)} \hlopt
    \hlkwd{ungroup}\hlstd{()} \hlopt
    \hlkwd{expand_grid}\hlstd{(}\hlkwc{time} \hlstd{= times)} \hlopt
    \hlkwd{rowwise}\hlstd{()} \hlopt
    \hlkwd{mutate}\hlstd{(}
      \hlkwc{mu} \hlstd{=} \hlkwd{plogis}\hlstd{((time} \hlopt{-} \hlkwd{rnorm}\hlstd{(}\hlnum{1}\hlstd{, mu_time_to_mid, sd_time))}\hlopt{/}\hlnum{7}\hlstd{)} \hlopt{*} \hlstd{maxOD,}
      \hlkwc{absorbance} \hlstd{= mu} \hlopt{+} \hlkwd{rnorm}\hlstd{(}\hlnum{1}\hlstd{,} \hlnum{0}\hlstd{,} \hlnum{0.03}\hlstd{)}
    \hlstd{)} \hlopt
    \hlkwd{ungroup}\hlstd{()} \hlopt
    \hlkwd{mutate}\hlstd{(}\hlkwc{treatment} \hlstd{= treatment_label)}
\hlstd{\}}

\hlcom{# --- ENSURE THESE ADJUSTED VALUES ARE USED ---}
\hlcom{# Slower growth for A and B to create failures}
\hlstd{df_A} \hlkwb{<-} \hlkwd{make_growth}\hlstd{(n_rep,} \hlstr{"A"}\hlstd{,} \hlkwc{mu_time_to_mid} \hlstd{=} \hlnum{45}\hlstd{,} \hlkwc{sd_time} \hlstd{=} \hlnum{6}\hlstd{,} \hlkwc{maxOD} \hlstd{=} \hlnum{1.1}\hlstd{,} \hlkwc{prop_no_reach} \hlstd{=} \hlnum{0.2}\hlstd{)}
\hlstd{df_B} \hlkwb{<-} \hlkwd{make_growth}\hlstd{(n_rep,} \hlstr{"B"}\hlstd{,} \hlkwc{mu_time_to_mid} \hlstd{=} \hlnum{40}\hlstd{,} \hlkwc{sd_time} \hlstd{=} \hlnum{7}\hlstd{,} \hlkwc{maxOD} \hlstd{=} \hlnum{1.15}\hlstd{,} \hlkwc{prop_no_reach} \hlstd{=} \hlnum{0.2}\hlstd{)}
\hlstd{df_C} \hlkwb{<-} \hlkwd{make_growth}\hlstd{(n_rep,} \hlstr{"C"}\hlstd{,} \hlkwc{mu_time_to_mid} \hlstd{=} \hlnum{22}\hlstd{,} \hlkwc{sd_time} \hlstd{=} \hlnum{5}\hlstd{,} \hlkwc{maxOD} \hlstd{=} \hlnum{1.0}\hlstd{,} \hlkwc{prop_no_reach} \hlstd{=} \hlnum{0.08}\hlstd{)}

\hlstd{df_raw} \hlkwb{<-} \hlkwd{bind_rows}\hlstd{(df_A, df_B, df_C)} \hlopt
    \hlkwd{mutate}\hlstd{(}\hlkwc{treatment} \hlstd{=} \hlkwd{factor}\hlstd{(treatment))}
\end{alltt}
\end{kframe}
\end{knitrout}

\section{Visualization and Event Definition}

\subsection{Raw Growth Curves and Threshold}

We plot individual growth curves and treatment means. This shows variability and average trends. The dashed red line marks our target $\text{OD} = 0.6$ threshold.

\begin{knitrout}
\definecolor{shadecolor}{rgb}{0.969, 0.969, 0.969}\color{fgcolor}\begin{kframe}
\begin{alltt}
\hlstd{df_means} \hlkwb{<-} \hlstd{df_raw} \hlopt
    \hlkwd{group_by}\hlstd{(treatment, time)} \hlopt
    \hlkwd{summarise}\hlstd{(}\hlkwc{mean_abs} \hlstd{=} \hlkwd{mean}\hlstd{(absorbance),} \hlkwc{.groups} \hlstd{=} \hlstr{"drop"}\hlstd{)}

\hlstd{threshold} \hlkwb{<-} \hlnum{0.6}

\hlstd{p_raw} \hlkwb{<-} \hlkwd{ggplot}\hlstd{()} \hlopt{+}
    \hlkwd{geom_line}\hlstd{(}\hlkwc{data} \hlstd{= df_raw,}
              \hlkwd{aes}\hlstd{(time, absorbance,} \hlkwc{group} \hlstd{= sample,} \hlkwc{color} \hlstd{= treatment),}
              \hlkwc{alpha} \hlstd{=} \hlnum{0.2}\hlstd{,} \hlkwc{show.legend} \hlstd{=} \hlnum{FALSE}\hlstd{)} \hlopt{+}
    \hlkwd{geom_line}\hlstd{(}\hlkwc{data} \hlstd{= df_means,}
              \hlkwd{aes}\hlstd{(time, mean_abs,} \hlkwc{color} \hlstd{= treatment),}
              \hlkwc{linewidth} \hlstd{=} \hlnum{1.1}\hlstd{)} \hlopt{+}
    \hlkwd{geom_hline}\hlstd{(}\hlkwc{yintercept} \hlstd{= threshold,} \hlkwc{linetype} \hlstd{=} \hlstr{"dashed"}\hlstd{,} \hlkwc{color} \hlstd{=} \hlstr{"red"}\hlstd{,} \hlkwc{linewidth} \hlstd{=} \hlnum{0.8}\hlstd{)} \hlopt{+}
    \hlkwd{annotate}\hlstd{(}\hlstr{"text"}\hlstd{,} \hlkwc{x} \hlstd{=} \hlkwd{max}\hlstd{(df_raw}\hlopt{$}\hlstd{time)}\hlopt{*}\hlnum{0.7}\hlstd{,} \hlkwc{y} \hlstd{= threshold} \hlopt{+} \hlnum{0.04}\hlstd{,} \hlkwc{label} \hlstd{=} \hlkwd{paste0}\hlstd{(}\hlstr{"Threshold = "}\hlstd{, threshold),} \hlkwc{color} \hlstd{=} \hlstr{"red"}\hlstd{)} \hlopt{+}
    \hlkwd{labs}\hlstd{(}\hlkwc{x} \hlstd{=} \hlstr{"Time (hours)"}\hlstd{,}
         \hlkwc{y} \hlstd{=} \hlstr{"Absorbance (OD)"}\hlstd{,}
         \hlkwc{title} \hlstd{=} \hlstr{"Raw Growth Curves by Treatment with Target Threshold"}\hlstd{)} \hlopt{+}
    \hlkwd{theme_minimal}\hlstd{()}
\hlstd{p_raw}
\end{alltt}
\end{kframe}
\includegraphics[width=\maxwidth]{figure/rawcurves-1} 
\end{knitrout}

\section{Analysis 1: Threshold Crossing (Speed Analysis)}

\subsection{Defining the Event (Survival Data Setup)}

We define the event as the first time a culture reaches $\text{OD} \geq 0.6$. If a culture never reaches this threshold by the end of the growth experiment (72 hours), it is censored at $t=72$.

\begin{knitrout}
\definecolor{shadecolor}{rgb}{0.969, 0.969, 0.969}\color{fgcolor}\begin{kframe}
\begin{alltt}
\hlstd{time_to_event} \hlkwb{<-} \hlstd{df_raw} \hlopt
    \hlkwd{group_by}\hlstd{(sample, treatment)} \hlopt
    \hlkwd{arrange}\hlstd{(time)} \hlopt
    \hlkwd{summarise}\hlstd{(}
      \hlkwc{event_time} \hlstd{= \{}
        \hlstd{hit_rows} \hlkwb{<-} \hlkwd{which}\hlstd{(absorbance} \hlopt{>=} \hlstd{threshold)}
        \hlkwa{if}\hlstd{(}\hlkwd{length}\hlstd{(hit_rows)} \hlopt{==} \hlnum{0}\hlstd{)} \hlnum{NA_real_} \hlkwa{else} \hlstd{time[}\hlkwd{min}\hlstd{(hit_rows)]}
      \hlstd{\},}
      \hlkwc{last_time} \hlstd{=} \hlkwd{max}\hlstd{(time),}
      \hlkwc{.groups} \hlstd{=} \hlstr{"drop"}
    \hlstd{)} \hlopt
    \hlkwd{mutate}\hlstd{(}
      \hlkwc{status} \hlstd{=} \hlkwd{if_else}\hlstd{(}\hlkwd{is.na}\hlstd{(event_time),} \hlnum{0L}\hlstd{,} \hlnum{1L}\hlstd{),} \hlcom{# 1=Event Occurred, 0=Censored}
      \hlkwc{time} \hlstd{=} \hlkwd{if_else}\hlstd{(}\hlkwd{is.na}\hlstd{(event_time), last_time, event_time)}
    \hlstd{)}

\hlkwd{cat}\hlstd{(}\hlstr{"Censoring Status (0 = Censored, 1 = Event Occurred):\textbackslash{}n"}\hlstd{)}
\end{alltt}
\begin{verbatim}
## Censoring Status (0 = Censored, 1 = Event Occurred):
\end{verbatim}
\begin{alltt}
\hlkwd{print}\hlstd{(time_to_event} \hlopt \hlkwd{count}\hlstd{(treatment, status))}
\end{alltt}
\begin{verbatim}
## # A tibble: 3 x 3
##   treatment status     n
##   <fct>      <int> <int>
## 1 A              1    10
## 2 B              1    10
## 3 C              1    10
\end{verbatim}
\end{kframe}
\end{knitrout}

\subsection{Kaplan-Meier Curves: Visualizing Speed}

Kaplan-Meier curves estimate the probability of not yet reaching the threshold over time. Steeper, earlier drops mean faster growth. We use `surv.median.line = ``none'' to avoid the interpolation error encountered earlier.

\begin{knitrout}
\definecolor{shadecolor}{rgb}{0.969, 0.969, 0.969}\color{fgcolor}\begin{kframe}
\begin{alltt}
\hlstd{km_fit} \hlkwb{<-} \hlkwd{survfit}\hlstd{(}\hlkwd{Surv}\hlstd{(time, status)} \hlopt{~} \hlstd{treatment,} \hlkwc{data} \hlstd{= time_to_event)}

\hlkwd{ggsurvplot}\hlstd{(km_fit,} \hlkwc{data} \hlstd{= time_to_event,} \hlkwc{risk.table} \hlstd{=} \hlnum{TRUE}\hlstd{,} \hlkwc{pval} \hlstd{=} \hlnum{TRUE}\hlstd{,}
           \hlkwc{conf.int} \hlstd{=} \hlnum{TRUE}\hlstd{,} \hlkwc{palette} \hlstd{=} \hlstr{"Dark2"}\hlstd{,}
           \hlkwc{surv.median.line} \hlstd{=} \hlstr{"none"}\hlstd{,}
           \hlkwc{title} \hlstd{=} \hlstr{"Kaplan-Meier: Time to Reach Absorbance Threshold (OD 0.6)"}\hlstd{,}
           \hlkwc{xlab} \hlstd{=} \hlstr{"Time (hours)"}\hlstd{,} \hlkwc{legend.title} \hlstd{=} \hlstr{"Treatment"}\hlstd{,}
           \hlkwc{break.time.by} \hlstd{=} \hlnum{12}\hlstd{)}
\end{alltt}
\end{kframe}
\includegraphics[width=\maxwidth]{figure/km-1} 
\end{knitrout}

\subsection{Cox Proportional Hazards Model: Quantifying Speed}

The Cox model estimates Hazard Ratios (HR), which quantify the relative speed of hitting the threshold. We use Treatment A as the reference group.

\begin{itemize}
  \item  $\text{HR} > 1$: Treatment reaches the threshold faster than A.
  \item  $\text{HR} < 1$: Treatment reaches the threshold slower than A.
\end{itemize}

\begin{knitrout}
\definecolor{shadecolor}{rgb}{0.969, 0.969, 0.969}\color{fgcolor}\begin{kframe}
\begin{alltt}
\hlstd{time_to_event} \hlkwb{<-} \hlstd{time_to_event} \hlopt \hlkwd{mutate}\hlstd{(}\hlkwc{treatment} \hlstd{=} \hlkwd{relevel}\hlstd{(treatment,} \hlkwc{ref} \hlstd{=} \hlstr{"A"}\hlstd{))}
\hlstd{cox1} \hlkwb{<-} \hlkwd{coxph}\hlstd{(}\hlkwd{Surv}\hlstd{(time, status)} \hlopt{~} \hlstd{treatment,} \hlkwc{data} \hlstd{= time_to_event)}
\hlkwd{summary}\hlstd{(cox1)}
\end{alltt}
\begin{verbatim}
## Call:
## coxph(formula = Surv(time, status) ~ treatment, data = time_to_event)
## 
##   n= 30, number of events= 30 
## 
##               coef exp(coef) se(coef)     z Pr(>|z|)    
## treatmentB  1.0199    2.7729   0.5053 2.019   0.0435 *  
## treatmentC  2.7386   15.4646   0.6127 4.470 7.83e-06 ***
## ---
## Signif. codes:  0 '***' 0.001 '**' 0.01 '*' 0.05 '.' 0.1 ' ' 1
## 
##            exp(coef) exp(-coef) lower .95 upper .95
## treatmentB     2.773    0.36063     1.030     7.465
## treatmentC    15.465    0.06466     4.654    51.387
## 
## Concordance= 0.817  (se = 0.032 )
## Likelihood ratio test= 21.02  on 2 df,   p=3e-05
## Wald test            = 20.34  on 2 df,   p=4e-05
## Score (logrank) test = 26.43  on 2 df,   p=2e-06
\end{verbatim}
\end{kframe}
\end{knitrout}

\subsection{Hazard Ratio Forest Plot}

This plot visualizes the HR estimates and confidence intervals (CI). If the CI for an HR does not cross 1 (the blue dashed line), the difference in speed is statistically significant.

\begin{knitrout}
\definecolor{shadecolor}{rgb}{0.969, 0.969, 0.969}\color{fgcolor}\begin{kframe}
\begin{alltt}
\hlstd{hr} \hlkwb{<-} \hlkwd{tidy}\hlstd{(cox1,} \hlkwc{exponentiate} \hlstd{=} \hlnum{TRUE}\hlstd{,} \hlkwc{conf.int} \hlstd{=} \hlnum{TRUE}\hlstd{)}

\hlkwd{ggplot}\hlstd{(hr,} \hlkwd{aes}\hlstd{(}\hlkwc{x} \hlstd{= term,} \hlkwc{y} \hlstd{= estimate))} \hlopt{+}
    \hlkwd{geom_point}\hlstd{()} \hlopt{+}
    \hlkwd{geom_errorbar}\hlstd{(}\hlkwd{aes}\hlstd{(}\hlkwc{ymin} \hlstd{= conf.low,} \hlkwc{ymax} \hlstd{= conf.high),} \hlkwc{width} \hlstd{=} \hlnum{0.1}\hlstd{)} \hlopt{+}
    \hlkwd{geom_hline}\hlstd{(}\hlkwc{yintercept} \hlstd{=} \hlnum{1}\hlstd{,} \hlkwc{linetype} \hlstd{=} \hlstr{"dashed"}\hlstd{,} \hlkwc{color} \hlstd{=} \hlstr{"blue"}\hlstd{)} \hlopt{+}
    \hlkwd{coord_flip}\hlstd{()} \hlopt{+}
    \hlkwd{labs}\hlstd{(}\hlkwc{y} \hlstd{=} \hlstr{"Hazard Ratio (HR)"}\hlstd{,} \hlkwc{x} \hlstd{=} \hlstr{"Coefficient"}\hlstd{,} \hlkwc{title} \hlstd{=} \hlstr{"Cox Model: HR and 95% CI (Ref: Treatment A)"}\hlstd{)} \hlopt{+}
    \hlkwd{theme_minimal}\hlstd{()}
\end{alltt}
\end{kframe}
\includegraphics[width=\maxwidth]{figure/forest-1} 
\end{knitrout}

\subsection{Assumptions Check: Proportional Hazards}

The Cox model assumes the HR is constant over time. A non-significant p-value ($\text{p} > 0.05$) from the Schoenfeld residuals test indicates the assumption holds.

\begin{knitrout}
\definecolor{shadecolor}{rgb}{0.969, 0.969, 0.969}\color{fgcolor}\begin{kframe}
\begin{alltt}
\hlstd{ph_test} \hlkwb{<-} \hlkwd{cox.zph}\hlstd{(cox1)}
\hlstd{ph_test}
\end{alltt}
\begin{verbatim}
##           chisq df    p
## treatment  1.74  2 0.42
## GLOBAL     1.74  2 0.42
\end{verbatim}
\end{kframe}
\end{knitrout}

---

\section{Analysis 2: End of Event (Success Probability Analysis)}

In this alternative analysis, we ignore when the event happened and only ask: Did the culture succeed in reaching $\text{OD} \geq 0.6$ by the time the experiment ended (72 hours)?

This shifts our focus from speed to final probability of success. Since the outcome is binary (Success=1 or Failure=0), we use Logistic Regression.

\subsection{Event Definition (Logistic Data Setup)}

We filter the data to the final time point and create a binary success variable.

\begin{knitrout}
\definecolor{shadecolor}{rgb}{0.969, 0.969, 0.969}\color{fgcolor}\begin{kframe}
\begin{alltt}
\hlstd{final_data} \hlkwb{<-} \hlstd{df_raw} \hlopt
    \hlkwd{filter}\hlstd{(time} \hlopt{==} \hlkwd{max}\hlstd{(time))} \hlopt
    \hlkwd{mutate}\hlstd{(}
        \hlkwc{success} \hlstd{=} \hlkwd{if_else}\hlstd{(absorbance} \hlopt{>=} \hlstd{threshold,} \hlnum{1}\hlstd{,} \hlnum{0}\hlstd{),}
        \hlkwc{treatment} \hlstd{=} \hlkwd{relevel}\hlstd{(treatment,} \hlkwc{ref} \hlstd{=} \hlstr{"A"}\hlstd{)}
    \hlstd{)}

\hlkwd{cat}\hlstd{(}\hlstr{"Success/Failure Status at 72 Hours:\textbackslash{}n"}\hlstd{)}
\end{alltt}
\begin{verbatim}
## Success/Failure Status at 72 Hours:
\end{verbatim}
\begin{alltt}
\hlkwd{print}\hlstd{(final_data} \hlopt \hlkwd{count}\hlstd{(treatment, success))}
\end{alltt}
\begin{verbatim}
## # A tibble: 3 x 3
##   treatment success     n
##   <fct>       <dbl> <int>
## 1 A               1    10
## 2 B               1    10
## 3 C               1    10
\end{verbatim}
\end{kframe}
\end{knitrout}

\subsection{Logistic Regression Model}

The model estimates Odds Ratios (OR), which quantify the relative odds of a culture achieving $\text{OD} \geq 0.6$ at the end of the experiment compared to the reference group (Treatment A).

 $\text{OR} > 1$: Treatment has higher odds of success than A.
 $\text{OR} < 1$: Treatment has lower odds of success than A.

\begin{knitrout}
\definecolor{shadecolor}{rgb}{0.969, 0.969, 0.969}\color{fgcolor}\begin{kframe}
\begin{alltt}
\hlstd{logis_model} \hlkwb{<-} \hlkwd{glm}\hlstd{(success} \hlopt{~} \hlstd{treatment,} \hlkwc{data} \hlstd{= final_data,} \hlkwc{family} \hlstd{=} \hlstr{"binomial"}\hlstd{)}
\hlkwd{summary}\hlstd{(logis_model)}
\end{alltt}
\begin{verbatim}
## 
## Call:
## glm(formula = success ~ treatment, family = "binomial", data = final_data)
## 
## Deviance Residuals: 
##       Min         1Q     Median         3Q        Max  
## 3.971e-06  3.971e-06  3.971e-06  3.971e-06  3.971e-06  
## 
## Coefficients:
##               Estimate Std. Error z value Pr(>|z|)
## (Intercept)  2.557e+01  6.831e+04       0        1
## treatmentB  -1.759e-09  9.660e+04       0        1
## treatmentC  -1.759e-09  9.660e+04       0        1
## 
## (Dispersion parameter for binomial family taken to be 1)
## 
##     Null deviance: 0.000e+00  on 29  degrees of freedom
## Residual deviance: 4.731e-10  on 27  degrees of freedom
## AIC: 6
## 
## Number of Fisher Scoring iterations: 24
\end{verbatim}
\end{kframe}
\end{knitrout}

\subsection{Odds Ratio Plot}

We visualize the OR estimates and their confidence intervals. If the CI does not cross 1, the difference in the odds of success is statistically significant.

\begin{knitrout}
\definecolor{shadecolor}{rgb}{0.969, 0.969, 0.969}\color{fgcolor}\begin{kframe}
\begin{alltt}
\hlstd{or_results} \hlkwb{<-} \hlkwd{tidy}\hlstd{(logis_model,} \hlkwc{exponentiate} \hlstd{=} \hlnum{TRUE}\hlstd{,} \hlkwc{conf.int} \hlstd{=} \hlnum{TRUE}\hlstd{)} \hlopt
    \hlkwd{filter}\hlstd{(term} \hlopt{!=} \hlstr{"(Intercept)"}\hlstd{)} \hlcom{# Remove the intercept for the plot}
\end{alltt}


{\ttfamily\noindent\color{warningcolor}{\#\# Warning: glm.fit: fitted probabilities numerically 0 or 1 occurred}}

{\ttfamily\noindent\color{warningcolor}{\#\# Warning: glm.fit: fitted probabilities numerically 0 or 1 occurred}}

{\ttfamily\noindent\color{warningcolor}{\#\# Warning: glm.fit: fitted probabilities numerically 0 or 1 occurred}}

{\ttfamily\noindent\color{warningcolor}{\#\# Warning: glm.fit: fitted probabilities numerically 0 or 1 occurred}}

{\ttfamily\noindent\color{warningcolor}{\#\# Warning: glm.fit: fitted probabilities numerically 0 or 1 occurred}}

{\ttfamily\noindent\color{warningcolor}{\#\# Warning: glm.fit: fitted probabilities numerically 0 or 1 occurred}}

{\ttfamily\noindent\color{warningcolor}{\#\# Warning: glm.fit: fitted probabilities numerically 0 or 1 occurred}}

{\ttfamily\noindent\color{warningcolor}{\#\# Warning: glm.fit: fitted probabilities numerically 0 or 1 occurred}}

{\ttfamily\noindent\color{warningcolor}{\#\# Warning: glm.fit: fitted probabilities numerically 0 or 1 occurred}}

{\ttfamily\noindent\color{warningcolor}{\#\# Warning: glm.fit: fitted probabilities numerically 0 or 1 occurred}}

{\ttfamily\noindent\color{warningcolor}{\#\# Warning: glm.fit: fitted probabilities numerically 0 or 1 occurred}}

{\ttfamily\noindent\color{warningcolor}{\#\# Warning: glm.fit: fitted probabilities numerically 0 or 1 occurred}}

{\ttfamily\noindent\color{warningcolor}{\#\# Warning: glm.fit: fitted probabilities numerically 0 or 1 occurred}}

{\ttfamily\noindent\color{warningcolor}{\#\# Warning: glm.fit: fitted probabilities numerically 0 or 1 occurred}}

{\ttfamily\noindent\color{warningcolor}{\#\# Warning: glm.fit: fitted probabilities numerically 0 or 1 occurred}}

{\ttfamily\noindent\color{warningcolor}{\#\# Warning: glm.fit: fitted probabilities numerically 0 or 1 occurred}}

{\ttfamily\noindent\color{warningcolor}{\#\# Warning: glm.fit: fitted probabilities numerically 0 or 1 occurred}}

{\ttfamily\noindent\color{warningcolor}{\#\# Warning: glm.fit: fitted probabilities numerically 0 or 1 occurred}}

{\ttfamily\noindent\color{warningcolor}{\#\# Warning: glm.fit: fitted probabilities numerically 0 or 1 occurred}}

{\ttfamily\noindent\color{warningcolor}{\#\# Warning: glm.fit: fitted probabilities numerically 0 or 1 occurred}}

{\ttfamily\noindent\color{warningcolor}{\#\# Warning: glm.fit: fitted probabilities numerically 0 or 1 occurred}}

{\ttfamily\noindent\color{warningcolor}{\#\# Warning: glm.fit: fitted probabilities numerically 0 or 1 occurred}}

{\ttfamily\noindent\color{warningcolor}{\#\# Warning: glm.fit: fitted probabilities numerically 0 or 1 occurred}}

{\ttfamily\noindent\color{warningcolor}{\#\# Warning: glm.fit: fitted probabilities numerically 0 or 1 occurred}}

{\ttfamily\noindent\color{warningcolor}{\#\# Warning: glm.fit: fitted probabilities numerically 0 or 1 occurred}}

{\ttfamily\noindent\color{warningcolor}{\#\# Warning: glm.fit: fitted probabilities numerically 0 or 1 occurred}}

{\ttfamily\noindent\color{warningcolor}{\#\# Warning: glm.fit: fitted probabilities numerically 0 or 1 occurred}}

{\ttfamily\noindent\color{warningcolor}{\#\# Warning: glm.fit: fitted probabilities numerically 0 or 1 occurred}}

{\ttfamily\noindent\color{warningcolor}{\#\# Warning: glm.fit: fitted probabilities numerically 0 or 1 occurred}}

{\ttfamily\noindent\color{warningcolor}{\#\# Warning: glm.fit: fitted probabilities numerically 0 or 1 occurred}}

{\ttfamily\noindent\color{warningcolor}{\#\# Warning: glm.fit: fitted probabilities numerically 0 or 1 occurred}}

{\ttfamily\noindent\color{warningcolor}{\#\# Warning: glm.fit: fitted probabilities numerically 0 or 1 occurred}}

{\ttfamily\noindent\color{warningcolor}{\#\# Warning: glm.fit: fitted probabilities numerically 0 or 1 occurred}}

{\ttfamily\noindent\color{warningcolor}{\#\# Warning: glm.fit: fitted probabilities numerically 0 or 1 occurred}}

{\ttfamily\noindent\color{warningcolor}{\#\# Warning: glm.fit: fitted probabilities numerically 0 or 1 occurred}}

{\ttfamily\noindent\color{warningcolor}{\#\# Warning: glm.fit: fitted probabilities numerically 0 or 1 occurred}}

{\ttfamily\noindent\color{warningcolor}{\#\# Warning: glm.fit: fitted probabilities numerically 0 or 1 occurred}}

{\ttfamily\noindent\color{warningcolor}{\#\# Warning: glm.fit: fitted probabilities numerically 0 or 1 occurred}}

{\ttfamily\noindent\color{warningcolor}{\#\# Warning: glm.fit: fitted probabilities numerically 0 or 1 occurred}}

{\ttfamily\noindent\color{warningcolor}{\#\# Warning: glm.fit: fitted probabilities numerically 0 or 1 occurred}}

{\ttfamily\noindent\color{warningcolor}{\#\# Warning: glm.fit: fitted probabilities numerically 0 or 1 occurred}}

{\ttfamily\noindent\color{warningcolor}{\#\# Warning: glm.fit: fitted probabilities numerically 0 or 1 occurred}}

{\ttfamily\noindent\color{warningcolor}{\#\# Warning: glm.fit: fitted probabilities numerically 0 or 1 occurred}}

{\ttfamily\noindent\color{warningcolor}{\#\# Warning: glm.fit: fitted probabilities numerically 0 or 1 occurred}}

{\ttfamily\noindent\color{warningcolor}{\#\# Warning: glm.fit: fitted probabilities numerically 0 or 1 occurred}}

{\ttfamily\noindent\color{warningcolor}{\#\# Warning: glm.fit: fitted probabilities numerically 0 or 1 occurred}}

{\ttfamily\noindent\color{warningcolor}{\#\# Warning in regularize.values(x, y, ties, missing(ties), na.rm = na.rm): collapsing to unique 'x' values}}

{\ttfamily\noindent\bfseries\color{errorcolor}{\#\# Error in approx(sp\$y, sp\$x, xout = cutoff): need at least two non-NA values to interpolate}}\begin{alltt}
\hlkwd{ggplot}\hlstd{(or_results,} \hlkwd{aes}\hlstd{(}\hlkwc{x} \hlstd{= term,} \hlkwc{y} \hlstd{= estimate))} \hlopt{+}
    \hlkwd{geom_point}\hlstd{()} \hlopt{+}
    \hlkwd{geom_errorbar}\hlstd{(}\hlkwd{aes}\hlstd{(}\hlkwc{ymin} \hlstd{= conf.low,} \hlkwc{ymax} \hlstd{= conf.high),} \hlkwc{width} \hlstd{=} \hlnum{0.1}\hlstd{)} \hlopt{+}
    \hlkwd{geom_hline}\hlstd{(}\hlkwc{yintercept} \hlstd{=} \hlnum{1}\hlstd{,} \hlkwc{linetype} \hlstd{=} \hlstr{"dashed"}\hlstd{,} \hlkwc{color} \hlstd{=} \hlstr{"blue"}\hlstd{)} \hlopt{+}
    \hlkwd{coord_flip}\hlstd{()} \hlopt{+}
    \hlkwd{labs}\hlstd{(}\hlkwc{y} \hlstd{=} \hlstr{"Odds Ratio (OR)"}\hlstd{,} \hlkwc{x} \hlstd{=} \hlstr{"Coefficient"}\hlstd{,} \hlkwc{title} \hlstd{=} \hlstr{"Logistic Model: Odds of Success at 72 hrs (Ref: Treatment A)"}\hlstd{)} \hlopt{+}
    \hlkwd{theme_minimal}\hlstd{()}
\end{alltt}


{\ttfamily\noindent\bfseries\color{errorcolor}{\#\# Error in ggplot(or\_results, aes(x = term, y = estimate)): object 'or\_results' not found}}\end{kframe}
\end{knitrout}

\end{document}
