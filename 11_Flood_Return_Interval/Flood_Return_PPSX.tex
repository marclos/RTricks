\documentclass{beamer}
\usepackage{graphicx}
\usepackage{amsmath}
\usepackage{hyperref}

\title{Flood Return Interval Analysis}
\author{EA30}
\date{\today}

\begin{document}

\frame{\titlepage}

\begin{frame}{Introduction}
\begin{itemize}
\item Flood return interval analysis is used in hydrology to estimate the likelihood of flood events.
\item It helps in infrastructure planning, risk assessment, and environmental studies.
\item Based on statistical analysis of historical flood data
\item Key concepts include return period, annual exceedance probability, and plotting position.
\end{itemize}
\end{frame}

\begin{frame}{Key Definitions}
\begin{itemize}
\item \textbf{Return Period (T):} The average interval of time between floods of a certain magnitude or greater.
\item \textbf{Annual Exceedance Probability (AEP):} The probability that a flood of a certain magnitude will be equaled or exceeded in any given year.
\end{itemize}
\vspace{0.5cm}
\[
  T = \frac{1}{P}
  \quad \text{or} \quad
  P = \frac{1}{T}
  \]
\end{frame}

\begin{frame}{Data Requirements}
\begin{itemize}
\item Annual peak discharge data for a river or stream.
\item Data should span multiple decades for reliable analysis.
\item Quality control and consistency checks are essential.
\end{itemize}
\end{frame}

\begin{frame}{Plotting Position Formula}
\begin{itemize}
\item Used to estimate return periods from ranked data.
\item Common formula: \textbf{Weibull's Formula}
        \[
        T = \frac{n+1}{m}
        \]
        where:
        \begin{itemize}
            \item $n$ = number of years of record
            \item $m$ = rank of the flood (1 = highest)
        \end{itemize}
    \end{itemize}
\end{frame}

\begin{frame}{Example Calculation}
    \begin{itemize}
        \item 10 years of peak discharge data
        \item Rank the largest flood (m=1)
        \item Return period: 
        \[
        T = \frac{10 + 1}{1} = 11 \text{ years}
        \]
        \item This flood has a 1 in 11 chance (9.1\%) of occurring each year.
    \end{itemize}
\end{frame}

\begin{frame}{Limitations}
    \begin{itemize}
        \item Assumes stationarity (climate does not change over time).
        \item Sensitive to outliers and record length.
        \item Doesn't predict exact timing, only probability.
    \end{itemize}
\end{frame}

\begin{frame}{Applications}
    \begin{itemize}
        \item Designing bridges, culverts, and levees.
        \item Floodplain zoning and insurance.
        \item Climate change impact studies.
    \end{itemize}
\end{frame}

\begin{frame}{Conclusion}
    \begin{itemize}
        \item Flood return interval analysis is a critical tool in hydrology.
        \item Helps quantify flood risk based on historical data.
        \item Should be combined with modern models and climate data for future projections.
    \end{itemize}
\end{frame}

\begin{frame}{References}
    \begin{itemize}
        \item Chow, V.T., Maidment, D.R., and Mays, L.W. (1988). \textit{Applied Hydrology}.
        \item USGS Water Resources: \url{https://water.usgs.gov/}
    \end{itemize}
\end{frame}


\begin{frame}{Analysis Steps}
    \begin{itemize}
        \item Select Site from USGS database.
        \item Make sure it has a long record, +30 years.
        \item Create new folder in Rstudio Server
        \item Upload data to R
        \item Download marc's R functions from github.com
        \item Upload marc's R script functions to Rstudio
    \end{itemize}
\end{frame}


\begin{frame}[fragile]{Analysis Steps}
    \begin{itemize}
        \item Create Rmd File -- type all your code into the Rdm file!
        \item get path for R functions
        \item Add R block that includes
    \begin{verbatim}
    source("path/return_interval_functions.R")
    \end{verbatim}
        \item assign mysite (use quotes!)
        \item assign Q\_max (as highest peak flow of interest in 2025
        \item obtain peak flow data using function
        \item plot the data using function
        \item if time, split data and plot two time frames
    \end{itemize}
\end{frame}

\end{document}